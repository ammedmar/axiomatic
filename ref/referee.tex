\documentclass{amsart}
\input{../aux/style}
\input{../aux/usualcmds}
\addbibresource{../aux/usualpapers.bib}

%%%%%%%%%%%%%%%%%%%%%%
% !TEX root = ../axiomatic.tex

% Write new commands below
\newcommand{\F}{\Ftwo}
\renewcommand{\P}{\mathrm{P}}
\newcommand{\D}{\mathrm{D}}
\newcommand{\p}{\mathrm{p}}
\newcommand{\q}{\mathrm{q}}
\DeclareMathOperator*{\displaytensor}{\otimes}
\let\union\cup
\renewcommand{\cup}{\smallsmile}
\DeclareMathOperator{\cbd}{\delta}
\newcommand{\EZop}{\cZ}
\newcommand{\Med}{\cM}
\newcommand{\UM}{\forget(\cM)}
\renewcommand{\simplex}{\mathbb{\Delta}}
\newcommand{\barxi}{{\bar{\xi}}}
\newcommand{\bareta}{{\bar{\eta}}}
\newcommand{\barzeta}{{\bar{\zeta}}}
\DeclareMathOperator{\TR}{TR}

%\newcommand{\oL}{\overline{L}}
%\newcommand{\oR}{\overline{R}}

%Tabbing Environment
\newenvironment{tab}{\list{}{\rightmargin 0pt}\item\relax}{\endlist} % add commands here
\addbibresource{../aux/bibliography.bib} % add references here
\usepackage{enumitem}
\setlist{label=\arabic{enumi}.,itemsep=\medskipamount, left=0pt}

%%%%%%%%%%%%%%%%%%%%%%
\title[Referee reply]{REFEREE REPLY \\ An axiomatic characterization of Steenrod cup-$i$ products}

\newcommand{\ar}{\medskip\noindent\textit{Reply}:\ }
\renewcommand{\thesection}{\arabic{section}}

\begin{document}
	\noindent\today
	\maketitle

	We would like to thank the referee for a careful and insightful analysis of our paper, and for the many suggestions improving its presentation.
	We copy their report for completeness.

	\section{Reviewer's summary}

	The Steenrod squares in the cohomology of a simplicial complex were first defined by Steenrod from a collection of bilinear operations called cup-i products.
	These operations were later revisited and generalized to the odd primes in the work of McClure-Smith and Berger-Fresse. At the same time, Real gave another	very natural construction of cup-i products. More recently, Medina-Mardones	has given two new treatments to these operations: one from the viewpoint of props and another more handicrafted.
	Are all these constructions of cup-i products isomorphic? in this paper	Medina-Mardones answers in the affirmative by providing three natural-looking
	properties that characterize the cup-i constructions up to isomorphism (just as	Steenrod squares are characterized by four axioms).
	The article is well-written and clear.
	Nonetheless, there are some comments that need to be addressed by the author.

	\section{Reviewer's individual items}

	\begin{enumerate}
		\item Definition 4: the operation \(\cup_i\) is defined on cochains, but in the definition it is applied to chains.

		\ar To improve the clarity of the definition we have added the following sentence: ``In the following definition we identify a simplex with its dual cochain."

		\item Section 3.1, first paragraph: that is natural \(\rightarrow\) that are natural.

		\ar Changed as suggested.

		\item Lemma 14: “from either V or W”: it is not enough clear that either they are missing from V or they are missing from W.

		\ar Improved to ``if an only if $j$ and $j+1$ are both missing from $V$ or $W$."

		\item Lemma 17: “an element”: an element of which set precisely?

		\ar Added ``in $\cP(\simplex^n)^{\ot 2}_{i+n}$ with each $V_\lambda \ot W_\lambda$ in its basis"

		\item Lemma 17: the sudden appearance of \(\Lambda(i, n)\) is slightly disturbing for the reader, who has not seen it defined before.

		\ar This indexing set is now introduced in the statement of the Lemma.

		\item Lemma 17, proof of (1): how is naturality used to prove the statement?

		\ar We replaced the proof with the following more explicit version:

		\noindent (1) This follows from the fact that $\chains(\gsimplex^0)^{\ot 2}$ is 1-dimensional and generated by $[0] \ot [0] = \Psi_0^{\ot 2} (\emptyset \ot \emptyset)$.
		Explicitly, if $x$ is a $0$-simplex with dual basis element denoted $x^\vee$, the only two options for $x^\vee \cup_0 x^\vee$ are $x^\vee$ or $0$, with the first holding if and only if $\triangle_0[0] = \emptyset \ot \emptyset$.

		\item Lemma 19: is the partition unique?

		\ar It turns out to be, but that is proven later, in the proof of Lemma 21.
		In lemma 19 we do not claim or prove the unicity of the partition.

		\item Lemma 20: at this point the reader might not remember that \(\Delta\) is the canonical cup product, while \(\triangle\) is the one being inspected. In fact, he might not have noticed that a new symbol was introduced in Lemma 16.

		\ar Added a reminder in the statement.

		\item Section 4.7: there is an \(r\) that has not been declared before.

		\ar Typo: $r$ was supposed to be $k$. It is now fixed.

		\item Section 5.1, first paragraph: Lemma 8 or Lemma 9?

		\ar It is both and it is now fixed.

		\item Theorem 28 is not a side theorem of the paper: it is mentioned in the abstract. Therefore, it deserves a proper proof: part of the justification of the main theorem of the article is that it allows a simpler proof of the equivalence of Steenrod’s and Real’s constructions than a direct comparison. But that is not apparent with the current writing.

		\ar A complete proof has been added.

		\item Section 6: Recall that an \(E_\infty\)-operad is an operad with the same homology as \(Z\) whose symmetric action on each arity is free. I think an \(E_\infty\)-operad needs to be projective or free as well.

		\ar The standard definition, found for example in May, is that an $E_\infty$-operad is free as a symmetric module and not necessarily as an operad or cofibrant in a model category of operads.

		\item Page 18, first paragraph: you say that the surjection operad is the image of \(U(M)\) in \(Z\).
		Isn’t that the point of Appendix 1 in “A finitely presented \(E_\infty\)-prop I: algebraic context”. If so, it would be good to quote that appendix.

		\ar Changed as suggested.

		\item Apart from these comments, here is a suggestion: it would be interesting to give examples of cup-\(i\) product constructions that fail to satisfy any of the three axioms (free, non-degenerate, and irreducible).

		\ar NOT DONE
	\end{enumerate}

	\section{Other changes}

	\begin{enumerate}
		\item Author's affiliation updated.
		\item Thanks to the referee added to the acknowledgments.
	\end{enumerate}
\end{document}