% !TEX root = ../axiomatic.tex

\section{Introduction} \label{s:introduction}

In \cite{steenrod1947products}, Steenrod introduced by means of formulas the \mbox{cup-$i$} products on the cochains of spaces.
These bilinear maps give rise to Steenrod squares, cohomology operations
\[
\Sq^k \colon \rH^\vee(X; \F) \to \rH^\vee(X; \F)
\]
laying at the heart of stable homotopy theory.

Steenrod's formulas for the \mbox{cup-$i$} products extend the Alexander-Whitney product on cochains.
This non-commutative product $\cup_0$ induces the commutative algebra structure on cohomology, and we can interpret the higher \mbox{cup-$i$} products
\[
\cup_i\ \colon \cochains(X; \F) \tensor \cochains(X; \F) \to \cochains(X; \F)
\]
as coherent homotopies witnessing the derived commutativity of $\cup_0$ at the cochain level.

In later work by Steenrod \cite{steenrod1962cohomology} and others, an indirect argument based on the acyclic carrier theorem is used to establish the existence of \mbox{cup-$i$} products and consequently of Steenrod squares.
This approach became the standard since any set of choices for the \mbox{cup-$i$} products homotopic to Steenrod's original gives rise to the same cohomology operations which, by then, had been axiomatically characterized.
As a consequence, the need to interact with a specific set of choices for the \mbox{cup-$i$} products largely declined.

Interest in actual cochain representatives and operations, like the \mbox{cup-$i$} products, has recently been rekindled by their use in the study of topological phases of matter \cite{gaiotto2016spin, kapustin2017fermionic, barkeshli2021classification} (see also \cite{brumfiel2016pontrjagin,brumfiel2018pontrjagin})
and topological data analysis \cite{carlsson2005barcode, medina2018persistence}.
Additionally, the combinatorics of Steenrod's cup-$i$ products has been related to higher category theory by deducing the nerve construction from them \cite{street1987orientals, medina2020globular}, and their geometry has been explored using fibrations of convex polytopes \cite{medina2022fib_poly}.

In the present work we introduce an axiomatic characterization of Steenrod's original cup-$i$ construction up to isomorphism -- not just homotopy.
Loosely expressed, our axioms demand that a cup-$i$ construction be natural with respect to simplicial maps, parameterized by the smallest possible complex, not identically $0$ or reducible to subsimplices, and as free as possible with respect to transpositions.

The relevance of this result is twofold.
On the one hand, as we overview below, it clarifies that all available cup-$i$ constructions represent the same isomorphism class.
At least for two of these, no previous proof of the equivalence with Steenrod's existed in print.
On the other hand, it highlights the fundamental nature of Steenrod's \mbox{cup-$i$} construction illuminating its unexpected connections with convex geometry and higher category theory.

The first alternative construction of \mbox{cup-$i$} products is due to Real \cite{real1996computability} and it is based on the Eilenberg--Zilber contraction.
It was further developed by Gonz\'alez-D\'iaz--Real \cite{gonzalez-diaz1999steenrod} and used as the basis of an algorithm for the computation of Steenrod squares implementated by Palmieri on \texttt{SAGE}.\footnote{This project is available at \url{https://www.sagemath.org/}}
We will review this construction in \cref{ss:real} and use our axioms to show it is isomorphic -- in fact equal -- to Steenrod's original.

In \cite{medina2021fast_sq}, new formulas defining a \mbox{cup-$i$} construction were introduced and used to present a faster algorithm for the computation of Steenrod squares for finite simplicial complexes.
This algorithm was implemented in the package \texttt{steenroder} computing persistence Steenrod barcodes for topological data analysis.\footnote{This project is currently hosted at \url{https://github.com/Steenroder/steenroder}}
The question of comparing the resulting \mbox{cup-$i$} construction with either Steenrod's or Gonz\'alez-D\'iaz--Real's was not addressed.
We use our axiomatic characterization to show in \cref{ss:original} that these agree.

A generalization of the notion of \mbox{cup-$i$} product structure is that of $E_\infty$-algebra.
Using work by \cite[\subsectionSymbol4.5]{benson1998representations} generalizing Steenrod's formulas, McClure--Smith \cite{mcclure2003multivariable} and Berger--Fresse \cite{berger2004combinatorial} constructed natural $E_\infty$-algebra structures on the normalized cochains of simplicial sets.
As we describe in \cref{ss:operads}, these structures induce a bijection between the set of bases of the arity 2 part of their operads and the set of \mbox{cup-$i$} constructions satisfying our axioms.

In \cite{medina2020prop1}, the normalized chains of standard simplicial sets where shown to be equipped with a natural $\Med$-bialgebra structure generated by the Alexander--Whitney map, the augmentation, and an algebraic version of the join.
This structure generalizes that of McClure--Smith and Berger--Fresse, in particular, we observe that Steenrod's \mbox{cup-$i$} construction can be explicitly obtained by composing the Alexander--Whitney and join maps.


Analogues of Steenrod's \mbox{cup-$i$} products defining Steenrod's mod $p$ cohomology operations effectively for simplicial and cubical sets were introduced in \cite{medina2021may_st}.
We expect to provide an axiomatic characterization of these in future work.

\subsection*{Outline}

\cref{s:preliminaries} presents the foundations of the theory of simplicial sets and chain complexes.
Our axiomatic characterization is presented in \cref{s:statement}.
The reader is encouraged to start the paper there and refer back if needed.
We discuss Steenrod squares in \cref{s:squares}.
This section is logically independent of the rest of the paper but serves to provide context for the subject.
Before presenting any proofs, we recast in \cref{s:reformulation} the main definitions and theorems of \cref{s:statement} in terms of a functor naturally isomorphic to that of normalized chains.
This functor seems interesting on its own right.
The proof of our main result occupies \cref{s:proof}.
We finish this work by showing in \cref{s:others} that all other \mbox{cup-$i$} constructions in the literature are isomorphic to Steenrod's original.

