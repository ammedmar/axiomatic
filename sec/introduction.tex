% !TEX root = ../axiomatic.tex

\section{Introduction} \label{s:introduction}

In \cite{steenrod1947products}, Steenrod introduced by means of formulas the \textit{\mbox{cup-$i$} products} on the cochains of spaces.
These bilinear maps were used to define cohomology operations
\[
\Sq^k \colon \rH^\vee(X; \F) \to \rH^\vee(X; \F),
\]
the \textit{Steenrod squares}, laying at the heart of stable homotopy theory.

Steenrod's formulas for the \mbox{cup-$i$} products extend the Alexander-Whitney product on cochains.
This non-commutative product $\cup_0$ induces the commutative algebra structure on cohomology, and we can interpret the higher \mbox{cup-$i$} products
\[
\cup_i\ \colon \cochains(X; \F)^{\ot 2} \to \cochains(X; \F)
\]
as coherent homotopies witnessing the derived commutativity of $\cup_0$ at the cochain level.

In later work by Steenrod and others, an indirect argument based on the acyclic carrier theorem is used to establish the existence of \mbox{cup-$i$} products and consequently of Steenrod squares.
This approach became the standard since any set of choices for the \mbox{cup-$i$} products homotopic to Steenrod's original gives rise to the same cohomology operations which, by then, had been axiomatically characterized.
As a consequence, the need to interact with a specific set of choices for the \mbox{cup-$i$} products largely declined.

Interest in actual cochain representatives and operations, like the \mbox{cup-$i$} products, has recently been rekindled by their use in the study of topological phases of matter.
A small sample of this physics literature is \cite{gaiotto2016spin, kapustin2017fermionic, meng2018classification, wang2020construction, barkeshli2021classification, tata2021anomalies, tata2021cubical}.
We also mention some mathematical advances motivated by this research \cite{brumfiel2016pontrjagin, brumfiel2018pontrjagin, medina2020cartan, medina2021adem}.
The importance of effective constructions is also highlighted by topological data analysis where cup-$i$ products have been used to extract finer features of real-world data using the open-source project \href{https://github.com/Steenroder/steenroder}{\texttt{steenroder}}.

In the present work we introduce an axiomatic characterization of Steenrod's original cup-$i$ construction up to isomorphism -- not just homotopy.
Loosely expressed, our axioms demand that a cup-$i$ construction be natural with respect to simplicial maps, parameterized by the smallest possible complex, not identically $0$ or reducible to subsimplices, and as free as possible with respect to transpositions.

We now say a few words on the relevance of this result.
First, it continues the axiomatic work on cohomology operations \cite{serre1053eilenberg_maclane, cartan1955iteration, steenrod1962cohomology} to the cochain level.
Second, it clarifies that all available cup-$i$ constructions represent the same isomorphism class.
For two of these, no previous proof of the equivalence with Steenrod's existed in print.
Additionally, we show that the operadic constructions of McClure--Smith \cite{mcclure2003multivariable} and Berger--Fresse \cite{berger2004combinatorial} define a bijection between the set of cup-$i$ constructions satisfying our axioms and that of bases of the arity two part of their operads.
Third, it highlights the fundamental nature of Steenrod's \mbox{cup-$i$} construction giving more context for its connections with physics, as documented above, higher category theory, where the nerve construction can be deduced from Steenrod's construction \cite{street1987orientals, medina2020globular}, and convex geometry, where it is obtained from a natural iterated fiber polytope construction \cite{billera1992fiber_polytope, medina2022fib_poly}.

The first alternative construction of \mbox{cup-$i$} products is due to Real \cite{real1996computability} and it is based on the Eilenberg--Zilber contraction.
It was further developed by Gonz\'alez-D\'iaz--Real \cite{gonzalez-diaz1999steenrod} and used as the basis for an algorithm, implemented by Palmieri in the open-source project \href{https://www.sagemath.org/}{\texttt{SAGE}}, for the computation of Steenrod squares.
We will review this construction in \cref{ss:real} and use our axioms to show it is isomorphic -- in fact equal -- to Steenrod's original.

In \cite{medina2021fast_sq}, new formulas defining a \mbox{cup-$i$} construction were introduced and used to present a faster algorithm for the computation of Steenrod squares for finite simplicial complexes.
The question of comparing the resulting \mbox{cup-$i$} construction with either Steenrod's or Real's was not addressed.
We use our axiomatic characterization to show in \cref{ss:original} that our construction agrees with Steenrod's.

A generalization of the notion of \mbox{cup-$i$} product structure is that of $E_\infty$-algebra.
Using work by \cite[\subsectionSymbol4.5]{benson1998representations} generalizing Steenrod's formulas, McClure--Smith \cite{mcclure2003multivariable} and Berger--Fresse \cite{berger2004combinatorial} constructed natural $E_\infty$-algebra structures on the normalized cochains of simplicial sets.
As we describe in \cref{ss:operads}, these structures induce a bijection between the set of bases of the arity 2 part of their operads and the set of \mbox{cup-$i$} constructions satisfying our axioms.

These operadic constructions are obtained by dualizing $E_\infty$-coalgebra structures that extend the Alexander--Whitney diagonal coproduct.
In \cite{medina2020prop1}, the normalized chains of standard simplicial sets where equipped with a natural $\Med$-bialgebra structure generated by the Alexander--Whitney coproduct, the augmentation map, and an algebraic version of the join of simplices.
This structure generalizes that of McClure--Smith and Berger--Fresse.
In particular, the \mbox{cup-$i$} coproducts axiomatized here can be expressed as explicit compositions of two natural maps: the diagonal and the join.

A (non-degenerate) $E_\infty$-coalgebra structure on the normalized chains of simplicial sets defines operations on their mod $p$ cohomology \cite{steenrod1953cyclic, may1970general}.
When $p = 2$ these agree with Steenrod squares.
Analogues of Steenrod's \mbox{cup-$i$} coproducts effectively defining these operations were introduced in \cite{medina2021may_st} and implemented in the open-source project \href{https://github.com/ammedmar/comch}{\texttt{ComCH}}.
These \textit{cup-$(p,i)$ coproducts} are expressible in terms of the diagonal and the join, and we expect to study their moduli in future work, as done here for $p=2$.

\subsection*{Outline}

\cref{s:preliminaries} reviews the required preliminaries from the theory of simplicial sets.
Our axiomatic characterization is presented in \cref{s:statement}.
The reader is encouraged to start the paper in this section referring back if needed.
We discuss Steenrod squares in \cref{s:squares}.
This section, serving to provide context for the subject, is logically independent of the rest of the paper.
Before presenting any proofs, we recast in \cref{s:reformulation} the main definitions and theorems of \cref{s:statement} in terms of a functor naturally isomorphic to that of normalized chains.
The proof of our main result occupies \cref{s:proof}.
We finish this work by showing in \cref{s:others} that all other \mbox{cup-$i$} constructions in the literature are isomorphic to Steenrod's original.

