% !TEX root = ../axiomatic.tex

\section{Introduction} \label{s:introduction}

In \cite{steenrod1947products}, Steenrod introduced by means of formulae the cup-$i$ products on the cochains of spaces. These bilinear maps give rise to the natural cohomology operations
\begin{equation*}
\Sq^k : H^*(X; \F) \to H^{*}(X; \F)
\end{equation*}
laying at the heart of stable homotopy theory.

Steenrod's formulae for the cup-$i$ products extend the Alexander-Whitney product on cochains. This non-commutative product induces the commutative algebra structure in cohomology, and we can interpret the higher cup-$i$ products
\begin{equation*}
\smallsmile_i\ : \cochains(X; \F) \tensor \cochains(X; \F) \to \cochains(X; \F)
\end{equation*}
as coherent homotopies enforcing the derived commutativity at the cochain level.

In later work by Steenrod \cite{steenrod1962cohomology}, May \cite{may1970general}, and others, an indirect argument based on the acyclic carrier theorem is used to establish the existence of cup-$i$ products and consequently of Steenrod squares. This approach became the standard since any set of choices for the cup-$i$ products homotopic to Steenrod's original one gives rise to the same cohomology operations which, by then, had been axiomatically characterized. As a consequence, the need to interact with a specific set of choices for the cup-$i$ products largely declined.

Attention to actual cochain representatives and cochain operations, like the \mbox{cup-$i$} products, has recently resurfaced in connection with condensed matter physics \cite{} and topological data analysis \cite{carlsson2009topology,tauzin2020giottotda}, since in these fields there is a need to effectively compute cohomological invariants.

Steenrod's original set of choices for the cup-$i$ products reappeared in the context of operads and props in the work of McClure-Smith \cite{mcclure2003multivariable}, Berger-Fresse \cite{berger2004combinatorial} and the author \cite{medina2018algebraic}, and in \cite{medina2018cellular} Steenrod's cup-$i$ products were shown to be determined, along with a full $E_\infty$-structure, by a cellular bialgebra structure on the standard interval. Additionally, there is a deep connection between Steenrod's cup-$i$ products and higher categories. In \cite{medina2019globular} and \cite{bibid} it is shown that Steenrod's cup-$i$ products determine and are determined by Street's nerve of higher categories. In this correspondence, transposition of $\smallsmile_i$ and arrow reversal of $i$-morphisms are intertwined. We are faced then with the following question: why are Steenrod's original set of choices so ubiquitous?

In the present work we give an answer in the form of an axiomatic characterization. Loosely expressed, Steenrod's original cup-$i$ products are characterized up to isomorphism by being: 1) natural with respect to simplicial maps, 2) parameterized by the smallest possible complex, 3) not all $0$, and 4) as free as possible with respect to transpositions.