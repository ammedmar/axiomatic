% !TEX root = ../axiomatic.tex

\section{Introduction}\label{s:introduction}

In his seminal paper \cite{steenrod1947products}, Steenrod introduced the \textit{cup-$i$ products} on the cochains of spaces using explicit formulas.
These bilinear maps give rise to the celebrated \textit{Steenrod squares}
\[
\Sq^k \colon \rH^\vee(X; \F) \to \rH^\vee(X; \F)
\]
laying at the heart of stable homotopy theory.

Steenrod's formulas for the cup-$i$ products extend the Alexander--Whitney product on cochains.
The cup-$0$ product $\cup_0$ induces the commutative algebra structure on cohomology, and the higher cup-$i$ products
\[
\cup_i\ \colon \cochains(X; \F)^{\ot 2} \to \cochains(X; \F)
\]
can be interpreted as coherent homotopies that witness the derived commutativity of $\cup_0$ at the cochain level.

Later work by Steenrod and others established the existence of the cup-$i$ products and, consequently, of the Steenrod squares using an indirect argument based on the acyclic carrier theorem.
This approach became standard since any set of choices for the cup-$i$ products homotopic to Steenrod's original gives rise to the same cohomology operations, which had by then been axiomatically characterized.
Consequently, the interest in specific formulas for the cup-$i$ products diminished in the second half of the century.

In recent years, there has been renewed interest in the use of actual cochain representatives, driven in part by their application to the study of topological phases of matter.
A small sample of the physics literature considering this approach includes \cite{gaiotto2016spin, kapustin2017fermionic, meng2018classification, wang2020construction, barkeshli2021classification, tata2021anomalies, tata2021cubical}, and some of the mathematical articles motivated by this viewpoint are \cite{brumfiel2016pontrjagin, brumfiel2018pontrjagin, medina2020cartan, medina2021adem}.
The application of effective constructions is further underscored by their use in topological data analysis, as demonstrated by the use of cup-$i$ products in \cite{medina2022per_st} for extracting features from real-world data through the open-source project \href{https://github.com/Steenroder/steenroder}{\texttt{steenroder}}.


In the present work we introduce an axiomatic characterization of Steenrod's original cup-$i$ construction up to isomorphism -- not just homotopy.
Loosely expressed, our axioms demand that a cup-$i$ construction be natural with respect to simplicial maps, parameterized by the smallest possible complex, not identically $0$ or reducible to subsimplices, and as free as possible with respect to transpositions.

We now turn to the significance of this result.
In addition to extending the axiomatic framework for cohomology operations \cite{serre1053eilenberg_maclane, cartan1955iteration, steenrod1962cohomology} to the cochain level, we demonstrate that all cup-$i$ constructions defined by formulas in the literature \cite{steenrod1947products, real1996computability, mcclure2003multivariable, berger2004combinatorial, medina2020prop1, medina2023fast_sq} belong to the same isomorphism class.

This universality highlights the fundamental nature of Steenrod's cup-$i$ construction and provides additional context for its connections to physics, as discussed earlier, as well as to higher category theory, where the nerve construction can be derived from Steenrod's construction \cite{street1987orientals, medina2020globular}.
We note that this universality is grounded in convex geometry, since the cup-$i$ products can be canonically constructed from the orthonormal basis of $\R^\infty$ using a specific model of the standard simplices.

The cup-$i$ products are part of a larger algebraic structure on cochains that encodes the entire homotopy type of spaces with certain finiteness assumptions \cite{mandell2006homotopy_type}.
Several explicit constructions of these so-called $E_\infty$-algebra structures on simplicial cochains have been developed \cite{mcclure2003multivariable, berger2004combinatorial, medina2020prop1}.
In this paper, we review these extensions and show that the set of bases of the arity $2$ part of the $E_\infty$-operads of McClure--Smith and Berger--Fresse are both in bijective correspondence with the elements in the isomorphism class defined by our axioms.

\subsection*{Outline}

We present the notion of cup-$i$ construction and our axiomatic characterization in \cref{s:statement}.
We postpone the proof of our main theorem until after \cref{s:reformulation}, where the statement is recasted in more category theoretic terms.
\cref{s:proof} is devoted to the proof of our axiomatic characterization.
In \cref{s:others} and \cref{s:operads} we show that all formulas in the literature give rise to isomorphic \mbox{cup-$i$} constructions.
We conclude this work in \cref{s:epilogue} discussing related and future work.