\section{Independence of the axioms}

In this section we construct examples showing the necessity of each of our axioms.

\subsection{Non-degeneracy}

The cup-$i$ construction defined by $\triangle_i = 0$ for each positive integer $i$ is both irreducible and free.

\subsection{Irreducibility}

\subsubsection{A general construction}

\begin{lemma}
	Let $\triangle$ be a cup-$i$ construction and fix $i_0 > 0$ and $n_0 \geq 0$.
	For all non-negative integers $i$ and $n$ define
	\[
	\widetilde\triangle_i[n] =
	\begin{cases}
		T\triangle_{i_0}[n_0] & \text{if }i=i_0 \text{ and } n=n_0, \\
		\big(\triangle_{i_0-1} + \triangle_{i_0}\bd\big)[n_0+1] & \text{if }i=i_0-1 \text{ and } n=n_0+1, \\
		\big(\triangle_{i_0-1} + \bd\triangle_{i_0}\big)[n_0] & \text{if }i=i_0-1 \text{ and } n=n_0, \\
		\triangle_{i}[n] & \text{otherwise}.
	\end{cases}
	\]
	Then $\widetilde\triangle = \set[\big]{\widetilde\triangle_i[n]}_{i,n}$ is a cup-$i$ construction.
\end{lemma}

\begin{proof}
	We need to verify that for all $i$ and $n$ the identity
	\[
	\big(\bd \widetilde\triangle_{i} + \widetilde\triangle_{i}\bd\big)[n] =
	(1+T) \widetilde\triangle_{i-1}[n]
	\]
	holds, knowing it does so for $\triangle$.
	A consequence of this identity that we will use without further mention is the following
	\[
	(1+T)\triangle_i\bd = (1+T)\bd\triangle_i.
	\]
	We will split the proof into five cases
	\[
	i < i_0-1 \quad;\quad i = i_0-1 \quad;\quad i = i_0 \quad;\quad i = i_0+1 \quad;\quad i > i_0+1.
	\]
	For $i > i_0+1$ the identity on $\widetilde\triangle$ reduces to that for $\triangle$ for all values of $n$.
	For $i = i_0+1$ there is one value of $n$ to consider.
	If $n = n_0$ we have
	\begin{align*}
		\big(\bd \widetilde\triangle_{i_0+1} + \widetilde\triangle_{i_0+1} \bd\big) &=
		\big(\bd \triangle_{i_0+1} + \triangle_{i_0+1} \bd\big) \\ &=
		(1+T)\triangle_{i_0} \\ &=
		(1+T)T\triangle_{i_0} \\ &=
		(1+T)\widetilde\triangle_{i_0}.
	\end{align*}
	For $i = i_0$ there are two values of $n$ to consider.
	For $n = n_0+1$ we have
	\begin{align*}
		\big(\bd \widetilde\triangle_{i_0} + \widetilde\triangle_{i_0} \bd\big) &=
		\big(\bd \triangle_{i_0} + T\triangle_{i_0} \bd\big) \\ &=
		\big((1+T)\bd\triangle_{i_0} + T(\bd\triangle_{i_0} + \triangle_{i_0}\bd)\big) \\ &=
		(1+T)\big(\triangle_{i_0}\bd \,+\, \triangle_{i_0-1}\big) \\ &=
		(1+T)\widetilde\triangle_{i_0-1}
	\end{align*}
	and for $n = n_0$ we have
	\begin{align*}
		\big(\bd \widetilde\triangle_{i_0} + \widetilde\triangle_{i_0} \bd\big) &=
		\big(\bd T \triangle_{i_0} + \triangle_{i_0} \bd\big) \\ &=
		\big((1+T)\bd\triangle_{i_0} + \bd\triangle_{i_0} + \triangle_{i_0}\bd\big) \\ &=
		(1+T)\big(\bd\triangle_{i_0} + \triangle_{i_0-1}\big) \\ &=
		(1+T)\widetilde\triangle_{i_0-1}.
	\end{align*}
	For $i = i_0-1$ there are three cases to consider.
	For $n = n_0+2$ we have
	\begin{align*}
		\big(\bd\widetilde\triangle_{i_0-1} + \widetilde\triangle_{i_0-1} \bd\big) &=
		\big(\bd\triangle_{i_0-1} + (\triangle_{i_0-1} + \triangle_{i_0} \bd)\bd\big) \\ &=
		(1+T)\triangle_{i_0-2} \\ &=
		(1+T)\widetilde\triangle_{i_0-2}.
	\end{align*}
	For $n = n_0+1$ we have
	\begin{align*}
		\big(\bd\widetilde\triangle_{i_0-1} + \widetilde\triangle_{i_0-1} \bd\big) &=
		\big(\bd(\triangle_{i_0-1} + \triangle_{i_0} \bd) + (\triangle_{i_0-1} + \bd\triangle_{i_0})\bd \big) \\ &=
		(1+T)\triangle_{i_0-2} \\ &=
		(1+T)\widetilde\triangle_{i_0-2}.
	\end{align*}
	For $n = n_0$ we have
	\begin{align*}
		\big(\bd\widetilde\triangle_{i_0-1} + \widetilde\triangle_{i_0-1} \bd\big) &=
		\big(\bd(\triangle_{i_0-1} + \bd\triangle_{i_0}) + \triangle_{i_0-1}\bd \big) \\ &=
		(1+T)\triangle_{i_0-2} \\ &=
		(1+T)\widetilde\triangle_{i_0-2}.
	\end{align*}
	If $i<i_0-1$ the identity on $\widetilde\triangle$ reduces to that for $\triangle$ for all values of $n$.
\end{proof}

\subsubsection{A specific example}

Let us consider the canonical cup-$i$ construction $\Delta$ and the integers $i_0=1$ and $n_0=3$.
The cup-$i$ construction $\widetilde\triangle$, defined above, is both non-degenerate and free but not irreducible as we will inspect.
If $i \neq i_0-1$ or $n \not\in \set{n_0,n_0+1}$ then $\widetilde\triangle_i[n]$ is equal to $\triangle_i[n]$ or $T\triangle_i[n]$.
With the assistance of \texttt{ComCH} we compute the two cases where the axioms could be broken:
\[
\widetilde\triangle_0[4] = (\Delta_0 + \Delta_1\partial)[4] =
\]
\noindent
{\ttfamily
	(0)(23) + (2)(23) + (3)(23) + ()(123) + (01)(3) + (13)(3) +\\
	(1)(13) + (012)() + (12)(1) + (12)(2) + (0)(03) + (3)(03) +\\
	(02)(0) + (2)(02) + (0)(01) + (1)(01).}
\[
\widetilde\triangle_0[3] = (\Delta_0 + \partial\Delta_1)[3] =
\]
\noindent
{\ttfamily
	(1234)() + (234)(0) + (34)(01) + (4)(012) + ()(0123) + (0)(034) +\\
	(02)(04) + (023)(0) + (0)(014) + (03)(01) + (0)(012) + (1)(134) +\\
	(12)(14) + (123)(1) + (1)(014) + (13)(01) + (1)(012) + (2)(234) +\\
	(12)(24) + (123)(2) + (2)(024) + (23)(02) + (2)(012) + (3)(234) +\\
	(13)(34) + (123)(3) + (3)(034) + (23)(03) + (3)(013) + (4)(234) +\\
	(14)(34) + (124)(4) + (4)(034) + (24)(04) + (4)(014).}

\medskip\noindent and conclude that $\widetilde\triangle$ is free and non-degenerate, but not irreducible.

\subsection{Freeness}

\subsubsection{Irreducible chains}

A basis element $V \ot W \in \cP(\simplex^n)^{\ot 2}$ is said to be \textit{irreducible} if $V \cap W = \emptyset$ and \textit{reducible} otherwise.
Let $\pired \colon \cP(\simplex^n)^{\ot 2} \to \cP(\simplex^n)^{\ot 2}$ be the projection to the subspace generated by reducible basis elements.
Explicitly,
\[
\pired(V \ot W) =
\begin{cases}
	V \ot W, & \text{if $V \ot W$ is reducible},\\
	\hfil 0, & \text{if not}.
\end{cases}
\]
A chain $\zeta \in \cP(\simplex^n)^{\ot 2}$ is said to be \textit{irreducible} if $\pired(\zeta) = 0$.
Notice that, by \cref{l:properties}, a cup-$i$ construction $\triangle$ is irreducible if and only if $\triangle_i[n]$ is irreducible for all $i, n \in \N$.

\subsubsection{Partition chains}

For any $U \subseteq \set{0,\dots,n}$, its \textit{partition chain} $\zeta_U$ is defined as the sum of all basis elements $V \ot W$ in $\cP(\simplex^n)^{\ot 2}$ with $V \union W = U$ and $V \cap W = \emptyset$.
Partition chains are irreducible by definition.
Additionally, the boundary $\bd \zeta_U$ of any partition chain is also irreducible as is demonstrated by the following:

\begin{lemma}\label{l:partition chains}
	If $\zeta_U$ be a partition chain in $\cP(\simplex^n)^{\ot 2}$ then
	\[
	\bd \zeta_U = \sum_{\bar u \notin U} \zeta_{\bar u.U}
	\]
	where the sum is over all elements $\bar u \in \{0,\dots,n\} \setminus U$ and $\zeta_{\bar u.U}$ is the partition chain of $\set{\bar u} \union U$.
\end{lemma}

\begin{proof}
	Let $\Lambda$ be the set of basis elements $V \ot W$ with $\zeta_U = \sum_\Lambda V \ot W$, then
	\[
	\bd \zeta_U =
	\sum_{\Lambda} \!\Big(\sum_{w \in W} w.V \ot W \ + \
	\sum_{v \in V} V \ot v.W \ + \
	\sum_{\bar u \notin U} \bar u.V \ot  W + V \ot \bar u.W\Big).
	\]
	We then have
	\begin{align*}
		\pired(\bd \zeta_U) &=
		\sum_{\Lambda} \!\Big(\sum_{w \in W} w.V \ot W \ + \
		\sum_{v \in V} V \ot v.W\Big), \\
		\pi^\perp_{\text{red}}(\bd \zeta_U) &=
		\sum_{\Lambda} \sum_{\bar u \notin U} \bar u.V \ot  W + V \ot \bar u.W.
	\end{align*}
	It suffices to show that $\pired(\bd \zeta_U) = 0$, since clearly $\pi^\perp_{\text{red}}(\bd \zeta_U) = \sum_{\bar u \notin U} \zeta_{\bar u.U}$.
	Let us consider two basis elements $V \ot W$ and $V' \ot W'$ in $\Lambda$, and, if possible, elements $v \in V$, $w \in W$, $v' \in V'$, and $w' \in W'$.
	The following simple implications hold by direct inspection
	\begin{align*}
		&(w.V \ot W = w'.V' \ot W') &\implies& &&(w = w') \wedge (V = V') \wedge (W = W'), \\
		&(V \ot v.W = V' \ot v'.W') &\implies& &&(v = v') \wedge (V = V') \wedge (W = W'), \\
		&(w.V \ot W = V' \ot v'.W') &\implies& &&(w = v') \wedge (V' = w.V) \wedge (W' = W \setminus v'), \\
		&(V \ot v.W = w'.V' \ot W') &\implies& &&(v = w') \wedge (V' = V \setminus w) \wedge (W' = v.W).
	\end{align*}
\end{proof}

%\begin{lemma}\label{l:partition chains}
%	An irreducible chain $\zeta \in \cP(\simplex^n)^{\ot 2}$ satisfies $(\pired \circ \bd)(\zeta) = 0$ iff $\zeta$ is a sum of partition chains.
%\end{lemma}

%\begin{proof}
%	Consider an irreducible chain $\zeta = \sum_{\Lambda} V \ot W$ with $\Lambda$ a subset of the basis of $\cP(\simplex^n)^{\ot 2}$.
%	Its boundary can be decomposed as follows
%	\[
%	\bd \zeta = \sum_{\Lambda} \Big(\sum_{w \in W} w.V \ot W \ + \ \sum_{v \in V} V \ot v.W \ +
%	\sum_{\bar u \notin V \union W} \bar u.V \ot  W + V \ot \bar u.W\Big).
%	\]
%	Therefore,
%	\[
%	(\pired \circ \bd) \zeta = \sum_{\Lambda} \Big(\sum_{w \in W} w.V \ot W \ + \ \sum_{v \in V} V \ot v.W\Big).
%	\]
%	Let us consider two irreducible basis elements $V \ot W$ and $V' \ot W'$, and elements $v \in V$, $w \in W$, $v' \in V'$, and $w' \in W'$.
%	The following simple implications hold by direct inspection
%	\begin{align*}
%		&(w.V \ot W = w'.V' \ot W') &\implies& &&(w = w') \wedge (V = V') \wedge (W = W'), \\
%		&(V \ot v.W = V' \ot v'.W') &\implies& &&(v = v') \wedge (V = V') \wedge (W = W'), \\
%		&(w.V \ot W = V' \ot v'.W') &\implies& &&(w = v') \wedge (V' = w.V) \wedge (W' = W \setminus v'), \\
%		&(V \ot v.W = w'.V' \ot W') &\implies& &&(v = w') \wedge (V' = V \setminus w) \wedge (W' = v.W).
%	\end{align*}
%	Now, if $(\pired \circ \bd) \zeta = 0$, these imply that for $V \ot W \in \Lambda$, $v \in V$, and $w \in W$ both $w.V \ot W \setminus w \in \Lambda$ and $V \setminus v \ot v.W \in \Lambda$.
%	From this it follows that $\Lambda$ contains all irreducible basis elements $V' \ot W'$ with $V' \union W' = V \ot W$, that is to say, all summands of the partition chain of $V \union W$.
%	Conversely, the above also implies that $(\pired \circ \bd)\zeta_U = 0$ for any partition chain $\zeta_U$.
%	\anibal{Maybe write more about this direction.}
%\end{proof}

\subsubsection{A general construction}

\begin{lemma}
	Let $\triangle$ be a cup-$i$ construction.
	Consider $i_0, n_0 \geq 0$ and a sum of partition chains $\zeta$ in $\cP(\simplex^{n_0})^{\ot 2}$ of degree $n_0+i_0+1$.
	For all non-negative integers $i$ and $n$ define
	\[
	\widetilde\triangle_i[n] =
	\begin{cases}
		\triangle_{i_0}[n_0] + \bd\zeta & \text{if } i=i_0, n=n_0, \\
		\triangle_{i}[n] & \text{otherwise}.
	\end{cases}
	\]
	Then $\widetilde\triangle = \set[\big]{\widetilde\triangle_i[n]}_{i,n}$ is a cup-$i$ construction.
\end{lemma}

\begin{proof}
	There are two non-trivial cases:
	(1) Since $(1+T) \zeta = 0$ we have
	\begin{align*}
		\bd \widetilde\triangle_{i_0+1}[n_0] + \widetilde\triangle_{i_0+1} \bd \,[n_0] &=
		\bd \triangle_{i_0+1}[n_0] + \triangle_{i_0+1} \bd \,[n_0] \\ &=
		(1+T)\triangle_{i_0}[n_0] \\ &=
		(1+T)\triangle_{i_0}[n_0] + (1+T) \bd \zeta \\ &=
		(1+T)\widetilde\triangle_{i_0}[n_0] .
	\end{align*}
	(2) Since $\bd^2 = 0$ we have
	\begin{align*}
		\bd \widetilde\triangle_{i_0}[n_0] + \widetilde\triangle_{i_0} \bd \,[n_0] &=
		\bd \triangle_{i_0}[n_0] + \bd^2 \zeta + \triangle_{i_0} \bd \,[n_0] \\ &=
		(1+T)\triangle_{i_0}[n_0] \\ &=
		(1+T)\widetilde\triangle_{i_0}[n_0]. \qedhere
	\end{align*}
\end{proof}

\subsubsection{Explicit example}

Let us consider the canonical cup-$i$ construction $\Delta$, the integers $i_0 = 0$ and $n_0 = 2$, and the partition chain $\zeta_{\set{1}}$ of the set $\set{1}$.
Using \texttt{comch} we have
\[
\bd \zeta_{\set{1}} = \zeta_{\set{01}} + \zeta_{\set{12}} =
\]
\noindent
{\ttfamily (01)() + (1)(0) + (0)(1) + ()(01) + (12)() + (2)(1) + (1)(2) + ()(12).}
\[
\widetilde\triangle_0[2] = \Delta_0[2] + \bd \zeta_{\set{1}} =
\]
\noindent
{\ttfamily (2)(0) + (01)() + (1)(0) + (0)(1) + (2)(1) + (1)(2) + ()(12)}.

\medskip \noindent This is an irreducible element but is not free.
Additionally, we can see that
\[
(s_0 \ot s_0)\widetilde\triangle_0[2] = (s_1 \ot s_1)\widetilde\triangle_0[2] = 0.
\]
