\section{Independence of the axioms}

In this section we construct examples showing the necessity of each of our axioms.

\subsection{The trivial construction}

The cup-$i$ construction defined by $\triangle_i = 0$ for each positive integer $i$ is both irreducible and free, but not non-degenerate.

\subsection{A non-free construction}

Consider the cycle
\[
z = (1+T)(012 \ot 1 + 012 \ot 2 + 01 \ot 12 + 02 \ot 12)
\]
in $\chains(\gsimplex^2) \ot \chains(\gsimplex^2)$, and let $\zeta$ be such that $\Psi_2^{\ot2}(\zeta) = z$.
Explicitly,
\[
\zeta = (1+T)\big(\emptyset \ot \set{02} + \emptyset \ot \set{01} + \set{2} \ot \set{0} + \set{1} \ot \set{0}\big).
\]
Define for each pair $i,n$
\[
\triangle_i[n] =
\begin{cases}
	\Delta_0[2] + \zeta & \text{if } i=0 \text{ and } n=2, \\
	\Delta_{i}[n] & \text{otherwise}.
\end{cases}
\]
This defines a cup-$i$ construction that is non-degenerate, irreducible but not free.

\subsection{A family of reducible constructions}

Before providing an example of a cup-$i$ construction that is both non-degenerate and free but not irreducible, we first introduce a general method to produce new cup-$i$ constructions from old ones.

\begin{lemma}
	Let $\triangle$ be a cup-$i$ construction and fix $i_0>0$ and $n_0 \geq 0$.
	For all non-negative integers $i$ and $n$ define
	\[
	\widetilde\triangle_i[n] =
	\begin{cases}
		T\triangle_{i_0}[n_0] & i=i_0, n=n_0, \\
		\big(\triangle_{i_0-1} + \triangle_{i_0}\bd\big)[n_0+1] & i=i_0-1, n=n_0+1, \\
		\big(\triangle_{i_0-1} + \bd\triangle_{i_0}\big)[n_0] & i=i_0-1, n=n_0, \\
		\triangle_{i}[n] & \text{otherwise}.
	\end{cases}
	\]
	Then $\widetilde\triangle = \set[\big]{\widetilde\triangle_i[n]}_{i,n}$ is a cup-$i$ construction.
\end{lemma}

\begin{proof}
	We need to verify that for all $i$ and $n$ the identity
	\[
	\big(\bd \widetilde\triangle_{i} + \widetilde\triangle_{i}\bd\big)[n] =
	(1+T) \widetilde\triangle_{i-1}[n]
	\]
	holds, knowing it does so for $\triangle$.
	A consequence of this identity that we will use without further mention is the following
	\[
	(1+T)\triangle_i\bd = (1+T)\bd\triangle_i.
	\]
	We will split the proof into five cases
	\[
	i < i_0-1 \quad;\quad i = i_0-1 \quad;\quad i = i_0 \quad;\quad i = i_0+1 \quad;\quad i > i_0+1.
	\]
	For $i > i_0+1$ the identity on $\widetilde\triangle$ reduces to that for $\triangle$ for all values of $n$.
	For $i = i_0+1$ there is one value of $n$ to consider.
	If $n = n_0$ we have
	\begin{align*}
		\big(\bd \widetilde\triangle_{i_0+1} + \widetilde\triangle_{i_0+1} \bd\big) &=
		\big(\bd \triangle_{i_0+1} + \triangle_{i_0+1} \bd\big) \\ &=
		(1+T)\triangle_{i_0} \\ &=
		(1+T)T\triangle_{i_0} \\ &=
		(1+T)\widetilde\triangle_{i_0}.
	\end{align*}
	For $i = i_0$ there are two values of $n$ to consider.
	For $n = n_0+1$ we have
	\begin{align*}
		\big(\bd \widetilde\triangle_{i_0} + \widetilde\triangle_{i_0} \bd\big) &=
		\big(\bd \triangle_{i_0} + T\triangle_{i_0} \bd\big) \\ &=
		\big((1+T)\bd\triangle_{i_0} + T(\bd\triangle_{i_0} + \triangle_{i_0}\bd)\big) \\ &=
		(1+T)\big(\triangle_{i_0}\bd \,+\, \triangle_{i_0-1}\big) \\ &=
		(1+T)\widetilde\triangle_{i_0-1}
	\end{align*}
	and for $n = n_0$ we have
	\begin{align*}
		\big(\bd \widetilde\triangle_{i_0} + \widetilde\triangle_{i_0} \bd\big) &=
		\big(\bd T \triangle_{i_0} + \triangle_{i_0} \bd\big) \\ &=
		\big((1+T)\bd\triangle_{i_0} + \bd\triangle_{i_0} + \triangle_{i_0}\bd\big) \\ &=
		(1+T)\big(\bd\triangle_{i_0} + \triangle_{i_0-1}\big) \\ &=
		(1+T)\widetilde\triangle_{i_0-1}.
	\end{align*}
	For $i = i_0-1$ there are three cases to consider.
	For $n = n_0+2$ we have
	\begin{align*}
		\big(\bd\widetilde\triangle_{i_0-1} + \widetilde\triangle_{i_0-1} \bd\big) &=
		\big(\bd\triangle_{i_0-1} + (\triangle_{i_0-1} + \triangle_{i_0} \bd)\bd\big) \\ &=
		(1+T)\triangle_{i_0-2} \\ &=
		(1+T)\widetilde\triangle_{i_0-2}.
	\end{align*}
	For $n = n_0+1$ we have
	\begin{align*}
		\big(\bd\widetilde\triangle_{i_0-1} + \widetilde\triangle_{i_0-1} \bd\big) &=
		\big(\bd(\triangle_{i_0-1} + \triangle_{i_0} \bd) + (\triangle_{i_0-1} + \bd\triangle_{i_0})\bd \big) \\ &=
		(1+T)\triangle_{i_0-2} \\ &=
		(1+T)\widetilde\triangle_{i_0-2}.
	\end{align*}
	For $n = n_0$ we have
	\begin{align*}
		\big(\bd\widetilde\triangle_{i_0-1} + \widetilde\triangle_{i_0-1} \bd\big) &=
		\big(\bd(\triangle_{i_0-1} + \bd\triangle_{i_0}) + \triangle_{i_0-1}\bd \big) \\ &=
		(1+T)\triangle_{i_0-2} \\ &=
		(1+T)\widetilde\triangle_{i_0-2}.
	\end{align*}
	If $i<i_0-1$ the identity on $\widetilde\triangle$ reduces to that for $\triangle$ for all values of $n$.
\end{proof}

Let us consider the canonical cup-$i$ construction $\Delta$ and the integers $i_0=1$ and $n_0=3$.
The cup-$i$ construction $\widetilde\triangle$, defined above, is both non-degenerate and free but not irreducible as we will inspect.
If $i \neq i_0-1$ or to $n \not\in \set{n_0,n_0+1}$ then $\widetilde\triangle_i[n]$ is equal to $\triangle_i[n]$ or $T\triangle_i[n]$.
With the assistance of \texttt{ComCH} we compute the two cases where the axioms could be broken:
\[
\widetilde\triangle_0[4] = (\Delta_0 + \Delta_1\partial)[4] =
\]
\noindent
{\ttfamily
	(1234)() + (234)(0) + (34)(01) + (4)(012) + ()(0123) + (0)(034) +\\
	(02)(04) + (023)(0) + (0)(014) + (03)(01) + (0)(012) + (1)(134) +\\
	(12)(14) + (123)(1) + (1)(014) + (13)(01) + (1)(012) + (2)(234) +\\
	(12)(24) + (123)(2) + (2)(024) + (23)(02) + (2)(012) + (3)(234) +\\
	(13)(34) + (123)(3) + (3)(034) + (23)(03) + (3)(013) + (4)(234) +\\
	(14)(34) + (124)(4) + (4)(034) + (24)(04) + (4)(014)}
\[
\widetilde\triangle_0[3] = (\Delta_0 + \partial\Delta_1)[3] =
\]
\noindent
{\ttfamily
	(0)(23) + (2)(23) + (3)(23) + ()(123) + (01)(3) + (13)(3) +\\
	(1)(13) + (012)() + (12)(1) + (12)(2) + (0)(03) + (3)(03) +\\
	(02)(0) + (2)(02) + (0)(01) + (1)(01)}

\vspace*{1em}
\noindent and conclude that $\widetilde\triangle$ is freeness and non-degenerate, but not irreducible.

\subsection{Non-degenerate, irreducible, \& not free}

Let us once again start with a method to produce new cup-$i$ constructions from old ones.

\begin{lemma}
	Let $\triangle$ be a cup-$i$ construction and fix $i_0,n_0 \geq 0$.
	Define
	\[
	\widetilde\triangle_i[n] =
	\begin{cases}
		(\triangle_{i_0} + (1+T)\bd\triangle_{i_0+1})[n_0] & \text{if } i=i_0 \text{ and } n=n_0, \\
		\hfil\triangle_{i}[n] & \text{otherwise}.
	\end{cases}
	\]
	for all non-negative $i$ and $n$.
	Then $\widetilde\triangle = \set[\big]{\widetilde\triangle_i[n]}_{i,n}$ is a cup-$i$ construction.
\end{lemma}

\begin{proof}
	We need to verify that for all $i$ and $n$ the identity
	\[
	\big(\bd \widetilde\triangle_{i} + \widetilde\triangle_{i}\bd\big)[n] =
	(1+T) \widetilde\triangle_{i-1}[n]
	\]
	holds, knowing it does so for $\triangle$.
	For $i = i_0+1$ there is one case to consider, $n = n_0$:
	\begin{align*}
		\bd\widetilde\triangle_{i_0+1} + \widetilde\triangle_{i_0+1}\bd &=
		\bd\triangle_{i_0+1} + \triangle_{i_0+1}\bd \\ &=
		(1+T)\triangle_{i_0} + (1+T)(1+T)\bd\triangle_{i_0+1}\\ &=
		(1+T)\widetilde\triangle_{i_0}.
	\end{align*}
	For $i = i_0$ there are two cases to consider.
	For $n = n_0+1$ we have
	\begin{align*}
		\bd\widetilde\triangle_{i_0} + \widetilde\triangle_{i_0}\bd &=
		\bd\triangle_{i_0} + \triangle_{i_0}\bd + (1+T)\bd\triangle_{i_0+1}\bd \\ &=
		\bd\triangle_{i_0} + \triangle_{i_0}\bd\\ &=
		(1+T)\widetilde\triangle_{i_0-1}.
	\end{align*}
	For $n = n_0$ we have
	\begin{align*}
		\bd\widetilde\triangle_{i_0} + \widetilde\triangle_{i_0}\bd &=
		\bd\triangle_{i_0} + \bd(1+T)\bd\triangle_{i_0+1} + \triangle_{i_0}\bd \\ &=
		\bd\triangle_{i_0} + \triangle_{i_0}\bd\\ &=
		(1+T)\widetilde\triangle_{i_0-1}.
	\end{align*}
	For all other values of $i$ and $n$ the identity on $\widetilde\triangle$ reduces to that on $\triangle$.
\end{proof}

%\begin{lemma}
%	Let $\triangle$ be a cup-$i$ construction and fix $i_0\geq0$.
%	If
%	\[
%	\widetilde\triangle_i =
%	\begin{cases}
%		\triangle_{i_0} + (1+T)\bd\triangle_{i_0+1} & \text{if }i=i_0, \\
%		\hfil\triangle_{i} & \text{otherwise}.
%	\end{cases}
%	\]
%	Then $\widetilde\triangle = \set[\big]{\widetilde\triangle_i[n]}_{i,n}$ is a cup-$i$ construction.
%\end{lemma}
%
%\begin{proof}
%	This is established by two direct computations:
%	\begin{align*}
%		\bd\widetilde\triangle_{i_0+1} + \widetilde\triangle_{i_0+1}\bd &=
%		\bd\triangle_{i_0+1} + \triangle_{i_0+1}\bd \\ &=
%		(1+T)\triangle_{i_0} + (1+T)(1+T)\bd\triangle_{i_0+1}\\ &=
%		(1+T)\widetilde\triangle_{i_0}.
%	\end{align*}
%	and
%	\begin{align*}
%		\bd\widetilde\triangle_{i_0} + \widetilde\triangle_{i_0}\bd &=
%		\bd\triangle_{i_0} + \bd(1+T)\bd\triangle_{i_0+1} + \triangle_{i_0}\bd + (1+T)\bd\triangle_{i_0+1}\bd \\ &=
%		\bd\triangle_{i_0} + \triangle_{i_0}\bd\\ &=
%		(1+T)\widetilde\triangle_{i_0-1}. \qedhere
%	\end{align*}
%\end{proof}

\begin{example*}
	Take $i_0 = 0$ and $n$ at least 3.
\end{example*}