% !TEX root = ../axiomatic.tex

\section{Reformulation} \label{s:reformulation}

Recall the functors $\chains$ and $\chains \ot \chains$ from $\sSet$ to $\Ch$, and the notion of natural linear transformation $\chains \to \chains \ot \chains$ between them, i.e., a linear map $\chains(X) \to \chains(X) \ot \chains(X)$ for every simplicial set $X$ that is natural with respect to simplicial maps.

%We have the following alternative description of this notion

\begin{lemma} \label{l:cup-i construction coalgebra}
	A cup-$i$ construction is canonically equivalent to a collection of natural linear transformations $\triangle_i \colon \chains \to \chains \ot \chains$ for $i \in \N$ with $\triangle_0$ a natural chain map and
	\[
	\bd \circ \, \triangle_i + \triangle_i \circ \bd =
	(1+T) \triangle_{i-1}
	\]
	for all $i > 0$.
\end{lemma}

%\begin{definition} \label{d:cup-i coproduct structure}
%	A \textbf{cup-$i$ coproduct structure} on a chain complex $C$ is a collection of linear maps
%	\[
%	\triangle_i \colon C \to C \ot C
%	\]
%	for $i \in \N$ with $\triangle_0$ a chain map and
%	\[
%	\bd \circ \triangle_i + \triangle_i \circ \bd =
%	(1+T) \triangle_{i-1}
%	\]
%	for $i > 0$.
%\end{definition}

%\begin{lemma} \label{l:cup-i coproduct structure equivalence}
%	If $C$ is finite dimensional, a cup-$i$ coproduct structure on $C$ is canonically equivalent to a cup-$i$ product structure on $C^\vee = \Hom(C, \Ftwo)$.
%\end{lemma}

\begin{proof}
	Let $C^\vee = \Hom(C, \F)$ with $C$ a finite dimensional chain complex.
	Using the hom-tensor adjunction and the finite dimensionality of $C$ we have
	\begin{align*}
	\Hom \big(W \ot_{\F[\Sym_2]} (C^\vee)^{\ot 2}, C^\vee \big) & \cong
	\Hom_{\F[\Sym_2]} \big( W, \Hom((C^\vee)^{\ot 2}, C^\vee) \big) \\ & \cong
	\Hom_{\F[\Sym_2]} \big( W, \Hom(C, C^{\ot 2}) \big)
	\end{align*}
	as chain complexes of $\F$-modules.
	In other words, the linear duality functor induces a bijection between cup-$i$ product structures on $C^\vee$ and $\F[\Sym_2]$-linear chain maps $\triangle \colon W \to \Hom(C, C^{\ot 2})$.
	The latter are canonically equivalent to linear maps $\triangle_i = \triangle(e_i)$ satisfying that $\triangle_0$ is a chain map and
	\[
	\bd \circ \, \triangle_i + \triangle_i \circ \bd =
	(1+T) \triangle_{i-1}
	\]
	for all $i > 0$ since
	\begin{align*}
	\bd \triangle (e_i) + \triangle \bd(e_i) &=
	\bd \triangle (e_i) + \triangle (1+T) (e_{i-1}) \\ &=
	\bd \circ \, \triangle_i + \triangle_i \circ \bd \ +\ (1+T) \triangle_{i-1}.
	\end{align*}

	By naturality, a cup-$i$ structure is determined by its restriction to representable simplicial sets $\simplex^n$.
	Since $\chains(\simplex^n)$ is finite dimensional, the previous argument involving only canonical equivalences proves the claim.
\end{proof}

Motivated by this formulation of cup-$i$ structure, we present a more explicit description of a general natural linear transformation $\chains \to \chains \ot \chains$.

\begin{lemma} \label{l:natural linear map}
	A natural linear transformation $f \colon \chains \to \chains \ot \chains$ is canonically equivalent to a collection of elements $f[n] \in \chains(\simplex^n) \ot \chains(\simplex^n)$
	for $n \in \N$ whose image under $\chains(\sigma_j) \ot \chains(\sigma_j)$ is $0$
	for each codegeneracy map $\sigma_j \colon [n] \to [n-1]$.
\end{lemma}

\begin{proof}
	By naturality $f$ is determined by its restriction to $\chains(\simplex^n)$ for $n \in \N$.
	Furthermore, for any non-degenerate simplex $\big( x \colon [m] \to [n] \big) \in \simplex^n_m$ we have
	\[
	f(x) = f \big( x \circ [m] \big) =
	\big( f \circ \chains(x) \big) [m] =
	\big( \chains(x) \ot \chains(x) \big) f[m],
	\]
	so the elements $f[m]$ with $m \in \N$ determine $f$.
	Here $[m]$ denotes the identity of the object $[m]$ and we are using $x$ to also denote the simplicial map $\simplex^m \to \simplex^n$ defined by $y \mapsto x \circ y$.

	The simplex associated to a codegeneracy map $\sigma_j \colon [n] \to [n-1]$ is degenerate in $\simplex^{n-1}$ so it is $0$ in $\chains(\simplex^{n-1})$.
	Therefore,
	\[
	0 = f(0) = f(\sigma_j) =
	f \big( \sigma_j \circ [n] \big) =
	\big( f \circ \chains(\sigma_j) \big) [n] =
	\big( \chains \sigma_j \ot \chains(\sigma_j) \big) f[n]
	\]
	as claimed.
\end{proof}

It will be convenient to use an alternative description of the functor of normalized chains $\chains$.
It is based on the fact that every non-degenerate simplex of $\simplex^n$ is a face of the identity $[n]$.

\begin{definition} \label{d:dual standard chains}
	Let $\P_{n-m}^n$ be the set of subsets of $\{0, \dots, n\}$ whose cardinality is $n-m$.
	Define the degree $m$ part of a chain complex $\cP(\simplex^n)$ by
	\[
	\cP(\simplex^n)_m = \begin{cases}
	\F\{\P_{n-m}^n\} & \text{if } 0 \leq m \leq n, \\
	\hfil 0 & \text{otherwise},
	\end{cases}
	\]
	and its differential by
	\[
	\bd U = \sum_{\bar u \in \{0, \dots, n\} \setminus U} \{\bar u\} \union U.
	\]
\end{definition}

\begin{lemma}
	The linear map
	\[
	\begin{tikzcd} [column sep=small, row sep=0]
	\Psi_n \colon &[-20pt] \cP(\simplex^n) \arrow[r] & \chains(\simplex^n) \\
	& U \arrow[r, mapsto] & d_U [n]
	\end{tikzcd}
	\]
	is a chain isomorphism for every $n \in \N$.
\end{lemma}

%\begin{lemma} \label{l:partial dU = dxU}
%	For any $x \in X_n$ and $U \in \P_{q}(n)$
%	\begin{equation} \label{lemma1: existence:eq1}
%	\partial_{n-q} \circ d_U(x) = \sum_{\bar{u} \in \overline{U}} d_{\bar{u}.U}(x).
%	\end{equation}
%\end{lemma}

\begin{proof}
	It can be easily seen that $\Psi_n$ induces a bijection of basis elements.
	We will verify that this assignment induces a chain map.
	Let $U = \{u_1 < \cdots < u_{n-m}\} \in \rP_{m-n}^n$.
	Using the relation $d_jd_u = d_ud_{j+1}$ if $u \leq j$ we have
	\begin{align*}
	\partial \Psi_n(U) &=
	\partial d_U[n] =
	\sum_{j=0}^{m} d_j\, d_{u_1} \cdots\, d_{u_{n-m}}[n] \\ &=
	\sum_{\bar u \in \{0, \dots, n\} \setminus U}
	\hspace*{-10pt} d_{u_1} \cdots\, d_{\bar{u}} \cdots\, d_{u_{n-m}}[n] \\ &=
	\sum_{\bar u \in \{0, \dots, n\} \setminus U} d_{\{\bar u\} \union U}[n] \\ &=
	\Psi_n(\partial U),
	\end{align*}
	as claimed.
\end{proof}

%\begin{lemma} \label{l:pigeon hole}
%	For any $x \in X_n$ and $q \in \{1, \dots, n\}$
%	\begin{equation} \label{e:pigeon hole 1}
%	\Delta_{n-q} \circ \partial_n (x)\ =\!
%	\sum_{U \in \P_{n-m}(n)} \left( \,
%	\sum_{u \in U^1} d_{u.U^0} \ot d_{U^1} +
%	\sum_{u \in U^0} d_{U^0} \ot d_{u.U^1} \right)(x \ot x).
%	\end{equation}
%\end{lemma}

\begin{definition}
	Let $\cP \colon \sSet \to \Ch$ be the functor defined on standard simplicial sets as in \cref{d:dual standard chains} and such that for $U \in \cP(\simplex^n)_m$ and $j \in \{0, \dots, n\}$:
	\[
	\cP(\delta_j)(U) = \{ u_1 < \dots < u_{k-1} < j < u_k+1 < \dots < u_{m-n}+1 \}
	\]
	where $k$ is determined by the inequalities, and
	\begin{equation} \label{e:action of P on a codegeneracy}
	\cP(\sigma_j)(U) = \begin{cases}
	U \setminus \{j+1\} & j+1 \in U, \\
	\hfil U \setminus \{j\} & j+1 \notin U \wedge j \in U, \\
	\hfil 0 & j+1 \notin U \wedge j \notin U.
	\end{cases}
	\end{equation}
\end{definition}

\begin{lemma}
	The chain isomorphisms $\{\Psi_n\}_{n \in \N}$ define a natural equivalence $\Psi$ between the functors $\cP$ and $\chains$.
\end{lemma}

\begin{proof}
	This follows from straightforward computations using the cosimplicial identities.
\end{proof}

%\anibal{Maybe we need this lemma:}
%\begin{lemma} \label{l:condition to be in the kernel of sxs}
%	Let $j \in \{0, \dots, n-1\}$.
%	An element $V \otimes W \in \cP[n] \otimes \cP[n]$ is in the kernel of $(\cP \sigma_j \ot \cP \sigma_j)$ if an only if both $j$ and $j+1$ are missing from either $V$ or $W$.
%\end{lemma}
%
%\begin{proof}
%	TBW
%\end{proof}

We can use the natural isomorphism $\Psi$ to interpret cup-$i$ constructions in terms of natural linear transformations $\triangle_i \colon \cP \to \cP \ot \cP$ as in \cref{l:cup-i construction coalgebra}.
Additionally, these are determined, as in \cref{l:natural linear map}, by elements $\triangle_i[n]$ in $\cP(\simplex^n) \ot \cP(\simplex^n)$ for $i, n \in \N$ laying in the kernel of $\cP(\sigma_j) \otimes \cP(\sigma_j)$ for every codegeneracy $\sigma_j \colon [n] \to [n-1]$.
For example, the canonical cup-$i$ construction (\cref{d:my cup-i construction}) is given by the collection indexed by $i, n \in \N$ of elements
\[
\Delta_i[n] = \sum_{U \in \rP^n_{n-i}} U^0 \otimes U^1
\]
if $i \leq n$ and $0$ otherwise, where for $\varepsilon \in \Ftwo \cong \{0,1\}$ we have $U^\varepsilon = \ind_U^{-1}(\varepsilon)$ with
\[
\begin{split}
\ind_U \colon U &\to \F \\
u_j &\mapsto u_j + j \mod 2.
\end{split}
\]
We refer to $\ind_U$ as the \textbf{index function of $U$} and to the ordered partition $U = U^0 \sqcup U^1$ as its \textbf{index splitting}.

%\begin{definition}
%	For any $U = \{u_1 < \dots < u_{m-n}\} \in \P_{m-n}^n$ the \textbf{index function} is defined by
%	\[
%	\begin{split}
%	\ind_U \colon U &\to \F \\
%	u_j &\mapsto u_j + j \mod 2,
%	\end{split}
%	\]
%	and the \textbf{index splitting} of $U$ is the ordered partition $U = U^0 \sqcup U^1$ with
%	\[
%	U^\varepsilon = \ind_U^{-1}(\varepsilon).
%	\]
%\end{definition}

%\begin{definition}
%	For $n, i \in \N$ we introduce the following elements of $\cP[n] \ot \cP[n]$:
%	\[
%	\Delta_i [n] \ =
%	\sum_{U \in \P_{n-i}(n)} {U^0} \ot {U^1}
%	\]
%	if $i \leq \{0, \dots, n\}$ and $\Delta_i [n] = 0$ if not.
%\end{definition}
%
%As proven in \cite{medina2021newformulas} these elements define a cup-$i$ construction which is non-degenerate by direct inspection.
%
%We now state a stronger formulation of Theorem \ref{t:main}
%
%\begin{theorem} \label{t:main reformulated}
%	If $\triangle$ is a free non-degenerate irreducible cup-$i$ construction, then, for every $i \in \N$, either $\triangle_i = \Delta_i$ or $\triangle_i = T \Delta_i$.
%\end{theorem}

%We will also prove the following lemma that helps verify the irreducibility of a cup-$i$ construction.
%
%\begin{lemma}
%	Let $\triangle$ be a free non-degenerate cup-$i$ construction, then, then $\triangle$ is irreducible if an only if for every $i \in \N$ and simplex $x$,
%	$\triangle_i = \Delta_i$ or $\triangle_i = T \Delta_i$.
%\end{lemma}