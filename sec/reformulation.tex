% !TEX root = ../axiomatic.tex

\section{Reformulation}\label{s:reformulation}

This section provides a category theoretic perspective on the notion of simplicial \mbox{cup-$i$} construction.

\subsection{Cup-\textit{i} coproducts}

A \textbf{cup-$i$ coproduct structure} on a chain complex $C$ is a $\Sym_2$-equivariant chain map
\[
\begin{tikzcd}[column sep=small,row sep=0]
	\triangle \colon &[-15pt] W \rar & \Hom(C, C^{\ot 2}) \\
	& e_i \rar[maps to] & \triangle_i.
\end{tikzcd}
\]
Such a map $\triangle$ is equivalent to a collection of linear maps $\set{\triangle_i}_{i\in\N}$ satisfying
\[
\bd \circ \, \triangle_i + \triangle_i \circ \bd =
(1+T) \triangle_{i-1}
\]
for all $i \in \N$ with the convention $\triangle_{-1} = 0$.

Let us denote $\Hom(C, \F)$ by $C^\vee$ and assume $C$ is finite dimensional.
Using the hom-tensor adjunction and the finite dimensionality of $C$ we have
\begin{equation}\label{eq:hom-tensor}
	\begin{split}
		\Hom \big(W \ot_{\F[\Sym_2]} (C^\vee)^{\ot 2}, C^\vee \big) & \cong
		\Hom_{\F[\Sym_2]} \big( W, \Hom((C^\vee)^{\ot 2}, C^\vee) \big) \\ & \cong
		\Hom_{\F[\Sym_2]} \big( W, \Hom(C, C^{\ot 2}) \big)
	\end{split}
\end{equation}
as chain complexes of $\F$-modules.
In other words, the linear duality functor induces a bijection between \mbox{cup-$i$} product structures on $C^\vee$ and cup-$i$ coproduct structures on $C$.

\subsection{Naturality}\label{ss:naturality}

A semi-simplicial cup-$i$ construction $\triangle$ is completely determined by its restriction to representable semi-simplicial sets
\[
W \ot_{\Ftwo[\Sym_2]} \chains(\simplex^n)^{\ot 2} \to \chains(\simplex^n).
\]
Since each of the chain complexes $\chains(\simplex^n)$ is finite dimensional, $\triangle$ is determined by the natural collection of $\Sym_2$-equivariant maps
\[
W \to \Hom(\chains(\simplex^n), \chains(\simplex^n)^{\ot 2})
\]
defined by the isomorphisms \eqref{eq:hom-tensor}.
These in turn correspond to a collection of natural linear maps $\set[\big]{\triangle_i \colon \chains(\simplex^n) \to \chains(\simplex^n)^{\ot 2}}_{i \in \N}$ satisfying
\[
\bd \circ \, \triangle_i + \triangle_i \circ \bd =
(1+T) \triangle_{i-1}
\]
with the convention $\simplex_{-1} = 0$.
In other words, a collection of linear natural transformations $\set{\triangle_i}_{i\in\N}$ from the functor $\chains \colon \simplex \to \Ch$ to the functor $\chains^{\ot 2}$.
Each natural transformation $\triangle_i$ is completely determined by the set $\set[\big]{\triangle_i[n] \in \chains(\simplex^n)^{\ot 2}_{n+i}}_{n\in\N}$ where $[n]$ denotes the identity $[n] \to [n]$.
%
%This defines a natural linear transformation from
%
%
%Since each of the chain complexes $\chains(\simplex^n)$ is finite dimensional, a (semi-)simplicial cup-$i$ construction is determined by a collection $\triangle$ of natural linear maps $\set[\big]{\triangle_i \colon \chains(\simplex^n) \to \chains(\simplex^n)^{\ot 2}}_{i \in \N}$ satisfying
%\[
%\bd \circ \, \triangle_i + \triangle_i \circ \bd =
%(1+T) \triangle_{i-1}
%\]
%for all $i \in \N$ with the convention $\triangle_{-1} = 0$.
%Let $\triangle_i[n]$ be the image of the top dimensional (non-degenerate) simplex in $\simplex^n$.
%The data $\set{\triangle_i[n]}_{i,n\in\N}$ completely determines $\triangle$.
To see this, consider a simplex $\big(x \colon [m] \to [n] \big) \in \simplex^n_m$ and notice that
\[
\triangle_i(x) = \triangle_i \big( x \circ [m] \big) =
\triangle_i \circ \chains(x) [m] =
\chains(x)^{\ot 2} \triangle_i[m],
\]
where $x$ also denotes the map $\simplex^m \to \simplex^n$ defined by $y \mapsto x \circ y$.
We will often present a semi-simplicial cup-$i$ construction $\triangle$ as the set $\set{\triangle_i[n]}_{i,n\in\N}$

In the simplicial context, the simplex associated to a codegeneracy map $\sigma_j \colon [n] \to [n-1]$ is degenerate in $\simplex^{n-1}$ so it is $0$ in $\chains(\simplex^{n-1})$.
Therefore,
\[
0 = \triangle_i(0) = \triangle_i(\sigma_j) =
\triangle_i \big( \sigma_j \circ [n] \big) =
\big( \triangle_i \circ \chains(\sigma_j) \big) [n] =
\chains(\sigma_j)^{\ot 2} \triangle_i[n],
\]
which proves the following.

\begin{lemma}
	A semi-simplicial cup-$i$ construction $\triangle$ defines a simplicial one if and only if for each $i,n\in\N$
	\[
	\chains(\sigma_j)^{\ot 2}\big(\triangle_i[n]\big) = 0
	\]
	for each codegeneracy map $\sigma_j \colon [n] \to [n-1]$.
\end{lemma}

\subsection{An alternative functor}\label{ss:alternative functor}

It will be convenient to use an alternative to the functor of normalized chains.

Consider $n,m \in \N$ and let $\P_{n-m}^n$ be the set of subsets of $\{0, \dots, n\}$ whose cardinality is $n-m$.
Let $\cP[n]$ be the chain complex defined by
\[
\cP[n]_m = \Ftwo\set{\P_{n-m}^n}, \qquad
\bd U = \sum_{\mathclap{\bar u \in \{0, \dots, n\} \setminus U}} \{\bar u\} \union U.
\]
For any coface $\delta_j \colon [n] \to [n+1]$ the chain map $\cP(\delta_j) \colon \cP[n] \to \cP[n+1]$ is defined by
\[
\cP(\delta_j)(U) = \{ u_1 < \dots < u_{k-1} < j < u_k+1 < \dots < u_{n-m}+1 \}
\]
where $k$ is determined by the inequalities.
For any codegeneracy $\sigma_j \colon [n] \to [n-1]$ the chain map $\cP(\sigma_j) \colon \cP[n] \to \cP[n-1]$ is defined by
\[
\cP(\sigma_j)(U) = \begin{cases}
	U \setminus \{j+1\}, & j+1 \in U, \\
	\hfil U \setminus \{j\}, & j+1 \notin U \text{ and } j \in U, \\
	\hfil 0, & j+1 \notin U \text{ and } j \notin U.
\end{cases}
\]

\begin{lemma*}
	The assignment
	\[
	\begin{tikzcd}[column sep=small,row sep=0]
		\Psi \colon &[-15] \P_{n-m}^n \rar & \simplex^n_m \\
		& U \rar[mapsto] & d_U [n]
	\end{tikzcd}
	\]
	defines a natural equivalence between the functors $\cP,\chains \colon \simplex \to \Ch$.
\end{lemma*}

\begin{proof}
	Since any non-degenerate simplex in $\simplex^n$ is a face of the top dimensional one, $\Psi$ defines a bijection between the basis of $\cP[n]_m$ and $\chains[n]_m$.
	Let $U = \{u_1 < \cdots < u_{n-m}\} \in \rP_{n-m}^n$.
	Using that for $u \leq j$ we have $d_jd_u = d_ud_{j+1}$ we can show that this bijection induces a chain isomorphism as follows:
	\begin{align*}
		\bd d_U[n] &=
		\sum_{\mathclap{j \in \{0, \dots, m\}}}
		d_j\, d_{u_1} \cdots\, d_{u_{n-m}}[n] \\ &=
		\sum_{\mathclap{\bar u \in \{0, \dots, n\} \setminus U}}
		d_{u_1} \cdots\, d_{\bar{u}} \cdots\, d_{u_{n-m}}[n] \\ &=
		\sum_{\mathclap{\bar u \in \{0, \dots, n\} \setminus U}}
		d_{\{\bar u\} \union U}[n].
	\end{align*}
	The compatibility with cofaces and codegeneracies is left to the interested reader.
\end{proof}

%The induced isomorphism of graded vector spaces defines a chain isomorphism
%\[
%\begin{tikzcd}[column sep=small,row sep=small]
%	\F\set{\P_{n}^n} \dar & \lar \F\set{\P_{n-1}^n} \dar & \lar \dotsb & \lar \F\set{\P_{0}^n} \dar \\
%	\chains(\simplex^n)_0 & \lar \chains(\simplex^n)_1 & \lar \dotsb & \lar \chains(\simplex^n)_n \\
%\end{tikzcd}
%\]

%A direct consequence of \cref{e:codegeneracy P} is the following.
%
%\begin{lemma}\label{l:kernel of sxs}
%	For any codegeneracy $\sigma_j \colon [n] \to [n-1]$ a basis element $V \ot W \in \cP(\simplex^n)^{\ot 2}$ is in $\ker \cP(\sigma_j)^{\ot 2}$ if an only if $j$ and $j+1$ are both missing from $V$ or $W$.
%\end{lemma}

\subsection{Cup-\textit{i} constructions}

By combining \cref{ss:naturality} and \cref{ss:alternative functor}, the data of a semi-simplicial cup-$i$ construction $\triangle$ is a collection, indexed by $i,n \in \N$, of elements
\[
\triangle_i [n] =
\sum_{\mathclap{\quad \lambda \in \Lambda(i,n)}} \, V_\lambda \ot W_\lambda
\]
in $\cP(\simplex^n)^{\ot 2}_{i+n}$
%in the kernel of $\cP(\sigma_j)^{\ot 2}$ for every codegeneracy $\sigma_j \colon [n] \to [n-1]$,
where $\Lambda(i,n)$ is a finite (possibly empty) indexing set, and $V_\lambda \ot W_\lambda$ is a basis element for each $\lambda \in \Lambda(i,n)$.

\anibal{Continue here}
\begin{lemma}\label{l:properties}
	The semi-simplicial \mbox{cup-$i$} construction $\triangle$ is:
	\begin{enumerate}[(i)]
%		\item Simplicial iff $\forall j \in \set{0,\dots,n-1}$ $\cP(\sigma_j)^{\ot 2} \triangle_i [n]$
%		\item Non-zero iff
%		$\exists i,j\in \N$, $\triangle_i [n] \neq 0$.
		\item Irreducible iff\,
		$\forall i,n \in \N$, $\forall \lambda \in \Lambda(i,n)$, $V_\lambda \cap W_\lambda = \emptyset$.
		\item Free iff\,
		$\forall i,n \in \N$, $\forall \lambda_1, \lambda_2 \in \Lambda(i,n)$, $V_{\lambda_1} \ot W_{\lambda_1} \neq W_{\lambda_2} \ot V_{\lambda_2}$ if $i \neq n$.
	\end{enumerate}
\end{lemma}

\begin{proof}
	By naturality, it suffices to verify these equivalences for the \mbox{cup-$i$} product structure defined on $\cochains(X)$ when $X = \simplex^n$ for every $n \in \N$.
	We will use the fact that the isomorphism $\Psi_n^{\ot 2} \colon \cP(\gsimplex^n)^{\ot 2} \to \chains(\gsimplex^n)^{\ot 2}$ sends basis elements to basis elements, as does the linear duality isomorphism between chains and cochains.

	\noindent (1) This follows from the fact that $\chains(\gsimplex^0)^{\ot 2}$ is 1-dimensional and generated by $[0] \ot [0] = \Psi_0^{\ot 2} (\emptyset \ot \emptyset)$.
	Explicitly, if $x$ is a $0$-simplex with dual basis element denoted $x^\vee$, the only two options for $x^\vee \cup_0 x^\vee$ are $x^\vee$ or $0$, with the first holding if and only if $\triangle_0[0] = \emptyset \ot \emptyset$.

	\noindent (2) To prove this equivalence notice that there is $k \in V_\lambda \cap W_\lambda$ for one of the summands of $\triangle_i [n]$ if an only the image of this summand under $\Psi_n^{\ot 2}$ is
	\[
	\underbrace{d_{V_\lambda \setminus k} \overbrace{d_k[n]}^{y}}_{y^{(1)}}
	\ot
	\underbrace{d_{W_\lambda \setminus k} \overbrace{d_k[n]}^{y}}_{y^{(2)}}
	\]
	with $y^{(1)} \smallsmile_i y^{(2)} \neq 0$.

	\noindent (3) This equivalence follows directly from considering the isomorphism $\Psi_n^{\ot 2}$.
\end{proof}

\subsection{Canonical cup-\textit{i} construction}

For any $U \in \rP^n_{n-i}$ the \textbf{index function} of $U$ is given by
\[
\begin{split}
	\ind_U \colon U &\to \F \\
	u_j &\mapsto u_j + j \mod 2.
\end{split}
\]
For any $U \in \rP^n_{n-i}$ and $\varepsilon \in \Ftwo \cong \{0,1\}$ we write $U^\varepsilon$ instead of $\ind_U^{-1}(\varepsilon)$.

By inspecting \cref{d:my cup-i} we have the following.

\begin{lemma}\label{l:canonical}
	The canonical simplicial \mbox{cup-$i$} construction is given by the collection of elements $\Delta_i[n] \in \cP(\simplex^n)^{\ot 2}_{i+n}$ for $i,n \in \N$ with
	\[
	\Delta_i[n] =
	\sum_{\mathclap{U \in \rP^n_{n-i}}} U^0 \ot U^1
	\]
	if $0 \leq i \leq n$ and $\Delta_i[n] = 0$ otherwise.
\end{lemma}

The following is deduced from \cref{l:properties} by inspecting \cref{l:canonical}.

\begin{theorem}\label{t:existence}
	The canonical \mbox{cup-$i$} construction is simplicial, non-zero, irreducible, and free.
\end{theorem}