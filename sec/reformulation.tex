% !TEX root = ../axiomatic.tex

\section{Reformulation}\label{s:reformulation}

This section provides a category theoretic perspective on \mbox{(semi-)}simplicial \mbox{cup-$i$} constructions.
We start by presenting a linear dual description.

\subsection{Cup-\textit{i} coproducts}

A \textbf{cup-$i$ coproduct structure} on a chain complex $C$ is an $\Sym_2$-equivariant chain map
\[
\begin{tikzcd}[column sep=small,row sep=0]
	\triangle \colon &[-15pt] W \rar & \Hom(C, C^{\ot 2}) \\
	& e_i \rar[maps to] & \triangle_i.
\end{tikzcd}
\]
Such a map $\triangle$ is equivalent to a collection of linear maps $\set{\triangle_i}_{i\in\N}$ satisfying
\[
\bd \circ \, \triangle_i + \triangle_i \circ \bd =
(1+T) \triangle_{i-1}
\]
for all $i \in \N$ with the convention $\triangle_{-1} = 0$.

Let us denote $\Hom(C, \F)$ by $C^\vee$ and assume $C$ is finite dimensional.
Using the hom-tensor adjunction and the finite dimensionality of $C$ we have
\begin{equation}\label{eq:hom-tensor}
	\begin{split}
		\Hom \big(W \ot_{\F[\Sym_2]} (C^\vee)^{\ot 2}, C^\vee \big) & \cong
		\Hom_{\F[\Sym_2]} \big( W, \Hom((C^\vee)^{\ot 2}, C^\vee) \big) \\ & \cong
		\Hom_{\F[\Sym_2]} \big( W, \Hom(C, C^{\ot 2}) \big)
	\end{split}
\end{equation}
as chain complexes of $\F$-modules.
In other words, the linear duality functor induces a bijection between \mbox{cup-$i$} product structures on $C^\vee$ and cup-$i$ coproduct structures on $C$.

%\subsection{(Semi-)simplicial chains}
%
%Recall the category of ordinals with strictly order-preserving maps
%\[
%\begin{tikzcd}
%	{[0]} \arrow[r,shift left=3pt, "\delta_0"] \arrow[r,shift right=3pt, "\delta_1"'] &
%	{[1]} \arrow[r,shift left=6pt, "\delta_0"] \arrow[r, "\delta_1"' description] \arrow[r,shift right=6pt, "\delta_2"'] &
%	\dotsb
%	\arrow[r,shift left=5pt, "\delta_0"] \arrow[r,shift right=5pt, "\delta_n"', "\scalebox{0.6}{$\vdots$}"] &
%	{[n]}
%	\arrow[r,shift left=5pt, "\delta_0"] \arrow[r,shift right=5pt, "\delta_{n+1}"', "\scalebox{0.6}{$\vdots$}"] &
%	\dotsb \
%\end{tikzcd}
%\]

\subsection{Naturality}\label{ss:naturality}

A (semi-)simplicial \mbox{cup-$i$} construction $\triangle$ is completely determined by its restriction to representable (semi-)simplicial sets
\[
W \ot_{\Ftwo[\Sym_2]} \chains(\simplex^n)^{\ot 2} \to \chains(\simplex^n).
\]
Given that each of the chain complexes $\chains(\simplex^n)$ is finite dimensional, $\triangle$ is determined by the natural collection of cup-$i$ coproduct structures
\[
W \to \Hom(\chains(\simplex^n), \chains(\simplex^n)^{\ot 2})
\]
defined by the hom-tensor adjunction \eqref{eq:hom-tensor}.
These in turn correspond to a collection, parameterized by $n \in \N$, of natural linear maps
\begin{equation}\label{eq:natural_linear_maps}
	\set[\big]{\triangle_i \colon \chains(\simplex^n) \to \chains(\simplex^n)^{\ot 2}}_{i \in \N}
\end{equation}
satisfying
\[
\bd \circ \, \triangle_i + \triangle_i \circ \bd =
(1+T) \triangle_{i-1}
\]
with $\triangle_{-1} = 0$.
Each natural transformation $\triangle_i$ in \eqref{eq:natural_linear_maps} is completely determined by the set
\[
\set[\big]{\triangle_i[n] \in \chains(\simplex^n)^{\ot 2}_{n+i}}_{n\in\N}
\]
where $[n]$ denotes the identity $[n] \to [n]$.
To see this, consider a simplex $\big(x \colon [m] \to [n] \big) \in \simplex^n_m$ and notice that
\[
\triangle_i(x) = \triangle_i \big( x \circ [m] \big) =
\triangle_i \circ \chains(x) [m] =
\chains(x)^{\ot 2} \triangle_i[m],
\]
where $x$ also denotes the map $\simplex^m \to \simplex^n$ defined by $y \mapsto x \circ y$.
We will often present a (semi-)simplicial \mbox{cup-$i$} construction $\triangle$ as the set $\set{\triangle_i[n]}_{i,n\in\N}$.

In the simplicial context, the simplex associated to a codegeneracy map $\sigma_j \colon [n] \to [n-1]$ is degenerate in $\simplex^{n-1}$ so it is $0$ in $\chains(\simplex^{n-1})$.
Therefore,
\[
0 = \triangle_i(0) = \triangle_i(\sigma_j) =
\triangle_i \big( \sigma_j \circ [n] \big) =
\big( \triangle_i \circ \chains(\sigma_j) \big) [n] =
\chains(\sigma_j)^{\ot 2} \triangle_i[n],
\]
which, together with the fact that codegeneracies and coface maps generate the simplex category, proves the following.

\begin{lemma}\label{l:simplicial_from_semisimplicial}
	A semi-simplicial \mbox{cup-$i$} construction $\triangle$ is simplicial if and only if $\chains(\sigma_j)^{\ot 2}\big(\triangle_i[n]\big) = 0$ for all $i,n\in\N$ and each codegeneracy map $\sigma_j \colon [n] \to [n-1]$.
\end{lemma}

\subsection{An alternative functor}\label{ss:equivalence_of_functors}

For $n,m \in \N$ let $\P_{n-m}^n$ be the set of subsets of $\set{0,\dots,n}$ whose cardinality is $n-m$.
Let $\cP(\simplex^n)$ be the chain complex defined by
\[
\cP(\simplex^n)_m = \Ftwo\set{\P_{n-m}^n}, \qquad
\bd U = \sum_{\mathclap{\bar{u} \notin U}} \bar{u}.U
\]
where $\bar{u} \notin U$ stands for $\bar{u} \in \set{0,\dots,n} \setminus U$ and $\bar{u}.U$ for $\set{\bar{u}} \union U$.
For any coface $\delta_j \colon [n] \to [n+1]$ the chain map $\cP(\delta_j) \colon \cP(\simplex^n) \to \cP(\simplex^{n+1})$ is defined by
\[
\cP(\delta_j)(U) = \set{u_1 < \dots < u_{k-1} < j < u_k+1 < \dots < u_{n-m}+1}
\]
where $k$ is determined by the inequalities.
For any codegeneracy $\sigma_j \colon [n] \to [n-1]$ the chain map $\cP(\sigma_j) \colon \cP(\simplex^n) \to \cP(\simplex^{n+1})$ is defined by
\begin{equation}\label{eq:codegeneracies}
	\cP(\sigma_j)(U) =
	\begin{cases}
		U \setminus \{j+1\}, & j+1 \in U, \\
		\hfil U \setminus \{j\}, & j+1 \notin U \text{ and } j \in U, \\
		\hfil 0, & j+1 \notin U \text{ and } j \notin U.
	\end{cases}
\end{equation}

\begin{lemma}
	The function
	\[
	\begin{tikzcd}[column sep=small,row sep=0]
		\Psi \colon &[-19] \P_{n-m}^n \rar & \simplex^n_m \\
		& U \rar[mapsto] & d_U [n]
	\end{tikzcd}
	\]
	induces a natural equivalence between the functors $\cP$ and $\chains$.
\end{lemma}

\begin{proof}
	Since any (non-degenerate) simplex in $\simplex^n$ is a face of the top dimensional one $[n]$, this assignment defines a bijection between the basis of $\cP(\simplex^n)_m$ and $\chains(\simplex^n)_m$.
	Let $U = \{u_1 < \cdots < u_{n-m}\} \in \rP_{n-m}^n$.
	Using that for $u \leq j$ we have $d_jd_u = d_ud_{j+1}$ we can show that this bijection induces a chain isomorphism as follows:
	\begin{align*}
		\bd d_U[n] &=
		\sum_{\mathclap{j \in \{0, \dots, m\}}}
		d_j\, d_{u_1} \cdots\, d_{u_{n-m}}[n] \\ &=
		\sum_{\mathclap{\bar u \in \{0, \dots, n\} \setminus U}}
		d_{u_1} \cdots\, d_{\bar{u}} \cdots\, d_{u_{n-m}}[n] \\ &=
		\sum_{\mathclap{\bar u \in \{0, \dots, n\} \setminus U}}
		d_{\{\bar u\} \union U}[n] \\ &=
		d_{\bd U}[n].
	\end{align*}
	The compatibility with cofaces and codegeneracies is left to the interested reader.
\end{proof}

\subsection{Cup-$i$ constructions via $\cP$}\label{ss:axioms_revisited}

By combining \cref{ss:naturality} and \cref{ss:equivalence_of_functors}, we can see that the data of a semi-simplicial \mbox{cup-$i$} construction $\triangle$ is a collection, indexed by $i,n \in \N$, of elements
\begin{equation*}\label{eq:cup-i_dually}
	\triangle_i [n] =
	\sum_{\mathclap{\quad \lambda \in \Lambda(i,n)}} \, V_\lambda \ot W_\lambda
\end{equation*}
in $\cP(\simplex^n)^{\ot 2}_{i+n}$
where $\Lambda(i,n)$ is a finite (possibly empty) indexing set, and $V_\lambda \ot W_\lambda$ is a basis element for each $\lambda \in \Lambda(i,n)$.
Additionally, one such construction is:
\begin{itemize}
	\item Irreducible iff\,
	$V_\lambda \cap W_\lambda = \emptyset$,
	\item Free iff\,
	$V_{\lambda} \ot W_{\lambda} \neq W_{\lambda'} \ot V_{\lambda'}$ if $i \neq n$.
\end{itemize}
To see this we notice that there is $k \in V_\lambda \cap W_\lambda$ for one of the summands of $\triangle_i [n]$ if an only the image of this summand under $\Psi^{\ot 2}$ is
\[
\underbrace{d_{V_\lambda \setminus k} \overbrace{d_k[n]}^{y}}_{y^{(1)}}
\ot
\underbrace{d_{W_\lambda \setminus k} \overbrace{d_k[n]}^{y}}_{y^{(2)}}
\]
with $(y^{(1)} \smallsmile_i y^{(2)})[n] \neq 0$.
This proves the claim about irreducibility, the one involving freeness follows from direct inspection.

\subsection{Canonical construction}

For any $U \in \rP^n_{n-i}$ the \textbf{index function} of $U$ is given by
\[
\begin{split}
	\ind_U \colon U &\to \F \\
	u_j &\mapsto u_j + j \mod 2.
\end{split}
\]
For any $U \in \rP^n_{n-i}$ and $\varepsilon \in \Ftwo \cong \{0,1\}$ we write $U^\varepsilon$ instead of $\ind_U^{-1}(\varepsilon)$.
Inspecting \cref{ss:canonical} we see that the canonical \mbox{cup-$i$} construction is defined by the elements
\[
\canonical_i[n] =
\sum_{\mathclap{\quad U \in \rP^n_{n-i}}} \ U^0 \ot U^1.
\]

\begin{theorem}\label{t:existence}
	The canonical \mbox{cup-$i$} construction is simplicial, non-zero, irreducible, and free.
\end{theorem}

\begin{proof}
	It is clearly non-zero and a straightforward application of \cref{ss:axioms_revisited} establishes its irreducibility and freeness.
	To see it is simplicial, we consider \cref{eq:codegeneracies} and an arbitrary basis element $U^0 \ot U^1$ appearing as a summand in $\canonical_i[n]$.
	Let $U = U^0 \union U^1$.
	Based on \cref{l:simplicial_from_semisimplicial}, we need to check that for any $j \in \set{0,\dots,n-1}$ we have
	\begin{equation}\label{eq:checking_simplicial}
		\cP(\sigma_j)(U^0) \ot \cP(\sigma_j)(U^1) = 0.
	\end{equation}
	If $j \notin U$ or $j+1 \notin U$ either $\cP(\sigma_j)(U^0) = 0$ or $\cP(\sigma_j)(U^1) = 0$, so Identity \eqref{eq:checking_simplicial} holds.
	Let us assume $j,j+1 \in U$.
	Since $\ind_U(j) = \ind_U(j+1)$ then either $j,j+1 \in U^0$ or $j,j+1 \in U^1$.
	In both cases we also have \eqref{eq:checking_simplicial}.
\end{proof}

\subsection{Canonical special cases}

We close with a couple of special cases of importance.
For any $n \in \N$,
\[
\canonical_n[n] = \emptyset \ot \emptyset
\]
and, if $n > 0$,
\[
\canonical_{n-1} [n] \ = \
\sum_{\mathclap{\substack{\quad u \in \{0,\dots,n\} \\ u \ \mathrm{odd}}}} \,\{u\} \ot \emptyset \ + \
\sum_{\mathclap{\substack{\quad u \in \{0,\dots,n\} \\ u \ \mathrm{even}}}} \,\emptyset \ot \{u\}.
\]