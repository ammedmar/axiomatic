% !TEX root = ../axiomatic.tex

\section{Reformulation} \label{s:reformulation}

In this section we give a more category theoretic description of the notion of \mbox{cup-$i$} construction using the functor of normalized chains.
We also introduce a functor naturally isomorphic to it -- interesting on its own right -- that simplifies the presentation of the canonical cup-$i$ construction and facilitates the proof of our main theorem.

\subsection{Natural linear transformations}

Recall the functors $\chains$ and $\chains \ot \chains$ from $\sSet$ to $\Ch$, and the notion of natural linear transformation $\chains \to \chains \ot \chains$ between them, i.e., a linear map $\chains(X) \to \chains(X) \ot \chains(X)$ for every simplicial set $X$ that is natural with respect to simplicial maps.

\begin{lemma} \label{l:coalgebra}
	A \mbox{cup-$i$} construction is canonically equivalent to a collection of natural linear transformations $\triangle_i \colon \chains \to \chains \ot \chains$ for $i \in \N$ satisfying
	\[
	\bd \circ \, \triangle_i + \triangle_i \circ \bd =
	(1+T) \triangle_{i-1}
	\]
	for all $i \geq 0$ with the convention $\triangle_{-1} = 0$.
\end{lemma}

\begin{proof}
	Let $C^\vee = \Hom(C, \F)$ with $C$ a finite dimensional chain complex.
	Using the hom-tensor adjunction and the finite dimensionality of $C$ we have
	\begin{align*}
	\Hom \big(W \ot_{\F[\Sym_2]} (C^\vee)^{\ot 2}, C^\vee \big) & \cong
	\Hom_{\F[\Sym_2]} \big( W, \Hom((C^\vee)^{\ot 2}, C^\vee) \big) \\ & \cong
	\Hom_{\F[\Sym_2]} \big( W, \Hom(C, C^{\ot 2}) \big)
	\end{align*}
	as chain complexes of $\F$-modules.
	In other words, the linear duality functor induces a bijection between \mbox{cup-$i$} product structures on $C^\vee$ and $\F[\Sym_2]$-linear chain maps $\triangle \colon W \to \Hom(C, C^{\ot 2})$.
	The latter are canonically equivalent to linear maps $\triangle_i = \triangle(e_i)$ satisfying
	\[
	\bd \circ \, \triangle_i + \triangle_i \circ \bd =
	(1+T) \triangle_{i-1}
	\]
	for all $i \geq 0$ with $\triangle_{-1} = 0$, since
	\begin{align*}
	\bd \triangle (e_i) + \triangle \bd(e_i) &=
	\bd \triangle (e_i) + \triangle (1+T) (e_{i-1}) \\ &=
	\bd \circ \, \triangle_i + \triangle_i \circ \bd \ +\ (1+T) \triangle_{i-1}.
	\end{align*}

	By naturality, a \mbox{cup-$i$} structure is determined by its restriction to representable simplicial sets $\simplex^n$.
	Since $\chains(\simplex^n)$ is finite dimensional, the previous argument involving only canonical equivalences proves the claim.
\end{proof}

Motivated by this formulation of \mbox{cup-$i$} structure we present the following.

\begin{lemma} \label{l:natural}
	A natural linear transformation $f \colon \chains \to \chains \ot \chains$ is canonically equivalent to a collection of elements $f[n] \in \chains(\simplex^n)^{\ot 2}$	for $n \in \N$ whose image under $\chains(\sigma_j)^{\ot 2}$ is $0$
	for each codegeneracy map $\sigma_j \colon [n] \to [n-1]$.
\end{lemma}

\begin{proof}
	By naturality $f$ is determined by its restriction to $\chains(\simplex^n)$ for $n \in \N$.
	Furthermore, for any non-degenerate simplex $\big( x \colon [m] \to [n] \big) \in \simplex^n_m$ we have
	\[
	f(x) = f \big( x \circ [m] \big) =
	\big( f \circ \chains(x) \big) [m] =
	\big( \chains(x) \ot \chains(x) \big) f[m],
	\]
	so the elements $f[m]$ with $m \in \N$ determine $f$.
	Here $[m]$ denotes the identity of the object $[m]$ and we are using $x$ to also denote the simplicial map $\simplex^m \to \simplex^n$ defined by $y \mapsto x \circ y$.

	The simplex associated to a codegeneracy map $\sigma_j \colon [n] \to [n-1]$ is degenerate in $\simplex^{n-1}$ so it is $0$ in $\chains(\simplex^{n-1})$.
	Therefore,
	\[
	0 = f(0) = f(\sigma_j) =
	f \big( \sigma_j \circ [n] \big) =
	\big( f \circ \chains(\sigma_j) \big) [n] =
	\big( \chains \sigma_j \ot \chains(\sigma_j) \big) f[n]
	\]
	as claimed.
\end{proof}

\subsection{The functor $\cP$}

It will be convenient to use an alternative to the functor of normalized chains which is naturally isomorphic to it.
This functor is constructed using the fact that every non-degenerate simplex of $\simplex^n$ is a face of the identity $[n]$.

\begin{definition} \label{d:dual chains}
	Let $\P_{n-m}^n$ be the set of subsets of $\{0, \dots, n\}$ whose cardinality is $n-m$.
	Define the degree $m$ part of a chain complex $\cP(\simplex^n)$ by
	\[
	\cP(\simplex^n)_m = \begin{cases}
	\F\{\P_{n-m}^n\}, & \text{if } 0 \leq m \leq n, \\
	\hfil 0, & \text{otherwise},
	\end{cases}
	\]
	and its differential by
	\begin{equation} \label{e:boundary of P}
	\bd U =
	\sum_{\mathclap{\bar u \in \{0, \dots, n\} \setminus U}} \{\bar u\} \union U.
	\end{equation}
\end{definition}

\begin{lemma}
	The linear map
	\[
	\begin{tikzcd} [column sep=small, row sep=0]
	\Psi_n \colon &[-20pt] \cP(\simplex^n) \arrow[r] & \chains(\simplex^n) \\
	& U \arrow[r, mapsto] & d_U [n]
	\end{tikzcd}
	\]
	is a chain isomorphism for every $n \in \N$.
\end{lemma}

\begin{proof}
	It can be easily seen that $\Psi_n$ induces a degree preserving bijection of basis elements.
	We will verify that this assignment induces a chain map.
	Let $U = \{u_1 < \cdots < u_{n-m}\} \in \rP_{n-m}^n$.
	Using the relation $d_jd_u = d_ud_{j+1}$ if $u \leq j$ we have
	\begin{align*}
	\bd \Psi_n(U) &=
	\bd d_U[n] =
	\sum_{\mathclap{j \in \{0, \dots, m\}}}
	d_j\, d_{u_1} \cdots\, d_{u_{n-m}}[n] \\ &=
	\sum_{\mathclap{\bar u \in \{0, \dots, n\} \setminus U}}
	d_{u_1} \cdots\, d_{\bar{u}} \cdots\, d_{u_{n-m}}[n] \\ &=
	\sum_{\mathclap{\bar u \in \{0, \dots, n\} \setminus U}}
	d_{\{\bar u\} \union U}[n] =
	\Psi_n(\bd U),
	\end{align*}
	as claimed.
\end{proof}

\begin{definition}
	Let $\cP \colon \sSet \to \Ch$ be the functor defined on standard simplicial sets as in \cref{d:dual chains} and such that for $U \in \cP(\simplex^n)_m$ and coface $\delta_j \colon [n] \to [n+1]$
	\[
	\cP(\delta_j)(U) = \{ u_1 < \dots < u_{k-1} < j < u_k+1 < \dots < u_{n-m}+1 \}
	\]
	where $k$ is determined by the inequalities, and for any codegeneracy $\sigma_j \colon [n] \to [n-1]$
	\begin{equation} \label{e:codegeneracy P}
	\cP(\sigma_j)(U) = \begin{cases}
	U \setminus \{j+1\}, & j+1 \in U, \\
	\hfil U \setminus \{j\}, & j+1 \notin U \text{ and } j \in U, \\
	\hfil 0, & j+1 \notin U \text{ and } j \notin U.
	\end{cases}
	\end{equation}
\end{definition}

\begin{lemma} \label{l:P and N}
	The chain isomorphisms $\{\Psi_n\}_{n \in \N}$ define a natural equivalence $\Psi$ between the functors $\cP$ and $\chains$.
\end{lemma}

\begin{proof}
	This follows from two straightforward computations using the cosimplicial identities \eqref{e:cosimplicial identities}.
\end{proof}

A direct consequence of \cref{e:codegeneracy P} is the following.

\begin{lemma} \label{l:kernel of sxs}
	For any codegeneracy $\sigma_j \colon [n] \to [n-1]$ a basis element $V \ot W \in \cP(\simplex^n)^{\ot 2}$ is in $\ker \big( \cP(\sigma_j)^{\ot 2} \big)$ if an only if both $j$ and $j+1$ are missing from either $V$ or $W$.
\end{lemma}

\subsection{Cup-$i$ constructions}

We can use \cref{l:coalgebra} and the natural isomorphism $\Psi$ of \cref{l:P and N} to interpret \mbox{cup-$i$} constructions in terms of natural linear transformations $\triangle_i \colon \cP \to \cP \ot \cP$ satisfying
\[
\bd \circ \, \triangle_i + \triangle_i \circ \bd =
(1+T) \triangle_{i-1}
\]
for all $i \in \N$ with the convention $\triangle_{-1} = 0$.
Additionally, by \cref{l:natural}, any $\triangle_i$ is determined by a collection of elements $\triangle_i [n] \in \cP(\simplex^n)^{\ot 2}_{i+n} $ indexed by $n \in \N$ with
\[
\triangle_i [n] =
\sum_{\mathclap{\lambda \in \Lambda(i,n)}} V_\lambda \ot W_\lambda
\]
in the kernel of $\cP(\sigma_j)^{\ot 2}$ for every codegeneracy $\sigma_j \colon [n] \to [n-1]$,
where $\Lambda(i,n)$ is a finite (possibly empty) indexing set, and $V_\lambda \ot W_\lambda$ is a basis element for each $\lambda \in \Lambda(i,n)$.

We will identify a \mbox{cup-$i$} construction and its associated set $\big\{ \triangle_i [n] \big\}_{i,n \in \N}$.

\subsection{Canonical \mbox{cup-$i$} construction}

\begin{definition}
	For any $U \in \rP^n_{n-i}$ the \textbf{index function} of $U$ is given by
	\[
	\begin{split}
	\ind_U \colon U &\to \F \\
	u_j &\mapsto u_j + j \mod 2.
	\end{split}
	\]
\end{definition}

\begin{notation*}
	For any $U \in \rP^n_{n-i}$ and $\varepsilon \in \Ftwo \cong \{0,1\}$ we write $U^\varepsilon$ instead of $\ind_U^{-1}(\varepsilon)$.
\end{notation*}

By inspecting \cref{d:my cup-i} we have the following.

\begin{lemma} \label{l:canonical}
	The canonical \mbox{cup-$i$} construction is given by the collection of elements $\Delta_i[n] \in \cP(\simplex^n)^{\ot 2}_{i+n}$ for $i,n \in \N$ with
	\[
	\Delta_i[n] =
	\sum_{\mathclap{U \in \rP^n_{n-i}}} U^0 \ot U^1
	\]
	if $i \leq n$ and $\Delta_i[n] = 0$ otherwise.
\end{lemma}