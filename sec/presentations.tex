% !TEX root = ../axiomatic.tex

\section{Other presentations} \label{s:other presentations}

We now review Steenrod's original cup-$i$ construction \cite{steenrod1947products} pag. 293, see also \cite{mcclure2003multivariable} pag. 682, and prove that it is free and non-degenerate.

\begin{definition}
	\textbf{Steenrod's cup-$i$ construction} $\Delta^{St} \colon W \to \mathcal{Z}(2)$ is defined by letting $\Delta_i^{St} \big( \id_{[n]} \big)$ be the sum over all sequences
	\[
	0 < p_1 < \cdots < p_{i+1} < n
	\]
	of the basis element
	\begin{equation} \label{equation: steenrod original i odd}
	\begin{split}
	[ 0, \dots, &{p_1} ] \ast [ {p_2}, \dots, {p_3} ] \ast \cdots \ast [ {p_{i+1}}, \dots, n ]\ \ot \\
	[ &{p_1}, \dots, {p_2} ] \ast \cdots \ast [ {p_{i}}, \dots, {p_{i+1}} ]
	\end{split}
	\end{equation}
	if $i$ is odd, and of
	\begin{equation} \label{equation: steenrod original i even}
	\begin{split}
	[ 0, \dots, &{p_1} ] \ast [ {p_2}, \dots, {p_3} ] \ast \cdots \ast [ {p_{i}}, \dots, {p_{i+1}} ]\ \ot \\
	[ &{p_1}, \dots, {p_2} ] \ast [ {p_3}, \dots, {p_4} ] \ast \cdots \ast [ {p_{i+1}}, \dots, n ]
	\end{split}
	\end{equation}
	if $i$ is even, where $\ast$ denotes the join of simplices:
	\[
	[{p_{k-1}}, \dots, {p_{k}} ] \ast [ {p_{k+1}}, \dots, p_{k+2}] = [{p_{k-1}}, \dots, p_k, p_{k+1}, \dots, p_{k+2}].
	\]
\end{definition}

\begin{proposition}
	Steenrod's cup-$i$ construction is free and non-degenerate.
\end{proposition}

\begin{proof}
	Steenrod's cup-$i$ construction if non-degenerate since $\Delta_0\big([0]\big) = [0] \ot [0]$. To prove it is free let us assume $i$ is odd with $i < n$. The $i$ even case is done analogously. Using Lemma \ref{lemma: free and non-deg in coproduct} in a proof by contradiction we can assume that there exist at least two distinct sequences
	\begin{align*}
	\begin{split}
	0 &= p_0 < p_1 < \cdots < p_{i+1} < p_{i+2} = n \\
	0 &= q_0 < q_1 \,< \cdots < q_{i+1} < q_{i+2} = n
	\end{split}
	\end{align*}
	such that
	\[
	\begin{split}
	&[ {p_0}, \dots, {p_1} ] \ast [ {p_2}, \dots, {p_3} ] \ast \cdots \ast [ {p_{i}}, \dots, {p_{i+1}} ]\ = \\
	&[ {q_1}, \dots, {q_2} ] \ast [ {q_3}, \dots, {q_4} ] \ast \cdots \ast [ {q_{i+1}}, \dots, {q_{i+2}} ]
	\end{split}
	\]
	and
	\[
	\begin{split}
	&[ {q_0}, \dots, {q_1} ] \ast [ {q_2}, \dots, {q_3} ] \ast \cdots \ast [ {q_{i}}, \dots, {q_{i+1}} ]\ = \\
	&[ {p_1}, \dots, {p_2} ] \ast [ {p_3}, \dots, {p_4} ] \ast \cdots \ast [ {p_{i+1}}, \dots, {p_{i+2}} ].
	\end{split}
	\]
	We will prove that $p_{r+1} = q_{r+1} = r$ for $0 \leq r \leq i$, in particular, this will imply the contradiction $i = n$. We have the base case of an induction argument since ${p_0} = {q_1} = {p_0} = {q_1} = 0$. The induction step follows from the identities
	\[
	\begin{split}
	[p_r] &\ast [p_{r+1}] = [q_r, q_{r}+1]\\
	[q_r] &\ast [q_{r+1}] = [p_r, p_{r}+1]
	\end{split}
	\]
	and the lemma is proven.
\end{proof}