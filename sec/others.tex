% !TEX root = ../axiomatic.tex

\section{Other formulas}\label{s:others}

In this section we first show that the canonical \mbox{cup-$i$} construction is equal to Steenrod's original (\cref{ss:original}).
We then review Real's formulas and show that the resulting simplicial \mbox{cup-$i$} construction is also equal to the canonical one (\cref{ss:real}).
The agreement of these three constructions seems to appear in print for the first time here.

\subsection{Original construction}\label{ss:original}

We now review, using the perspective developed in \cref{s:reformulation}, \textbf{Steenrod's \mbox{cup-$i$} construction} \cite[p.293]{steenrod1947products}.
If $i > n$ then $\triangle_i^\rS [n] = 0$, otherwise it is given by the sum over all ordered sequences of integers
\[
0 \leq p_1 < \cdots < p_{i+1} \leq n
\]
of the basis element
\begin{equation}\label{e:original i odd}
\begin{split}
[ 0, \dots, &{p_1} ] \ast [ {p_2}, \dots, {p_3} ] \ast \cdots \ast [ {p_{i+1}}, \dots, n ] \\
\ot \ [ &{p_1}, \dots, {p_2} ] \ast \cdots \ast [ {p_{i}}, \dots, {p_{i+1}} ]
\end{split}
\end{equation}
if $i$ is odd, and of
\begin{equation}\label{e:original i even}
\begin{split}
[ 0, \dots, &{p_1} ] \ast [ {p_2}, \dots, {p_3} ] \ast \cdots \ast [ {p_{i}}, \dots, {p_{i+1}} ] \\
\ot \ [ &{p_1}, \dots, {p_2} ] \ast [ {p_3}, \dots, {p_4} ] \ast \cdots \ast [ {p_{i+1}}, \dots, n ]
\end{split}
\end{equation}
if $i$ is even, where $\ast$ denotes the join of simplices:
\[
[{p_{k-1}}, \dots, {p_{k}} ] \ast [ {p_{k+1}}, \dots, p_{k+2}] = [{p_{k-1}}, \dots, p_k, p_{k+1}, \dots, p_{k+2}].
\]

\begin{theorem}\label{t:steenrod cup-i}
	Steenrod's \mbox{cup-$i$} construction agrees with the canonical one.
\end{theorem}

\begin{proof}
	We will first use our main result to prove that they are isomorphic.
	We can then conclude their equality by inspecting that $\triangle^\rS_i [i+1] \neq T \canonical_i [i+1]$.

	Steenrod's simplicial \mbox{cup-$i$} construction is \emph{non-zero} since $\canonical_0\big([0]\big) = [0] \ot [0] \neq 0$.
	It is \emph{irreducible} since for each basis element in $\triangle^\rS_i [n]$ with $i \leq n$, all integers $\{0, \dots, n\}$ appear in at least one of the tensor factors.
	To prove it is \emph{free} let us assume $i$ is odd with $i < n$.
	The case where $i$ is even is done analogously.
	If it is not free, then there exist two distinct sequences
	\begin{align*}
	\begin{split}
	0 &= p_0 \leq p_1 < \cdots < p_{i+1} \leq p_{i+2} = n \\
	0 &= q_0 \leq q_1 \,< \cdots < q_{i+1} \leq q_{i+2} = n
	\end{split}
	\end{align*}
	such that
	\[
	\begin{split}
	&[ {p_0}, \dots, {p_1} ] \ast [ {p_2}, \dots, {p_3} ] \ast \cdots \ast [ {p_{i}}, \dots, {p_{i+1}} ]\ = \\
	&[ {q_1}, \dots, {q_2} ] \ast [ {q_3}, \dots, {q_4} ] \ast \cdots \ast [ {q_{i+1}}, \dots, {q_{i+2}} ]
	\end{split}
	\]
	and
	\[
	\begin{split}
	&[ {q_0}, \dots, {q_1} ] \ast [ {q_2}, \dots, {q_3} ] \ast \cdots \ast [ {q_{i}}, \dots, {q_{i+1}} ]\ = \\
	&[ {p_1}, \dots, {p_2} ] \ast [ {p_3}, \dots, {p_4} ] \ast \cdots \ast [ {p_{i+1}}, \dots, {p_{i+2}} ].
	\end{split}
	\]
	We will prove that $p_{r+1} = q_{r+1} = r$ for $0 \leq r \leq i$, in particular, this will imply the contradiction $i = n$.
	We have the base case of an induction argument since $p_0 = q_1 = p_0 = q_1 = 0$.
	The induction step follows from the identities
	\[
	\begin{split}
	[p_r] &\ast [p_{r+1}] = [q_r, q_{r}+1], \\
	[q_r] &\ast [q_{r+1}] = [p_r, p_{r}+1].
	\end{split}
	\]

	Our main theorem then shows that for each $i \in \N$ either $\triangle^\rS_i = \canonical_i$ or $\triangle^\rS_i = T \canonical_i$.
	Consider the element $U = \{0\} \in \rP_{1}^{i+1}$ giving rise to the summand $U^1 \ot U^0 = \set{0} \ot \emptyset$ in $T \canonical_i [i+1]$.
	Applying the isomorphism $\cP(\simplex^n)^{\ot 2} \to \chains(\simplex^n)^{\ot 2}$ we obtain the basis element $[1,\dots,i+1] \otimes [0,\dots,i+1]$ which is not a summand of $\triangle^\rS_i [i+1]$.
	This concludes the proof.
\end{proof}

\begin{remark*}
	The following recursive description of Steenrod's original construction was introduced in \cite{medina2023dennis}:
	\begin{equation}\label{eq:prop cup-i}
		\begin{split}
			& \triangle^\rS_0 = \triangle_{\text{AW}}, \\
			& \triangle^\rS_i =
			(\ast \ot \id) \circ (\id \ot T \canonical_{i-1}) \circ \triangle_{\text{AW}}.
		\end{split}
	\end{equation}
	where $\triangle_{\text{AW}}$ is the Alexander--Whitney diagonal.
\end{remark*}

\subsection{AW--EZ contraction}\label{ss:real}

In work by Real \cite{real1996computability}, further developed by Gonz\'alez\-/D\'iaz\--Real \cite{gonzalez-diaz1999steenrod, gonzalez2003computation, gonzalez-diaz2005cocyclic}, alternative formulas defining a simplicial \mbox{cup-$i$} construction were introduced using the Alexander--Whitney and Eilenberg--Zilber linear natural transformations
\[
\AW \colon \chains(\simplex^n \times \simplex^n)
\rightleftarrows
\chains(\simplex^n) \ot \chains(\simplex^n) \,: \EZ
\]
and an explicit natural chain homotopy $\mathrm{SHI}$ between their non-trivial composition $\EZ \AW$ and the identity.
The natural linear transformations defining \textbf{Real's simplicial \mbox{cup-$i$} construction} are
\[
\triangle^{\mathrm{R}}_i = \AW (T \, \mathrm{SHI})^i.
\]

In \cite[Theorem~3.1]{gonzalez-diaz1999steenrod} these authors unravelled the above definition in terms of face maps, which we now review using the perspective developed in \cref{s:reformulation}.
If $i > n$ then $\triangle_i^{\mathrm{R}} [n] = 0$, otherwise it is given by
\begin{multline}\label{eq:real left}
	\triangle^{\mathrm{R}}_i [n] = \!
	\sum_{j_i=i}^{m} \ \sum_{j_{i-1}=i-1}^{j_i-1} \dots \sum_{j_0=0}^{j_1-1} \\
	\{j_0+1,\dots,j_1-1\} \union \{j_2+1,\dots,j_3-1\} \union \dots \union \{j_i+1,\dots,n\} \\ \ot \,
	\{0,\dots,j_0-1\} \union \{j_1+1,\dots,j_2-1\} \union \dots \union \{j_{i-1}+1,\dots,j_i-1\}
\end{multline}
if $n$ is even and by
\begin{multline}\label{eq:real right}
	\triangle^\rR_i [n] = \!
	\sum_{j_i=i}^{m} \ \sum_{j_{i-1}=i-1}^{j_i-1} \dots \sum_{j_0=0}^{j_1-1} \\
	\{j_0+1,\dots,j_1-1\} \union \{j_2+1,\dots,j_3-1\} \union \dots \union \{j_{i-1}+1,\dots,j_i-1\} \\ \ot \,
	\{0,\dots,j_0-1\} \union \{j_1+1,\dots,j_2-1\} \union \dots \union \{j_i+1,\dots,n\}
\end{multline}
if $n$ is odd.\footnote{To compare these formulas to those appearing in \cite[Theorem~3.1]{gonzalez-diaz1999steenrod} replace $i$ and $n$ respectively by $n$ and $m$.}

\begin{theorem}
	Real's \mbox{cup-$i$} construction agrees with the canonical one.
\end{theorem}

\begin{proof}
	We will use our main theorem to prove that they are isomorphic.
	We can then conclude their equality by inspecting that $\triangle^\rR_i [i+1] \neq T \canonical_i [i+1]$.
	Let us start by describing the indexing set of the sum in recursive terms.
	It consist of all integers $0 \leq j_0 < \dots < j_i \leq n$ with
	\begin{equation}\label{eq:indexing set recursively}
		j_i \in \set{i,\dots,n} \text{ and }
		i_k \in \set{k,\dots,j_{k+1}-1} \text{ for } k \in \set{0,\dots,i-1}.
	\end{equation}

	We can see that $\triangle^\rR$ is non-zero since, if $i=n=0$, the indexing set of the sum consists of a single $i_0 = 0$ and $\triangle^\rR_0 [0] = \emptyset \ot \emptyset$, using \cref{eq:real left}.

	We can see that $\triangle^\rR$ is irreducible since for each summand $V \ot W$, equal to
	\begin{align*}
		&\{j_0+1,\dots,j_1-1\} \union \{j_2+1,\dots,j_3-1\} \union \dots \union \{j_i+1,\dots,n\} \\ \ot \,
		&\{0,\dots,j_0-1\} \union \{j_1+1,\dots,j_2-1\} \union \dots \union \{j_{i-1}+1,\dots,j_i-1\}
	\end{align*}
	in the even case, or
	\begin{align*}
		&\{j_0+1,\dots,j_1-1\} \union \{j_2+1,\dots,j_3-1\} \union \dots \union \{j_{i-1}+1,\dots,j_i-1\} \\ \ot \,
		&\{0,\dots,j_0-1\} \union \{j_1+1,\dots,j_2-1\} \union \dots \union \{j_i+1,\dots,n\}
	\end{align*}
	in the odd one, we have $V \cap W = \emptyset$ given that $0 \leq j_0 < \dots < j_i \leq n$.

	To prove that $\triangle^\rR$ is free, let us focus on the case when $n$ is an even number, the odd case is analyzed similarly.
	Fix integers $0 \leq i \leq n$ and let us assume that $V \ot W = W' \ot V'$ for some pair of summands.
	After suppressing the union symbol and the commas this identity is explicitly given by
	\begin{align*}
		&\{j_0+1 \dots j_1-1\} \{j_2+1 \dots j_3-1\} \dots \{j_i+1 \dots n\} \\=
		&\{0 \dots j'_0-1\} \{j'_1+1 \dots j'_2-1\} \dots \{j'_{i-1}+1 \dots j'_i-1\}
	\end{align*}
	and
	\begin{align*}
		&\{j'_0+1 \dots j'_1-1\} \{j'_2+1 \dots j'_3-1\} \dots \{j'_i+1 \dots n\} \\=
		&\{0 \dots j_0-1\} \{j_1+1 \dots j_2-1\} \dots \{j_{i-1}+1 \dots j_i-1\}.
	\end{align*}
	Using the first of these identities we have that: if $j_i+1 \leq n$ then $n \in \{j_i+1 \dots n\}$ so $j'_i-1 \geq n$, which is impossible since $j'_i \in \set{i,\dots,n}$.
	Therefore, $j_i = n$.
	Using the other identity the same argument shows that $j'_i = n$.
	Let us consider this to be the base case of an induction argument on $k = i,i-1,\dots,0$, with hypothesis $j_k = j'_k = n-i+k$.
	Let us consider the induction step from $k$ to $k-1$.
	In this case,
	\[
	\set{j_{k+\ell}+1,\dots,j_{k+\ell+1}-1} = \emptyset = \set{j'_{k+\ell}+1,\dots,j'_{k+\ell+1}-1}
	\]
	for all $\ell \in \set{0,\dots,i-k-1}$.
	Let us assume $k$ is even.
	The first identity becomes
	\begin{align*}
		&\{j_0+1 \dots j_1-1\} \{j_2+1 \dots j_3-1\} \dots \{j_{k-2}+1 \dots j_{k-1}-1\} \\=
		&\{0 \dots j'_0-1\} \{j'_1+1 \dots j'_2-1\} \dots \{j'_{k-1}+1 \dots j'_k-1\}.
	\end{align*}
	If $j'_{k-1}+1 \leq j'_k-1$ then $j'_k - 1 = (n-i+k) - 1 \in \{j'_{k-1}+1 \dots j'_k-1\}$ so $(n-i+k)-1 = j_k - 1 \geq j_{k-1}-1$, which is impossible by \cref{eq:indexing set recursively}, so $j'_{k-1} = n-i+k-1$ by the same equation.
	A similar argument using the other identity shows that $j_{k-1} = n-i+k-1$.
	If $k$ is odd a variation of this argument completes the induction step.
	Finally, using that $j_0 = j'_0 = n-i$, the first identity reduces to
	\[
	\emptyset = \set{0,\dots, n-i-1}
	\]
	implying $n-i \leq 0$, and, since $n-i \geq 0$, we conclude $n=i$.

	It remains to be shown that $\triangle^\rR$ is not only isomorphic to the canonical \mbox{cup-$i$} construction $\canonical$, but equal.
	To do so, notice that for every non-negative integer $i$ the summand in $\triangle_i^\rR[i+1]$ associated the indexing element with $j_k = k+1$ for each $k \in \{0,\dots,i\}$ is $\emptyset \ot \{0\}$ which is not a summand present in $T \canonical_i[i+1]$.
	Therefore, $\triangle_i^\rR \neq T \canonical_i$ for every $i \geq 0$, so $\triangle^\rR = \canonical$ as claimed.
\end{proof}