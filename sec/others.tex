% !TEX root = ../axiomatic.tex

\section{Other constructions} \label{s:others}

\subsection{Original construction} \label{ss:original}

We now review from \cite[p.293]{steenrod1947products} \textbf{Steenrod's \mbox{cup-$i$} construction} (see also \cite[p.682]{mcclure2003multivariable}).
Following \cref{l:cup-i construction coalgebra}, we will describe it as a set of elements $\triangle_i^{\mathrm{S}} [n] \in \chains(\simplex^n)^{\ot 2}_{i+n}$ with $i,n \in \N$.
If $i > n$ then $\triangle_i^{\mathrm{S}} [n] = 0$, otherwise it is given by the sum over all ordered sequences of integers
\[
0 \leq p_1 < \cdots < p_{i+1} \leq n
\]
of the basis element
\begin{equation} \label{e:steenrod original i odd}
\begin{split}
[ 0, \dots, &{p_1} ] \ast [ {p_2}, \dots, {p_3} ] \ast \cdots \ast [ {p_{i+1}}, \dots, n ] \\
\ot \ [ &{p_1}, \dots, {p_2} ] \ast \cdots \ast [ {p_{i}}, \dots, {p_{i+1}} ]
\end{split}
\end{equation}
if $i$ is odd, and of
\begin{equation} \label{e:steenrod original i even}
\begin{split}
[ 0, \dots, &{p_1} ] \ast [ {p_2}, \dots, {p_3} ] \ast \cdots \ast [ {p_{i}}, \dots, {p_{i+1}} ] \\
\ot \ [ &{p_1}, \dots, {p_2} ] \ast [ {p_3}, \dots, {p_4} ] \ast \cdots \ast [ {p_{i+1}}, \dots, n ]
\end{split}
\end{equation}
if $i$ is even, where $\ast$ denotes the join of simplices:
\[
[{p_{k-1}}, \dots, {p_{k}} ] \ast [ {p_{k+1}}, \dots, p_{k+2}] = [{p_{k-1}}, \dots, p_k, p_{k+1}, \dots, p_{k+2}].
\]

\begin{theorem} \label{t:steenrod cup-i}
	Steenrod's \mbox{cup-$i$} construction agrees with the canonical one.
\end{theorem}

\begin{proof}
	We will use \cref{t:main} prove that they are isomorphic.
	We can then conclude their equality by inspecting that $\triangle^{\mathrm{S}}_i [i+1] \neq \Delta_i [i+1]$.

	Let us verify that Steenrod's \mbox{cup-$i$} construction is free, non-degenerate and irreducible.
	To prove it is \emph{free} let us assume $i$ is odd with $i < n$.
	The case where $i$ is even is done analogously.
	it it is not, then there exist two distinct sequences
	\begin{align*}
	\begin{split}
	0 &= p_0 \leq p_1 < \cdots < p_{i+1} \leq p_{i+2} = n \\
	0 &= q_0 \leq q_1 \,< \cdots < q_{i+1} \leq q_{i+2} = n
	\end{split}
	\end{align*}
	such that
	\[
	\begin{split}
	&[ {p_0}, \dots, {p_1} ] \ast [ {p_2}, \dots, {p_3} ] \ast \cdots \ast [ {p_{i}}, \dots, {p_{i+1}} ]\ = \\
	&[ {q_1}, \dots, {q_2} ] \ast [ {q_3}, \dots, {q_4} ] \ast \cdots \ast [ {q_{i+1}}, \dots, {q_{i+2}} ]
	\end{split}
	\]
	and
	\[
	\begin{split}
	&[ {q_0}, \dots, {q_1} ] \ast [ {q_2}, \dots, {q_3} ] \ast \cdots \ast [ {q_{i}}, \dots, {q_{i+1}} ]\ = \\
	&[ {p_1}, \dots, {p_2} ] \ast [ {p_3}, \dots, {p_4} ] \ast \cdots \ast [ {p_{i+1}}, \dots, {p_{i+2}} ].
	\end{split}
	\]
	We will prove that $p_{r+1} = q_{r+1} = r$ for $0 \leq r \leq i$, in particular, this will imply the contradiction $i = n$.
	We have the base case of an induction argument since $p_0 = q_1 = p_0 = q_1 = 0$.
	The induction step follows from the identities
	\[
	\begin{split}
	[p_r] &\ast [p_{r+1}] = [q_r, q_{r}+1], \\
	[q_r] &\ast [q_{r+1}] = [p_r, p_{r}+1].
	\end{split}
	\]
	Steenrod's \mbox{cup-$i$} construction is \emph{non-degenerate} since $\Delta_0\big([0]\big) = [0] \ot [0] \neq 0$.
	It is \emph{irreducible} since for each summand all integers $\{0, \dots, n\}$ appear in at least one of the tensor factors.

	Consider the element $U = \{0\} \in \rP_{1}^{i+1}$ giving rise to the summand $U^1 \ot U^0 = \{0\} \ot \emptyset$ in $T \Delta_i [i+1]$.
	Applying the isomorphism $\Psi^{\ot 2} \colon \cP(\simplex^n)^{\ot 2} \to \chains(\simplex^n)^{\ot 2}$ we obtain the basis element $[1,\dots,i+1] \otimes [0,\dots,i+1]$ which is not a summand of $\triangle^{\mathrm{S}}_i [i+1]$.
	This concludes the proof.
\end{proof}

\subsection{AW--EZ contraction} \label{ss:real}

In work by Real \cite{real1996computability} and Gonz\'alez-D\'iaz--Real \cite{gonzalez-diaz1999steenrod, gonzalez2003computation, gonzalez-diaz2005cocyclic}
an alternative \mbox{cup-$i$} construction was developed based on the Alexander--Whitney and Eilenberg--Zilber linear natural transformations
\[
\AW \colon \chains(\simplex^n \times \simplex^n)
\rightleftarrows
\chains(\simplex^n) \ot \chains(\simplex^n) \,: \EZ
\]
and an explicit natural chain homotopy $\mathrm{SHI}$ between their non-trivial composition and the identity.
The natural linear transformations defining the \textbf{Gonz\'alez-D\'iaz--Real's \mbox{cup-$i$} construction} (\cref{l:cup-i construction coalgebra}) are
\[
\triangle^{\mathrm{GR}}_i = \AW (T \, \mathrm{SHI})^i.
\]

In \cite[Corollary 3.2]{gonzalez-diaz1999steenrod} these authors unravelled the above definition in terms of face maps.
We use \cref{l:natural linear map,l:natural equivalence P and N} to describe these formulas as a set of elements $\triangle_i^{\mathrm{GR}} [n] \in \cP(\simplex^n)^{\ot 2}_{i+n}$ with $i,n \in \N$.
If $i > n$ then $\triangle_i^{\mathrm{GR}} [n] = 0$, otherwise it is given by
\begin{multline*}
\triangle^{\mathrm{GR}}_i [n] = \!
\sum_{j_i=S(i)}^{n} \ \sum_{j_{i-1}=S(i-1)}^{j_i-1} \dots \sum_{j_1=S(1)}^{j_2-1} \\
\{j_0+1 < \dots < j_1-1\} \union \{j_2+1 < \dots < j_3-1\} \union \dots \union \{j_i+1 < \dots < j_n\}\\ \ot \,
\{0 < \dots < j_0-1\} \union \{j_1+1 < \dots < j_2-1\} \union \dots \union \{j_{i-1}+1 < \dots < j_i-1\}
\end{multline*}
if $n$ is even and by
\begin{multline*}
\triangle^{\mathrm{GR}}_i [n] = \!
\sum_{j_i=S(i)}^{n} \ \sum_{j_{i-1}=S(i-1)}^{j_i-1} \dots \sum_{j_1=S(1)}^{j_2-1} \\
\{j_0+1 < \dots < j_1-1\} \union \{j_2+1 < \dots < j_3-1\} \union \dots \union \{j_{i-1}+1 < \dots < j_i-1\} \\ \ot \,
\{0 < \dots < j_0-1\} \union \{j_1+1 < \dots < j_2-1\} \union \dots \union \{j_i+1 < \dots < j_n\}
\end{multline*}
if $n$ is odd, where
\[
S(k) = j_{k+1} - j_{k+2} + \dots + (-1)^{k+i-1} j_i + (-1)^{k+i} \left\lfloor \frac{n+1}{2} \right\rfloor + \left\lfloor \frac{k}{2} \right\rfloor .
\]

\begin{remark}
	To compare the above formulas to those appearing in \cite[Corollary 3.2]{gonzalez-diaz1999steenrod} we mention that $i$ and $n$ here are respectively equal to $j-i=n$ and $i+j=m$ there.
\end{remark}

A small variation of the proof given for \cref{t:steenrod cup-i} establishes the following.

\begin{theorem}
	Gonz\'alez-D\'iaz--Real's \mbox{cup-$i$} construction agrees with the canonical one.
\end{theorem}

\subsection{Operads} \label{ss:operads}

In this section, independent of the main results of this work, we assume familiarity with the theory of operads over the category of chain complexes (of $\Ftwo$-modules).
After reviewing natural $E_\infty$-coalgebra structures on the normalized chains of simplicial sets introduced by McClure--Smith \cite{mcclure2003multivariable}, Berger--Fresse \cite{berger2004combinatorial}, and the author \cite{medina2020prop1,medina2021prop2}, we will describe associated \mbox{cup-$i$} constructions satisfying our axioms.

%We also assume familiarity with the notion of $E_\infty$-operad.
%The main example for us is the \textbf{Eilenberg--Zilber operad} $\EZop$ whose arity $r$ part $\EZop(r)$ is given by all natural linear transformation from $\chains$ to $\chains^{\ot r}$.
%We will describe two suboperads defining an $E_\infty$-coalgebra structure on chains.

\subsubsection{}

An \textbf{$\Med$-bialgebra} is a chain complex $B$ with three operations
\[
\Delta \colon B \to B \ot B, \qquad
\varepsilon \colon B \to \Ftwo, \qquad
\ast \colon B \ot B \to B,
\]
such that the first two define the structure of a counital coalgebra on $B$, and the third one satisfies
\begin{gather*}
\varepsilon \circ \ast = 0, \\
\bd \circ \ast + \ast \circ (\bd \ot \, \id) + \ast \circ (\id \ot \bd) =
\varepsilon \ot \id + \id \ot \varepsilon.
\end{gather*}
As proven in \cite{medina2020prop1}, any $\Med$-bialgebra $B$ is an $E_\infty$-coalgebra by discarding compositions of the generators defining maps of the form $B^{\ot s} \to B^{\ot r}$ for $s \neq 1$.
The $E_\infty$-operad controlling this structure is denoted $\UM$.

The complex of normalized chains of a standard simplicial set is naturally an $\Med$-bialgebra with the Alexander--Whitney coproduct $\Delta$, its counit $\varepsilon$ -- the augmentation map -- and the \textbf{join product} $\ast \colon \chains^{\ot 2}(\simplex^n) \to \chains(\simplex^n)$ defined by
\[
\ast \big(\left[v_0, \dots, v_p \right] \ot \left[v_{p+1}, \dots, v_q\right]\big) =
\begin{cases} \left[v_{\pi(0)}, \dots, v_{\pi(q)}\right] & \text{ if } v_i \neq v_j \text{ for } i \neq j, \\
\hfil 0 & \text{ if not}, \end{cases}
\]
where $\pi$ is the permutation that orders the vertices.
The natural $\UM$-coalgebra structure on $\chains(\simplex^n)$ extends to $\chains(X)$ for every simplicial set $X$.
Using this $E_\infty$-coalgebra structure, a \mbox{cup-$i$} construction is recursively defined by
\begin{equation} \label{e:prop cup-i}
\begin{split}
& \copr_0 = \copr, \\
& \copr_i =
(\ast \ot \id) \circ (\id \ot T \copr_{i-1}) \circ \copr.
\end{split}
\end{equation}
We can directly compare \cref{e:prop cup-i} to \cref{e:steenrod original i odd, e:steenrod original i even} to conclude that this \mbox{cup-$i$} construction agrees with Steenrod's original.

\subsubsection{}

The Alexander--Whitney coproduct and the join product are coassociative and associative respectively and we write
\begin{align*}
\Delta^1 &= \Delta, &
\ast^1 &= \ast, \\
\Delta^{m+1} &= (\Delta^m \ot \, \id) \circ \Delta, &
\ast^{m+1} &= \ast \circ (\ast^m \ot \, \id).
\end{align*}

Consider a surjection $s \colon \{1, \dots, \ell\} \to \{1, \dots, r\}$ which we represent by its order image $\big( s(1), \dots, s(\ell) \big)$.
The natural linear transformation associated to $s$ is
\begin{equation} \label{e:surjection action}
\Big( \ast^{\bars{s^{-1}(1)}} \ot \dotsb \ot \ast^{\bars{s^{-1}(r)}} \Big) \circ \pi_s \circ \Delta^{\ell-1}
\end{equation}
where $\ast^0 = \id$ and $\pi_s$ is the shuffle permutation defined by
\[
\big( \pi s(1), \dots, \pi s(\ell) \big) =
\big( 1, \dots, 1, \dots, r, \dots, r \big).
\]
This assignment defines the surjection operad $\cX$ of McClure--Smith \cite{mcclure2003multivariable} as a suboperad of the endomorphism operad of $\chains$.
Furthermore, it is clear that for any simplicial set $X$ the $\cX$-coalgebra on $\chains(X)$ is controlled by its $\UM$-coalgebra structure i.e., by the Alexander--Whitney coproduct and the join product.
Additionally, we can inspect \cref{e:prop cup-i} to conclude that the surjections $\big\{ (1,2), (1,2,1), (1,2,1,2), \dots \big\}$ define Steenrod's original \mbox{cup-$i$} construction.
In fact, $\cX(2)$ is isomorphic to $W$ so its set of bases is in bijection with the set of a \mbox{cup-$i$} constructions satisfying our axioms.

\subsubsection{}

Let $G$ be a finite group.
Its unreduced bar resolution is the simplicial set
\[
\begin{split}
EG_m &= \big\{ (g_0, \dots, g_m) \mid g_i \in G \big\}, \\
\face_j(g_0, \dots, g_m) &= (g_0, \dots, \widehat g_j, \dots, g_m), \\
\dege_j(g_0, \dots, g_m) &= (g_0, \dots, g_j, g_j, \dots, g_m).
\end{split}
\]
The partial composition of permutations
\[
\circ_j \colon \Sym_p \times \Sym_q \to \Sym_{p+q-1}
\]
induces an operad structure on the collection of normalized chains $\BE(r) = \chains(E \Sym_r)$.
This $E_\infty$-operad, introduced by Berger--Fresse \cite{berger2004combinatorial}, is referred to as the \textbf{Barratt--Eccles operad} and is here denoted by $\BE$.
These authors also define a quasi-isomorphism of operads $\mathrm{TR} \colon \BE \to \cX$ and use it to associate to any simplex $(\sigma_0, \dots, \sigma_m) \in E \Sym_r$ a natural linear transformation $\chains \to \chains^{\ot r}$.
In particular, for any simplicial set $X$ the $\BE$-coalgebra structure on $\chains(X)$ is controlled by its $\UM$-coalgebra structure, i.e., by the Alexander--Whitney coproduct and the join product.
Since for the isomorphism $\mathrm{TR} \colon \BE(2) \to \cX(2)$ we have
\[
\mathrm{TR}(\id, T, \id, \dots, T^i) =
\begin{cases}
(1,2,\dots,2,1) & \text{ if } m \text{ is even}, \\
(1,2,\dots,2,1) & \text{ if } m \text{ is odd},
\end{cases}
\]
the elements $\big\{ (\id, T, \id, \dots, T^i) \big\}_{i \in \N}$ define a \mbox{cup-$i$} construction that agrees with Steenrod's, and the set of bases of $\BE(2)$ is in bijection with the set of a \mbox{cup-$i$} constructions satisfying our axioms.