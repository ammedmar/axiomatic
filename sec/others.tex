% !TEX root = ../axiomatic.tex

\section{Other formulas defining cup-$i$ products}\label{s:others}

In this section we first show that the canonical \mbox{cup-$i$} construction is equal to Steenrod's original (\cref{ss:original}).
We then review Real's formulas and show that the resulting \mbox{cup-$i$} construction is also equal to the canonical one (\cref{ss:real}).
The agreement of these three constructions seems to appear in print for the first time here.

\subsection{Original construction}\label{ss:original}

We now review \textbf{Steenrod's \mbox{cup-$i$} construction} \cite[p.293]{steenrod1947products}.
Following \cref{l:coalgebra}, we will describe it as a set of elements $\triangle_i^{\mathrm{S}} [n] \in \chains(\simplex^n)^{\ot 2}_{i+n}$ with $i,n \in \N$.
If $i > n$ then $\triangle_i^{\mathrm{S}} [n] = 0$, otherwise it is given by the sum over all ordered sequences of integers
\[
0 \leq p_1 < \cdots < p_{i+1} \leq n
\]
of the basis element
\begin{equation}\label{e:original i odd}
\begin{split}
[ 0, \dots, &{p_1} ] \ast [ {p_2}, \dots, {p_3} ] \ast \cdots \ast [ {p_{i+1}}, \dots, n ] \\
\ot \ [ &{p_1}, \dots, {p_2} ] \ast \cdots \ast [ {p_{i}}, \dots, {p_{i+1}} ]
\end{split}
\end{equation}
if $i$ is odd, and of
\begin{equation}\label{e:original i even}
\begin{split}
[ 0, \dots, &{p_1} ] \ast [ {p_2}, \dots, {p_3} ] \ast \cdots \ast [ {p_{i}}, \dots, {p_{i+1}} ] \\
\ot \ [ &{p_1}, \dots, {p_2} ] \ast [ {p_3}, \dots, {p_4} ] \ast \cdots \ast [ {p_{i+1}}, \dots, n ]
\end{split}
\end{equation}
if $i$ is even, where $\ast$ denotes the join of simplices:
\[
[{p_{k-1}}, \dots, {p_{k}} ] \ast [ {p_{k+1}}, \dots, p_{k+2}] = [{p_{k-1}}, \dots, p_k, p_{k+1}, \dots, p_{k+2}].
\]

\begin{theorem}\label{t:steenrod cup-i}
	Steenrod's \mbox{cup-$i$} construction agrees with the canonical one.
\end{theorem}

\begin{proof}
	We will use \cref{t:main} to prove that they are isomorphic.
	We can then conclude their equality by inspecting that $\triangle^{\mathrm{S}}_i [i+1] \neq T \Delta_i [i+1]$.

	Steenrod's \mbox{cup-$i$} construction is \emph{non-degenerate} since $\Delta_0\big([0]\big) = [0] \ot [0] \neq 0$.
	It is \emph{irreducible} since for each basis element in $\triangle^{\mathrm{S}}_i [n]$ with $i \leq n$, all integers $\{0, \dots, n\}$ appear in at least one of the tensor factors.
	To prove it is \emph{free} let us assume $i$ is odd with $i < n$.
	The case where $i$ is even is done analogously.
	If it is not free, then there exist two distinct sequences
	\begin{align*}
	\begin{split}
	0 &= p_0 \leq p_1 < \cdots < p_{i+1} \leq p_{i+2} = n \\
	0 &= q_0 \leq q_1 \,< \cdots < q_{i+1} \leq q_{i+2} = n
	\end{split}
	\end{align*}
	such that
	\[
	\begin{split}
	&[ {p_0}, \dots, {p_1} ] \ast [ {p_2}, \dots, {p_3} ] \ast \cdots \ast [ {p_{i}}, \dots, {p_{i+1}} ]\ = \\
	&[ {q_1}, \dots, {q_2} ] \ast [ {q_3}, \dots, {q_4} ] \ast \cdots \ast [ {q_{i+1}}, \dots, {q_{i+2}} ]
	\end{split}
	\]
	and
	\[
	\begin{split}
	&[ {q_0}, \dots, {q_1} ] \ast [ {q_2}, \dots, {q_3} ] \ast \cdots \ast [ {q_{i}}, \dots, {q_{i+1}} ]\ = \\
	&[ {p_1}, \dots, {p_2} ] \ast [ {p_3}, \dots, {p_4} ] \ast \cdots \ast [ {p_{i+1}}, \dots, {p_{i+2}} ].
	\end{split}
	\]
	We will prove that $p_{r+1} = q_{r+1} = r$ for $0 \leq r \leq i$, in particular, this will imply the contradiction $i = n$.
	We have the base case of an induction argument since $p_0 = q_1 = p_0 = q_1 = 0$.
	The induction step follows from the identities
	\[
	\begin{split}
	[p_r] &\ast [p_{r+1}] = [q_r, q_{r}+1], \\
	[q_r] &\ast [q_{r+1}] = [p_r, p_{r}+1].
	\end{split}
	\]

	\cref{t:main} proves that $\triangle^{\mathrm{S}}_i = \Delta_i$ or $\triangle^{\mathrm{S}}_i = T \Delta_i$ for any $i \in \N$.
	Consider the element $U = \{0\} \in \rP_{1}^{i+1}$ giving rise to the summand $U^1 \ot U^0 = \{0\} \ot \emptyset$ in $T \Delta_i [i+1]$.
	Applying the isomorphism $\Psi^{\ot 2} \colon \cP(\simplex^n)^{\ot 2} \to \chains(\simplex^n)^{\ot 2}$ we obtain the basis element $[1,\dots,i+1] \otimes [0,\dots,i+1]$ which is not a summand of $\triangle^{\mathrm{S}}_i [i+1]$.
	This concludes the proof.
\end{proof}

A recursive formula for Steenrod's original construction which was introduced in \cite{medina2022dennis} is the following:
\begin{equation}\label{e:prop cup-i}
	\begin{split}
		& \Delta_0 = \Delta, \\
		& \Delta_i =
		(\ast \ot \id) \circ (\id \ot T \Delta_{i-1}) \circ \Delta.
	\end{split}
\end{equation}
%	We can directly compare \cref{e:prop cup-i} to \cref{e:original i odd,e:original i even} to conclude that this \mbox{cup-$i$} construction agrees with Steenrod's original.

\subsection{AW--EZ contraction}\label{ss:real}

In work by Real \cite{real1996computability}, further developed by Gonz\'alez\-/D\'iaz\--Real \cite{gonzalez-diaz1999steenrod, gonzalez2003computation, gonzalez-diaz2005cocyclic}, alternative formulas defining a \mbox{cup-$i$} construction were introduced using the Alexander--Whitney and Eilenberg--Zilber linear natural transformations
\[
\AW \colon \chains(\simplex^n \times \simplex^n)
\rightleftarrows
\chains(\simplex^n) \ot \chains(\simplex^n) \,: \EZ
\]
and an explicit natural chain homotopy $\mathrm{SHI}$ between their non-trivial composition $\EZ \AW$ and the identity.
The natural linear transformations defining \textbf{Real's \mbox{cup-$i$} construction} are
\[
\triangle^{\mathrm{R}}_i = \AW (T \, \mathrm{SHI})^i.
\]

In \cite[Corollary 3.2]{gonzalez-diaz1999steenrod} these authors unravelled the above definition in terms of face maps.
We use \cref{l:natural,l:P and N} to describe these formulas as a set of elements $\triangle_i^{\mathrm{R}} [n] \in \cP(\simplex^n)^{\ot 2}_{i+n}$ with $i,n \in \N$.
If $i > n$ then $\triangle_i^{\mathrm{R}} [n] = 0$, otherwise it is given by
\begin{multline*}
\triangle^{\mathrm{R}}_i [n] = \!
\sum_{j_i=S(i)}^{n} \ \sum_{j_{i-1}=S(i-1)}^{j_i-1} \dots \sum_{j_1=S(1)}^{j_2-1} \\
\{j_0+1 < \dots < j_1-1\} \union \{j_2+1 < \dots < j_3-1\} \union \dots \union \{j_i+1 < \dots < j_n\} \\ \ot \,
\{0 < \dots < j_0-1\} \union \{j_1+1 < \dots < j_2-1\} \union \dots \union \{j_{i-1}+1 < \dots < j_i-1\}
\end{multline*}
if $n$ is even and by
\begin{multline*}
\triangle^{\mathrm{R}}_i [n] = \!
\sum_{j_i=S(i)}^{n} \ \sum_{j_{i-1}=S(i-1)}^{j_i-1} \dots \sum_{j_1=S(1)}^{j_2-1} \\
\{j_0+1 < \dots < j_1-1\} \union \{j_2+1 < \dots < j_3-1\} \union \dots \union \{j_{i-1}+1 < \dots < j_i-1\} \\ \ot \,
\{0 < \dots < j_0-1\} \union \{j_1+1 < \dots < j_2-1\} \union \dots \union \{j_i+1 < \dots < j_n\}
\end{multline*}
if $n$ is odd, where
\[
S(k) = j_{k+1} - j_{k+2} + \dots + (-1)^{k+i-1} j_i + (-1)^{k+i} \left\lfloor \frac{n+1}{2} \right\rfloor + \left\lfloor \frac{k}{2} \right\rfloor .
\]

\begin{remark}
	To compare the above formulas to those appearing in \cite[Corollary 3.2]{gonzalez-diaz1999steenrod} we mention that $i$ and $n$ here are respectively equal to $j-i=n$ and $i+j=m$ there.
\end{remark}

A small variation of the proof given for \cref{t:steenrod cup-i} establishes the following.

\begin{theorem}
	Real's \mbox{cup-$i$} construction agrees with the canonical one.
\end{theorem}

