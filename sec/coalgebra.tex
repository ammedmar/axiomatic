% !TEX root = ../axiomatic.tex

\subsection{Pre-dual version}

We will now use the functors $\chains$ and $\chains \ot \chains$ from $\sSet$ to $\Ch$ to present an alternative definition of the notion of cup-$i$ construction, and use it to state an equivalent version of \cref{t:main}.

\begin{lemma} \label{l:cup-i construction coalgebra}
	A cup-$i$ construction is canonically equivalent to a collection of natural linear maps
	\[
	\triangle_i \colon \chains \to \chains \ot \chains
	\]
	for $i \in \N$ with $\Delta_0$ a chain map and
	\[
	\bd \circ \triangle_i + \triangle_i \circ \bd =
	(1+T) \triangle_{i-1}
	\]
	for $i > 0$.
\end{lemma}

\begin{proof}
	By naturality, a cup-$i$ construction is determined by its restriction to representable simplicial sets.

	Let $C^\vee = \Hom(C, \F)$ with $C$ a finite dimensional chain complex.
	The relevant case for us is $C = \chains \simplex^n$ for $n \in \N$.
	Using the hom-tensor adjunction and the finite dimensionality of $C$ we have
	\begin{align*}
	\Hom \big(W \ot_{\F[\Sym_2]} (C^\vee)^{\ot 2}, C^\vee \big) & \cong
	\Hom_{\F[\Sym_2]} \big( W, \Hom((C^\vee)^{\ot 2}, C^\vee) \big) \\ & \cong
	\Hom_{\F[\Sym_2]} \big( W, \Hom(C, C^{\ot 2}) \big)
	\end{align*}
	as chain complexes of $\F$-modules.
	In other words, the linear duality functor induces a bijection between symmetric products on $C^\vee$ and $\F[\Sym_2]$-linear chain maps $\triangle \colon W \to \Hom(C, C^{\tensor 2})$.
	The latter are equivalent to linear maps $\triangle_i = \triangle(e_i)$ with $\triangle_0$ a chain map and satisfying
	\[
	\bd \circ \triangle_i + \triangle_i \circ \bd =
	(1+T) \triangle_{i-1}
	\]
	for all $i > 0$ since
	\begin{align*}
	\bd \triangle (e_i) + \triangle \bd(e_i) &=
	\bd \triangle (e_i) + \triangle (1+T) (e_{i-1}) \\ &=
	\bd \circ \triangle_i + \triangle_i \circ \bd \ +\ (1+T) \triangle_{i-1}.
	\end{align*}
\end{proof}

We will refer to a cup-$i$ construction and to its associated collection of natural linear maps $\triangle_i \colon \chains \to \chains \ot \chains$ interchangeably.

\begin{remark}
	For any two cochains $\alpha$ and $\beta$ we have after unraveling the isomorphisms in the proof above that $\alpha \cup_i \beta = (\alpha \ot \beta) \triangle_i(-)$.
\end{remark}

\begin{lemma}
	A natural linear map $f \colon \chains \to \chains \ot \chains$ is canonically equivalent to a collection of elements
	\[
	f[n] \in \chains \simplex^n \ot \chains \simplex^n
	\]
	for $n \in \N$ such that
	\[
	(\chains \sigma_j \ot \chains \sigma_j) f[n] = 0
	\]
	for each codegeneracy map $\sigma_j \colon [n] \to [n-1]$.
\end{lemma}

\begin{proof}
	By naturality $f$ is determined by its restriction to $\chains \simplex^n$ for $n \in \N$.
	Furthermore, for any non-degenerate simplex $(x \colon [m] \to [n]) \in \simplex^n_m$ we have
	\[
	f(x) = f \big( x \circ [m] \big) =
	(f \circ \chains x) [m] =
	(\chains x \ot \chains x) f [m],
	\]
	so the elements $f[m]$ with $m \in \N$ determine $f$.
	Here $[m]$ denotes the identity of the object $[m]$ and we are using $x$ to also denote the simplicial map $\simplex^m \to \simplex^n$ defined by $y \mapsto x \circ y$.

	The simplex associated to a codegeneracy map $\sigma_j \colon [n] \to [n-1]$ is degenerate in $\simplex^{n-1}$ so it is $0$ in $\chains \simplex^{n-1}$.
	Therefore,
	\[
	0 = f(0) = f(\sigma_j) =
	f \big( \sigma_j \circ [n] \big) =
	(f \circ \chains \sigma_j) [n] =
	(\chains \sigma_j \ot \chains \sigma_j) f [n]
	\]
	as claimed.
\end{proof}

We record the following direct consequence for later reference:

\begin{lemma} \label{l:condition to be in the kernel of s}
	A basis element $\delta_V \ot \delta_W$ is in the kernel of $(s_j)^{\ot 2}$ if and only if both $j$ and $j+1$ are missing from either $V$ or $W$.
\end{lemma}

\begin{definition}
	Let $\Delta_i$ be the natural linear map defined by the elements
	\[
	\Delta_i [n] =
	\sum_{U \in \P_{n-i}(n)} \delta_{U^0} \ot \delta_{U^1}
	\]
	in $\chains \simplex^n \ot \chains \simplex^n$
	for $i \leq n$ and $\Delta_i [n] = 0$ otherwise.
\end{definition}

We now state an equivalent formulation of Theorem \ref{t:main}

\begin{theorem} \label{t:main reformulated}
	If $\triangle$ is a free non-degenerate irreducible cup-$i$ construction, then, for every $i \in \N$, either $\triangle_i = \Delta_i$ or $\triangle_i = T \Delta_i$.
\end{theorem}
