\subsection{Freeness revisited}

%The freeness of $\triangle$ can be expressed via the following

\begin{lemma} \label{lemma: freeness recasted}
	If
	\[
	\triangle_i(\id_{[n]}) = \sum_{\Lambda} \delta_{V_\lambda} \otimes \delta_{W_\lambda}
	\]
	for some $i < n$, then
	\[
	\delta_{V_{\lambda_1}} \otimes \delta_{W_{\lambda_1}} \neq
	\delta_{W_{\lambda_2}} \otimes \delta_{V_{\lambda_2}}
	\]
	for every $\lambda_1, \lambda_2 \in \Lambda$.
\end{lemma}

%A consequence of this lemma is

\begin{lemma} \label{lemma: (1+T) triangle = 0 implies triangle = 0}
    If
    \[
    (1 + T) \triangle_i(\id_{[n]}) = (1 + T) \sum_{\Lambda} \delta_{V_\lambda} \otimes \delta_{W_\lambda}
    \]
    for some $i < n$, then there exists a partition of $\Lambda = \Lambda_1 \sqcup \Lambda_2$ such that
    \begin{equation} \label{equation: splitting after 1+T}
    \triangle_i(\id_{[n]}) = \sum_{\Lambda_1} \delta_{V_\lambda} \otimes \delta_{W_\lambda} + \sum_{\Lambda_2} \delta_{W_\lambda} \otimes \delta_{V_\lambda}.
    \end{equation}
\end{lemma}

%\begin{proof}
%	We directly have that \eqref{equation: splitting after 1+T} holds up to an element in the kernel of $(1+T)$. This kernel is generated by elements of the form $\delta_U \otimes \delta_U$ and $\delta_V \otimes \delta_W + \delta_W \otimes \delta_V$. Therefore, Lemma \ref{lemma: freeness recasted} implies the element in the kernel must be $0$.
%\end{proof}