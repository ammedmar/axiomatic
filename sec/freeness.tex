\subsection{}

In this subsection we use $\cP \otimes \cP$ to present a characterization of the freeness property of cup-$i$ constructions and a useful consequence of it.

\begin{lemma} \label{l:freeness recasted}
	A cup-$i$ construction
	\[
	\triangle_i [n] = \sum_{\lambda \in \Lambda(i,n)} {V_\lambda} \ot {W_\lambda}
	\]
	is free if and only if
	\[
	{V_{\lambda_1}} \ot {W_{\lambda_1}} \neq
	{W_{\lambda_2}} \ot {V_{\lambda_2}}
	\]
	for all $\lambda_1, \lambda_2 \in \Lambda(i,n)$ whenever $i < n$.
\end{lemma}

\begin{proof}
	If this is not the case, then there is a summand of the form
\end{proof}

\begin{lemma} \label{l:(1+T) triangle = 0 implies triangle = 0}
    Let $\triangle$ be a free cup-$i$ construction.
    If
    \[
    (1 + T) \triangle_i [n] =
    (1 + T) \sum_{\lambda \in \Lambda(i,n)} {V_\lambda} \ot {W_\lambda}
    \]
    for some $i < n$, then there exists a partition of $\Lambda(i,n) = \Lambda_1 \sqcup \Lambda_2$ such that
    \begin{equation} \label{e:splitting after 1+T}
    \triangle_i [n] =
    \sum_{\lambda \in \Lambda_1} {V_\lambda} \ot {W_\lambda} +
    \sum_{\lambda \in \Lambda_2} {W_\lambda} \ot {V_\lambda}.
    \end{equation}
\end{lemma}

\begin{proof}
	We directly have that \eqref{e:splitting after 1+T} holds up to an element in the kernel of $(1+T)$.
	This kernel is generated by elements of the form $U \ot U$ and $V \ot W + W \ot V$.
	Therefore, Lemma \ref{l:freeness recasted} implies the element in the kernel must be $0$.
\end{proof}