% !TEX root = ../axiomatic.tex

\section{Cup-$i$ constructions}\label{s:statement}

Let $\Ftwo$ be the field with two elements.
We denote the (normalized) cochains with $\Ftwo$ coefficients of a simplicial or semi-simplicial set $X$ by $\cochains(X)$, graded homologically and concentrated in non-positive degrees.

\subsection{Cup-$i$ products}

Let $\Sym_2$ be the group with one non-identity element $T$.
Let
\[
\begin{tikzcd}[column sep=20pt]
W = \Big(
\Ftwo[\Sym_2]\{e_0\} &
\Ftwo[\Sym_2]\{e_1\} \arrow[l, "\ 1+T"'] &
\arrow[l, "\ 1+T"'] \cdots \Big)
\end{tikzcd}
\]
be the minimal free resolution of $\Ftwo$ by $\Ftwo[\Sym_2]$-modules.
We notice that the group of automorphisms of $W$ is isomorphic to $\prod_{\N} \Sym_2$ with any such $\phi$ determined by a choice, for every $i \in \N$, between $\phi(e_i) = e_i$ or $\phi(e_i) = Te_i$.

A \textbf{\mbox{cup-$i$} product structure} on a chain complex $A$ is a chain map
\[
W \ot_{\F[\Sym_2]} A^{\ot 2} \to A
\]
where $T$ acts by transposition on $A^{\ot 2}$ and by right multiplication on $W$.
We denote the image of of the class of $e_i \ot \alpha \ot \beta$ by $\alpha \cup_i \beta$.
Unpacking this structure we have the following defining identity for any $i \in \N$ and $\alpha, \beta \in A$:
\[
\alpha \cup_{i-1} \beta + \beta \cup_{i-1} \alpha =
\bd (\alpha \cup_{i} \beta) + (\bd\alpha) \cup_{i} \beta + \alpha \cup_{i} (\bd\beta)
\]
with the convention $\alpha \cup_{-1} \beta = 0$.
By setting $i = 0$ we see that $(A, \cup_0)$ is a differential graded algebra.
The identity obtained setting $i = 1$ implies that the algebra structure induced in the homology $HA$ of $A$ is commutative, with the product $\cup_1$ enforcing this relation at the chain level.
This product is itself not commutative, but it is so up to a homotopy defined by $\cup_2$, and so on.

An \textbf{isomorphism} of \mbox{cup-$i$} product structures on $A$, or simply a \textbf{cup-$i$ isomorphism}, is an automorphism $\phi$ of $W$ making the following diagram commute:
\begin{center}
	\begin{tikzcd}[column sep=5, row sep=15]
		W \displaytensor_{\F[\Sym_2]} A \arrow[dr, in=180, out=-90] \arrow[rr, "\phi \, \ot \, \id \, "] & &
		W \displaytensor_{\F[\Sym_2]} A. \arrow[dl, in=0, out=-90] \\
		& A &
	\end{tikzcd}
\end{center}

A \textbf{(semi-)simplicial \mbox{cup-$i$} construction} is a \mbox{cup-$i$} product structure on $\cochains(X)$ for every (semi-)simplicial set $X$ that is natural with respect to (semi-)simplicial maps.
Similarly, an \textbf{isomorphism} of (semi-)simplicial \mbox{cup-$i$} constructions is a natural family of cup-$i$ isomorphisms for each such $X$.
Please notice that a simplicial \mbox{cup-$i$} construction is a semi-simplicial one satisfying additional constraints, i.e., compatibility with degeneracy maps.

\subsection{Axioms}

The first axiom alluded to in the introduction, naturality, has been explicitly included in our definition of (semi-)simplicial \mbox{cup-$i$} construction; whereas the second, minimality, is manifested in the use of $W$ instead of an arbitrary free resolution of $\F$.
The \textbf{zero \mbox{cup-$i$} construction} assigns the zero map
\[
W \ot_{\F[\Sym_2]} \cochains(X)^{\ot 2} \to \cochains(X)
\]
to every (semi-)simplicial set $X$.
Any other construction is said to be \textbf{non-zero}.

%We discard the zero simplicial \mbox{cup-$i$} construction explicitly via the third axiom.
%We now introduce the last two axioms.

%A simplicial \mbox{cup-$i$} construction is \textbf{non-zero} if
%\[
%\boxed{x \cup_0 x \neq 0}
%\]
%for a $0$-simplex $x$.

Let $X$ be a semi-simplicial set.
Below we identify simplices with their associated chain and cochain basis elements.
A semi-simplicial \mbox{cup-$i$} construction is said to be:
\begin{itemize}
	\item \textbf{Irreducible} if for any simplex $x$ in $X$
	\[
	\boxed{\Big( y^{(1)} \cup_{i} y^{(2)} \Big)(x) = 0}
	\]
	whenever $y$ is a proper face of $x$ and $y^{(1)}$ and $y^{(2)}$ are faces of $y$.

	\item \textbf{Free} if for any two simplices $x$ and $x'$ in $X$
	\[
	\boxed{x \cup_{i} x' = x' \cup_{i} x} \
	\Longrightarrow \
	\boxed{x \cup_{i} x' = 0}
	\]
	whenever $\bars{x} \neq i$ or $\bars{x'} \neq i$.
\end{itemize}

\subsection{Main Theorem}\label{ss:main_theorem}

\textit{There is up to isomorphism only one non-zero, irreducible and free semi-simplicial \mbox{cup-$i$} construction.}

In \cref{s:others,s:operads} we will show that all semi-simplicial \mbox{cup-$i$} constructions defined in the literature represent the same isomorphism class, and in fact, they are all simplicial.

\subsection{Canonical cup-\textit{i} construction}\label{ss:canonical}

To prove our main result we will use the \mbox{cup-$i$} construction introduced in \cite{medina2023fast_sq}.
Before we recall its definition, let us introduce the following notation:
For any $n$-simplex $x$ and set $U = \{u_1 < \dots < u_r\} \subseteq \set[big]{0,\dots,n}$ we write $d_U(x)$ for the composition of face maps $d_{u_1}\!\! \dotsm d_{u_r}(x)$ with $d_{\emptyset}(x) = x$.

Let $X$ be a (semi-)simplicial set, $x$ a (non-degenerate) $n$-simplex of $X$, and $\alpha, \beta$ two cochains in $\cochains(X)$, then, by definition,
\[
(\alpha \cup_i \beta)(x) =
\begin{cases}
	(\alpha \ot \beta) \sum d_{U^0}(x) \ot d_{U^1}(x) &
	\text{if } i \in \{0, \dots, n\}, \\
	\hfil 0 &
	\text{otherwise},
\end{cases}
\]
where the sum is taken over all $U = \{u_1 < \cdots < u_{n-i}\} \subseteq \{0, \dots, n\}$ and
\[
U^0 = \{u_j \in U \mid u_j + j \equiv 0 \text{ mod } 2\}, \qquad
U^1 = \{u_j \in U \mid u_j + j \equiv 1 \text{ mod } 2\}.
\]
As an example we mention that, if $i = 0$, the above formula gives
\[
(\alpha \cup_0 \beta)(x) =
\sum_{j=0}^n \alpha \big(d_{j+1} \cdots d_{n}(x)\big) \cdot \beta \big(d_{0} \cdots d_{j-1}(x)\big),
\]
the well-known Alexander--Whitney product.

The verification that these formulas define a semi-simplicial \mbox{cup-$i$} construction is presented in \cite{medina2023fast_sq}.
We will refer to it as the \textbf{canonical \mbox{cup-$i$} construction} and show it agrees with Steenrod's original construction in \cref{t:steenrod cup-i}.