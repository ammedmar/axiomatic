% !TEX root = ../axiomatic.tex

\section{Cup-\textit{i} constructions}\label{s:statement}

%In this section, we introduce the concept of a simplicial \mbox{cup-$i$} construction and the axioms that characterize Steenrod's.

We denote the (normalized) cochains of a simplicial set $X$ with $\Ftwo$ coefficients by $\cochains(X)$, graded homologically and concentrated in non-positive degrees.

\subsection{Cup-\textit{i} products}

Let $\Sym_2$ be the group with one non-identity element $T$ and
\[
\begin{tikzcd}[column sep=20pt]
W = \Big(
\Ftwo[\Sym_2]\{e_0\} &
\Ftwo[\Sym_2]\{e_1\} \arrow[l, "\ 1+T"'] &
\arrow[l, "\ 1+T"'] \cdots \Big)
\end{tikzcd}
\]
the minimal free resolution of $\Ftwo$ by $\Ftwo[\Sym_2]$-modules.
We notice that the group of automorphisms of $W$ is isomorphic to $\prod_{\N} \Sym_2$ with any such $\phi$ defined by a choice between $\phi(e_i) = e_i$ or $\phi(e_i) = Te_i$ for every $i \in \N$.

A \textbf{\mbox{cup-$i$} product structure} on a chain complex $A$ is a chain map
\[
W \ot_{\F[\Sym_2]} A^{\ot 2} \to A
\]
where $T$ acts by transposition on $A^{\ot 2}$ and by right multiplication on $W$.
We denote the image of $[e_i \ot \alpha \ot \beta]$ by $\alpha \cup_i \beta$.

An \textbf{isomorphism} of \mbox{cup-$i$} product structures on $A$ is an automorphism $\phi$ of $W$ making the following diagram commute:
\begin{center}
	\begin{tikzcd}[column sep=5, row sep=15]
			W \displaytensor_{\F[\Sym_2]} A \arrow[dr, in=180, out=-90] \arrow[rr, "\phi \, \ot \, \id \, "] & &
			W \displaytensor_{\F[\Sym_2]} A \arrow[dl, in=0, out=-90] \\
			& A & \qquad.
	\end{tikzcd}
\end{center}

Unpacking this structure on $A$ we have the following identity for any $i \in \Z$ and $\alpha, \beta \in A$:
\[
\alpha \cup_{i-1} \beta + \beta \cup_{i-1} \alpha =
\bd (\alpha \cup_{i} \beta) + (\bd\alpha) \cup_{i} \beta + \alpha \cup_{i} (\bd\beta)
\]
with the convention $\alpha \cup_{i} \beta = 0$ for a negative integer $i$.
Setting $i = 0$ we see that $(A, \cup_0)$ is a differential graded algebra.
The identity obtained setting $i = 1$ implies that the algebra structure induced in the homology $HA$ of $A$ is commutative, with the product $\cup_1$ enforcing this relation at the chain level.
This product is itself not commutative, but it is so up to a homotopy defined by $\cup_2$, and so on.

A \textbf{semi-simplicial set}, sometimes referred to as a delta set, is a simplicial set with no degeneracy maps.
A \textbf{(semi-)simplicial \mbox{cup-$i$} construction} is a \mbox{cup-$i$} product structure on $\cochains(X)$ for every (semi-)simplicial set $X$ that is natural with respect to (semi-)simplicial maps.
An \textbf{isomorphism} of (semi-)simplicial \mbox{cup-$i$} constructions is a natural family of isomorphisms of cup-$i$ product structures.
Any simplicial cup-$i$ construction defines a semi-simplicial cup-$i$ construction, but not in the other direction.

\subsection{Axioms}

The first axiom alluded to in the introduction, naturality, has been explicitly included in our definition of (semi-)simplicial \mbox{cup-$i$} construction; whereas the second, minimality, is manifested in the use of $W$ instead of an arbitrary free resolution of $\F$.
The \textbf{zero cup-$i$ construction} assigns the zero map
\[
W \ot_{\F[\Sym_2]} \cochains(X)^{\ot 2} \to \cochains(X)
\]
to every (semi-)simplicial set $X$, and any other construction is said to be \textbf{non-zero}.

%We discard the zero simplicial cup-$i$ construction explicitly via the third axiom.
%We now introduce the last two axioms.

%A simplicial \mbox{cup-$i$} construction is \textbf{non-zero} if
%\[
%\boxed{x \cup_0 x \neq 0}
%\]
%for a $0$-simplex $x$.

Let $X$ be an arbitrary simplicial set.
Below we identify a non-degenerate simplex in $X$ with its associated chain and cochain basis elements.
A simplicial cup-$i$ construction is said to be \textbf{irreducible} if for any proper face $y$ of $x$, a non-degenerate simplex in $X$, we have
\[
\boxed{\Big( y^{(1)} \cup_{i} y^{(2)} \Big)(x) = 0}
\]
for any two faces $y^{(1)}$ and $y^{(2)}$ of $y$.
We say it is \textbf{free} if for any two simplices $x$ and $x'$ in $X$
\[
\boxed{x \cup_{i} x' = x' \cup_{i} x} \
\Longrightarrow \
\boxed{x \cup_{i} x' = 0}
\]
whenever $\bars{x} \neq i$ or $\bars{x'} \neq i$.

\subsection{Main result}

We will use the following, the main theorem of this paper, to prove in \cref{s:others,s:operads} that all simplicial \mbox{cup-$i$} constructions in the literature are isomorphic to Steenrod's original \cite{steenrod1947products}.

\begin{theorem}
	There is up to isomorphism only one non-zero, irreducible and free semi-simplicial \mbox{cup-$i$} construction.
\end{theorem}

\subsection{Canonical construction}

To prove our main result we will use the simplicial \mbox{cup-$i$} construction introduced in \cite{medina2023fast_sq}.
Before we recall its definition, let us introduce the following notation:
For any $n$-simplex $x$ and set $U = \{u_1 < \dots < u_r\} \subseteq \set[big]{0,\dots,n}$ we write $d_U(x)$ for the composition of face maps $d_{u_1}\! \dotsm d_{u_r}(x)$ with $d_{\emptyset}(x) = x$.

\begin{definition}[\cite{medina2023fast_sq}]\label{d:my cup-i}
	Let $X$ be a simplicial set, $x$ a non-degenerate $n$-simplex of $X$, and $\alpha, \beta$ two cochains in $\cochains(X)$.
	\[
	(\alpha \cup_i \beta)(x) =
	\begin{cases}
		(\alpha \ot \beta) \sum d_{U^0}(x) \ot d_{U^1}(x) &
		\text{if } i \in \{0, \dots, n\}, \\
		\hfil 0 &
		\text{otherwise},
	\end{cases}
	\]
	where the sum is taken over all $U = \{u_1 < \cdots < u_{n-i}\} \subseteq \{0, \dots, n\}$ and
	\[
	U^0 = \{u_j \in U \mid u_j + j \equiv 0 \text{ mod } 2\}, \qquad
	U^1 = \{u_j \in U \mid u_j + j \equiv 1 \text{ mod } 2\}.
	\]
\end{definition}

\begin{example}
	For $i = 0$, the above gives
	\[
	(\alpha \cup_0 \beta)(x) =
	\sum_{j=0}^n \alpha \big(d_{j+1} \cdots d_{n}(x)\big) \cdot \beta \big(d_{0} \cdots d_{j-1}(x)\big),
	\]
	the well-known \textbf{Alexander--Whitney product}.
\end{example}

The verification that these formulas define a semi-simplicial \mbox{cup-$i$} construction is presented in \cite{medina2023fast_sq}.
We will refer to it as the \textbf{canonical \mbox{cup-$i$} construction} and show it agrees with Steenrod's original construction in \cref{t:steenrod cup-i}.