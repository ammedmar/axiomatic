% !TEX root = ../axiomatic.tex

\section{Main statement} \label{s:statement}

In this section we introduce the notion of \mbox{cup-$i$} construction and the properties that characterize Steenrod's.
We state our main result and review a specific presentation of a \mbox{cup-$i$} construction satisfying these properties.

\subsection{Cup-$i$ constructions}

Let $\Sym_2$ be the group with one non-identity element $T$ and let
\[
\begin{tikzcd}[column sep=20pt]
W = \Big(
\Fp[\Sym_2]\{e_0\} &
\Fp[\Sym_2]\{e_1\} \arrow[l, "\ 1+T"'] &
\arrow[l, "\ 1+T"'] \cdots \Big)
\end{tikzcd}
\]
be the minimal projective resolution of $\F$ by $\F[\Sym_2]$-modules.
We notice that the group of automorphisms of $W$ is isomorphic to $\prod_{\N} \Sym_2$ with any such $\phi$ defined by a choice for every $i \in \N$ of
\[
\phi(e_i) = e_i
\quad \text{ or } \quad
\phi(e_i) = Te_i.
\]

\begin{definition}
	A \textbf{\mbox{cup-$i$} product structure} on a chain complex $A$ is a chain map
	\[
	W \ot_{\F[\Sym_2]} A^{\ot 2} \to A
	\]
	where $T$ acts by transposition on $A^{\ot 2}$ and by right multiplication on $W$.
	We denote the image of $[e_i \ot \alpha \ot \beta]$ by $\alpha \cup_i \beta$.
\end{definition}

\begin{remark}
	Consider a \mbox{cup-$i$} product structure on $A$.
	We can think of the product $\cup_0 \colon A \otimes A \to A$ as being commutative up to coherent homotopies, which are given by the maps $\cup_i \colon A \ot A \to A$ for $i > 0$.
	Notice that the induced map
	\[
	\Ftwo \otimes_{\Ftwo[\Sym_2]} H^{\ot 2} \to H
	\]
	on the homology $H$ of $A$ defines a \emph{commutative} product on it.
\end{remark}

\begin{definition}
	An \textbf{isomorphism} of \mbox{cup-$i$} product structures on $A$ is an automorphism $\phi$ of $W$ making the diagram
	\begin{center}
	\begin{tikzcd}[column sep=5, row sep=15]
	W \displaytensor_{\F[\Sym_2]} A \arrow[dr, in=180, out=-90] \arrow[rr, "\phi \, \ot \, \id \, "] & &
	W \displaytensor_{\F[\Sym_2]} A \arrow[dl, in=0, out=-90] \\
	& A &
 	\end{tikzcd}
	\end{center}
	commute.
\end{definition}

\begin{definition}
	A \textbf{\mbox{cup-$i$} construction} (more accurately termed \emph{\mbox{cup-$i$} product construction for simplicial sets}) is a \mbox{cup-$i$} product structure on $\cochains(X)$ for every simplicial set $X$ that is natural with respect to simplicial maps.
	An \textbf{isomorphism} of \mbox{cup-$i$} constructions is an isomorphism of \mbox{cup-$i$} structures on $\cochains(X)$ for every simplicial set $X$ that is natural with respect to simplicial maps.
\end{definition}

\subsection{Uniqueness}

The first axiom alluded to in the introduction, naturality, has been explicitly included in our definition of \mbox{cup-$i$} construction; whereas the second, minimality, is manifested in the use of $W$ instead of an arbitrary projective resolution of $\F$.

\begin{definition} \label{d:properties}
	A \mbox{cup-$i$} construction is \textbf{non-degenerate} if for any simplex $x$
	\[
	\boxed{x \cup_0 x \neq 0}
	\]
	whenever $\bars{x} = 0$.
	It is \textbf{irreducible} if for any proper face $y$ of $x$
	\[
	\boxed{\Big( y^{(1)} \cup_{i} y^{(2)} \Big)(x) = 0}
	\]
	for any two faces $y^{(1)}$ and $y^{(2)}$ of $y$.
	It is \textbf{free} if for any two simplices $x$ and $y$
	\[
	\boxed{x \cup_{i} y = y \cup_{i} x} \
	\Longrightarrow \
	\boxed{x \cup_{i} y = 0}
	\]
	whenever $\bars{x} \neq i$ or $\bars{y} \neq i$.
\end{definition}

We can now state our main result.

\begin{theorem} \label{t:main}
	There is up to isomorphism only one non-degenerate, irreducible and free \mbox{cup-$i$} construction.
\end{theorem}

We will use this result to prove in \cref{s:others} that all \mbox{cup-$i$} constructions available in the literature are isomorphic to Steenrod's original \cite{steenrod1947products}.

\subsection{Existence}

To prove our main result we will use formulas defining one such \mbox{cup-$i$} construction introduced in \cite{medina2021fast_sq}.

\begin{notation*}
	For any $n$-simplex $x$ and set $U = \{u_1 < \dots < u_r\} \subseteq \big\{ 0, \dots, n \big\}$ we write $d_U(x)$ for $d_{u_1}\! \dotsm d_{u_r}(x)$, with $d_{\emptyset}(x) = x$.
\end{notation*}

\begin{definition}[\cite{medina2021fast_sq}] \label{d:my cup-i}
	Let $X$ be a simplicial set, $x \in X_n$ and $\alpha, \beta \in \cochains(X)$.
	\begin{equation} \label{e:new formulas}
	(\alpha \cup_i \beta)(x) =
	\begin{cases}
	(\alpha \ot \beta) \sum d_{U^0}(x) \ot d_{U^1}(x) &
	\text{if } i \in \{0, \dots, n\}, \\
	\hfil 0 &
	\text{otherwise},
	\end{cases}
	\end{equation}
	where the sum is taken over all $U = \{u_1 < \cdots < u_{n-i}\} \subseteq \{0, \dots, n\}$ and
	\begin{equation*}
	U^0 = \{u_j \in U \mid u_j \equiv j \text{ mod } 2\}, \qquad
	U^1 = \{u_j \in U \mid u_j \not\equiv j \text{ mod } 2\}.
	\end{equation*}
\end{definition}

\begin{example}
	For $i = 0$, \cref{e:new formulas} gives
	\begin{equation*}
	(\alpha \cup_0 \beta)(x) =
	\sum_{j=0}^n \alpha \big(d_{j+1} \cdots d_{n}(x)\big) \cdot \beta \big(d_{0} \cdots d_{j-1}(x)\big),
	\end{equation*}
	the so-called \textbf{Alexander--Whitney product}.
\end{example}

The verification that our formulas in \cref{e:new formulas} define a \mbox{cup-$i$} construction is given in \cite{medina2021fast_sq}.
We will refer to it as the \textbf{canonical \mbox{cup-$i$} construction}.
Furthermore, it is non-degenerate, irreducible, and free (\cref{t:existence}) and agrees with Steenrod's original cup-$i$ construction (\cref{t:steenrod cup-i}).