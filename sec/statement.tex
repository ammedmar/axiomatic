% !TEX root = ../axiomatic.tex

\section{Main statement} \label{s:statement}

Let $\Sym_2$ be the group with one non-identity element $T$ and let
\[
\begin{tikzcd}[column sep = 20pt]
W = \Big(
\F[\Sym_2]\{e_0\} &
\F[\Sym_2]\{e_1\} \arrow[l, "\ 1+T"'] &
\arrow[l, "\ 1+T"'] \cdots \Big)
\end{tikzcd}
\]
be the minimal resolution of $\F$ by free $\F[\Sym_2]$-modules.

\begin{definition}
	A \textbf{cup-$i$ product structure} on a chain complex $A$ is a chain map
	\[
	W \ot_{\F[\Sym_2]} A^{\ot 2} \to A
	\]
	where $T$ acts by transposition on $A^{\ot 2}$.
	We denote the image of $[e_i \ot \alpha \ot \beta]$ by $\alpha \cup_i \beta$.
\end{definition}

\begin{remark}
	Since a chain map $\F \ot_{\F[\Sym_2]} A^{\ot 2} \to A$ defines a commutative product on $A$, we can think of a cup-$i$ product structure on $A$ as defining a product on $A$ that satisfies a minimally derived commutativity condition.
\end{remark}

\begin{definition}
	An \textbf{isomorphism} of cup-$i$ product structures on $A$ is an automorphism $\phi$ of $W$ making the diagram
	\begin{center}
	\begin{tikzcd}[column sep = 5]
	W \displaytensor_{\F[\Sym_2]} A \arrow[dr, in=180, out=-90] \arrow[rr, "\phi \, \ot \, \id"] & &
	W \displaytensor_{\F[\Sym_2]} A \arrow[dl, in=0, out=-90] \\
	& A &
	\end{tikzcd}
	\end{center}
	commute.
\end{definition}

\begin{definition}
	A \textbf{cup-$i$ construction} (more accurately termed cup-$i$ product construction on simplicial sets) is a cup-$i$ product structure on $\cochains(X)$ for every simplicial set $X$ that is natural with respect to simplicial maps.
	An \textbf{isomorphism} of cup-$i$ constructions is a collection of natural isomorphisms of cup-$i$ product structures.
\end{definition}

The first axiom alluded to in the introduction, naturality, has been explicitly included in our definition of cup-$i$ construction; whereas the second, minimality, is manifested in the definition of cup-$i$ product structure via the use of $W$ instead of an arbitrary projective resolution of $\F$.

\begin{definition}
	A cup-$i$ construction is \textbf{non-degenerate} if for any simplex $x$
	\[
	\boxed{x \cup_0 x \neq 0}
	\]
	whenever $\bars{x} = 0$.
	It is \textbf{free} if for any two simplices $x$ and $y$
	\[
	\boxed{x \cup_{i} y = y \cup_{i} x} \
	\Longrightarrow \
	\boxed{x \cup_{i} y = 0}
	\]
	whenever $\bars{x} \neq i$ or $\bars{y} \neq i$.
	It is \textbf{irreducible} if for any proper face $y$ of $x$
	\[
	\boxed{\Big( y^{(1)} \cup_{i} y^{(2)} \Big)(x) = 0}
	\]
	for any two faces $y^{(1)}$ and $y^{(2)}$ of $y$.
\end{definition}

We can now state our main result.

\begin{theorem} \label{t:main}
	There is up to isomorphism only one free non-degenerate irreducible cup-$i$ construction.
\end{theorem}

Furthermore, we will use this result to prove in \cref{s:other presentations} that all cup-$i$ constructions available in the literature are isomorphic to Steenrod's original \cite{steenrod1947products}.

To prove our main result we will use formulas defining one such cup-$i$ construction introduced in \cite{medina2021newformulas} and reviewed below.
The verification that these define a cup-$i$ construction is given in that same article.
In what follows we refer to it as the \textbf{canonical cup-$i$ construction}.

Recall from \cref{s:preliminaries} that for any $n$-simplex $x$ and set
\begin{equation*}
U = \{u_1 < \dots < u_r\} \subseteq \big\{ 0, \dots, n \big\}
\end{equation*}
we write $d_U(x) = d_{u_1}\! \dotsm \, d_{u_r}(x)$, with $d_{\emptyset}(x) = x$.

\begin{definition}[\cite{medina2021newformulas}] \label{d:my cup-i construction}
	Let $X$ be a simplicial set, $x \in X_n$ and $\alpha, \beta \in \cochains(X)$.
	\begin{equation} \label{e:new formulas}
	(\alpha \cup_i \beta)(x) =
	\begin{cases}
	(\alpha \ot \beta) \sum d_{U^0}(x) \ot d_{U^1}(x) &
	\text{if } i \in \{0, \dots, n\}, \\
	\hfil 0 &
	\text{otherwise},
	\end{cases}
	\end{equation}
	where the sum is taken over all subsets $U = \{u_1 < \cdots < u_{n-i}\} \subseteq \{0, \dots, n\}$ and
	\begin{equation} \label{e:partition subsets}
	U^0 = \{u_j \in U \mid u_j \equiv j \text{ mod } 2\}, \qquad
	U^1 = \{u_j \in U \mid u_j \not\equiv j \text{ mod } 2\}.
	\end{equation}
\end{definition}

\begin{example} \label{ex:alexander-whitney diagonal}
	For $i = 0$, \cref{e:new formulas} gives
	\begin{equation*}
	(\alpha \cup_0 \beta)(x) =
	\sum_{j=0}^n \alpha \big(d_{j+1} \cdots d_{n}(x)\big) \cdot \beta \big(d_{0} \cdots d_{j-1}(x)\big).
	\end{equation*}
	A binary operation known as the \textbf{Alexander--Whitney product}.
\end{example}

\begin{example} \label{ex:Sq0 is the identity}
	For any simplex $x \in X_n$, \cref{e:new formulas} gives
	\begin{equation*}
	(x \cup_n x)(x) = x(x) \cdot x(x) = 1.
	\end{equation*}
\end{example}

In particular, we see from this example that the canonical cup-$i$ construction is non-degenerate.
\anibal{check both free and irreducible}

%To do so we will provide an alternative characterization of irreducibility for free non-degenerate cup-$i$ structures in \cref{labellist} that is easier to check in specific examples.