% !TEX root = ../axiomatic.tex

\section{Statement}

Let $\Sym_2$ be the group with one non-identity element $T$ and let
\[
\begin{tikzcd}[column sep = 20pt]
W = \big(
\F[\Sym_2]\{e_0\} &
\F[\Sym_2]\{e_1\} \arrow[l, "\ 1+T"'] &
\arrow[l, "\ 1+T"'] \cdots \big)
\end{tikzcd}
\]
be the minimal resolution of $\F$ by free $\F[\Sym_2]$-modules.
A \textbf{symmetric product} on a chain complex $A$ is a chain map
\[
W \ot_{\F[\Sym_2]} A^{\ot 2} \to A
\]
where $T$ acts by multiplication on $W$ and by transposition on $A^{\otimes 2}$.
We denote the image of $[e_i \ot \alpha \ot \beta]$ by $\alpha \cup_i \beta$.

A \textbf{cup-$i$ construction} is a symmetric product on $\cochains(X)$ for every simplicial set $X$ that is natural with respect to simplicial maps.
An \textbf{isomorphism} of cup-$i$ constructions is an automorphism $\phi$ of $W$ making the diagram
\begin{center}
	\begin{tikzcd}[column sep = -5]
	W \displaytensor_{\F[\Sym_2]} \cochains(X) \arrow[dr, in=180, out=-90] \arrow[rr, "\phi \, \ot \, \id"] & &
	W \displaytensor_{\F[\Sym_2]} \cochains(X) \arrow[dl, in=0, out=-90] \\
	& \cochains(X) &
	\end{tikzcd}
\end{center}
commute for every simplicial set $X$.

The first axiom alluded to in the introduction, naturality, has been explicitly included in our definition of cup-$i$ construction; whereas the second, minimality, is manifested in the definition of cup-$i$ structure via the use of $W$.
A cup-$i$ construction is \textbf{non-degenerate} if it is not the $0$ map, and it is \textbf{free} if
\[
\boxed{x \cup_{i} y = y \cup_{i} x} \ \Longrightarrow\
\boxed{x \cup_{i} y = 0}
\]
whenever $\bars{x} \neq i$ or $\bars{y} \neq i$.
A cup-$i$ construction is \textbf{irreducible} if for any other another
\[
\boxed{x \cup_{i} y = x \cup_{i}^\prime y + \cbd(x \cup_{i+1}^\prime y)}
\ \Longrightarrow \
\boxed{x \cup_i y = 0}
\]
whenever $\bars{x} + \bars{y} \neq i, i-1$.

\begin{theorem} \label{theorem: main}
	There is, up to isomorphism, only one free non-degenerate irreducible cup-$i$ construction.
\end{theorem}

\begin{theorem}
	Steenrod's original cup-$i$ construction \cite{steenrod1947products} is free, non-degenerate and irreducible.
\end{theorem}

%\begin{remark}
%	Steenrod squares were axiomatized soon after their introduction with the Cartan formula being the least obvious of the axioms.
%	We think of Theorem \ref{theorem: main} as a continuation of this efforts and remark that in \cite{medina2020cartan} an effective cochain level proof of the Cartan formula is given.
%	ADEM
%\end{remark}
%
%\begin{remark}
%	The formulae in Definition \ref{definition: our cup-i products} have been used to provide new algorithms for the computation of Steenrod squares and cup-$i$ products on finite simplicial complexes.
%	See \cite{medina2018persistence} for a discussion of these algorithms and their incorporation into the field of topological data analysis.
%\end{remark}

%\subsection{conterexample}
%
%Every pair $\triangle_{i,n} = \Delta_{i,n}$ except for a fixed $i^\prime$ and $n^\prime$ with
%\[
%\triangle_{i^\prime+1, n} =
%T \Delta_{i^\prime, n}, \quad n \geq n^\prime,
%\]
%and
%\[
%\triangle_{i^\prime, n} = \begin{cases}
%T \Delta_{i^\prime, n} & n < n^\prime, \\
%\Delta_{i^\prime, n} + \Delta_{i+1, n-1} \circ \bd_n & n = n^\prime,
%\end{cases}
%\]
%
%For example, to $\Delta_0([0123])$ this adds
%\[
%\begin{split}
%[1,2,3] \otimes [1,2] + [1,3] \otimes [1,2,3] + [1,2,3] \otimes [2,3] + \\ [0,2,3] \otimes [0,2] + [0,3] \otimes [0,2,3] + [0,2,3] \otimes [2,3] + \\ [0,1,3] \otimes [0,1] + [0,3] \otimes [0,1,3] + [0,1,3] \otimes [1,3] + \\ [0,1,2] \otimes [0,1] + [0,2] \otimes [0,1,2] + [0,1,2] \otimes [1,2] \phantom{+}
%\end{split}
%\]


%\subsection{formulas}
%
%We now give a new description of Steenrod's original cup-$i$ construction.
%Our description is in a sense dual to his and the equivalence of both is stated as Proposition \ref{proposition: steenrod's equals ours}.
%For any positive integer $q$ let $\P_q$ be the collection of cardinality $q$ subsets of non-negative integers and
%\[
%\P_q(n) = \{ U \in \P_q\ |\ \forall u \in U, u \leq n\}.
%\]
%For $U = \{u_1 < \cdots < u_q\} \in \P_q$ let
%\[
%d_U \colon \chains(X) \to \chains(X)
%\]
%be the linear map defined on a basis element $x \in X_n$ by
%\[
%d_U (x) = d_{u_1} \cdots \, d_{u_q} (x)
%\]
%with the convention that $d_U(x) = 0$ if $n < u_q$.
%For each $u_r \in U$ define the \textbf{index of $u_r$ in $U$} as
%\[
%\ind_U(u_r) = u_r + r
%\]
%denoting $U^-$ (resp. $U^+$) the subset of $U$ containing all elements whose index in $U$ is odd (resp. even).
%Either of these sets could be empty and we declare $d_\emptyset = \id$ and $\P_0 = \{\emptyset\}$.
%
%\begin{definition} \label{definition: our cup-i products}
%	For any simplicial set $X$ and cochains $\alpha, \beta \in \cochains(X)$ define for any $c \in \chains(X)_n$
%	\[
%	(\alpha \cup_i \beta)(c) =
%	(\alpha \ot \beta) \sum_{U \in \P_{n-i}} d_{U^-}(c) \ot d_{U^+}(c)
%	\]
%	if  $i \leq n$ and to be $0$ otherwise.
%\end{definition}
