%\anibal{Make a more general statement of the following that includes other type of summands. It must also apply to the second lemma in the induction step.}
%
%A direct consequence of Lemma \cref{l:splitting of summands} is the following.
%
%\begin{lemma}
%	If for some integers $i < n$,
%	\begin{align*}
%	(1+T) \, \triangle_i [n] = \,
%	&(1+T) \, \Delta_i [n] \\ \defeq
%	&(1+T) \sum_{\P_{n-i}^n} {U^0} \ot {U^1}
%	\end{align*}
%	then there exists $\xi \colon \P_{n-i}(n) \to \F$ such that
%	\[
%	\triangle_i [n]\ =\! \sum_{\P_{n-i}^n} {U^{\xi}} \ot {U^{\bar{\xi}}}.
%	\]
%\end{lemma}

% THIS SEEMS TO BE WRONG, NOT ANY CUP-i CONST.
%\begin{lemma} \label{l:triangle boundary}
%	Let $\triangle$ be a free cup-$i$ construction.
%	If for some integers $i \leq n$ we have
%	\begin{align*}
%	(1+T) \, \triangle_i [n-1] = \,
%	&(1+T) \, \Delta_i [n-1] \\ \defeq
%	&(1+T) \sum_{\P_{n-i-1}^n} {U^0} \ot {U^1}
%	\end{align*}
%	then
%	\begin{align}
%	\label{e:triangle of boundary}
%	\triangle_i \bd \, [n] \ =\!
%	\sum_{\P_{n-i}^n} \left( \,
%	\sum_{u \in U^1} {u.U^0} \ot {U^1} \ +
%	\sum_{u \in U^0} {U^0} \ot {u.U^1} \right).
%	\end{align}
%\end{lemma}


%\anibal{absorve the following lemma into the proof of the next one, the only place where it's used.}
%
%\begin{lemma} \label{l:boundary triangle}
%	Let $\triangle$ be a free non-degenerate cup-$i$ construction.
%	If for some $i < n$ there is $\xi \colon \P_{n-i}^n \to \F$ such that
%	\[
%	\triangle_i [n]\ =\! \sum_{U \in \P_{n-i}^n \kern-5pt} {U^{\xi}} \ot {U^{\barxi}}
%	\]
%	then
%	\begin{align*}
%	\label{equation: boundary of triangle}
%	\bd \triangle_i [n]\ = &
%	\sum_{U \in \P_{n-i}^n \kern-5pt} \left( \, \sum_{u \in U^\barxi} u.U^\xi \ot U^\barxi \ +
%	\sum_{u \in U^\xi} U^\xi \ot u.U^\barxi \right) \\ + &
%	\sum_{U \in \P_{n-i}^n \kern-5pt} \sum_{x \notin U} \left( x.U^\xi \ot U^\barxi \ +\ U^\xi \ot {x.U^\barxi} \right).
%	\end{align*}
%\end{lemma}
%
%\anibal{where is the proof? It is immediate from the definition of $\bd$ in $\cP$}

%We say that a cup-$i$ construction if \textbf{reducible} if it is not irreducible and we have the following characterization.
%
%\begin{theorem}
%	A free non-degenerate cup-$i$ construction $\phi$ is reducible if and only there is $\phi(e_i)$
%
%	of reducible non-degenerate and free cup-$i$ constructions that
%
%\end{theorem}


%\anibal{Say something about the proof}

%\begin{remark}
%	Steenrod squares were axiomatized soon after their introduction with the Cartan formula being the least obvious of the axioms.
%	We think of Theorem \ref{theorem: main} as a continuation of this efforts and remark that in \cite{medina2020cartan} an effective cochain level proof of the Cartan formula is given.
%	ADEM
%\end{remark}
%
%\begin{remark}
%	The formulae in Definition \ref{definition: our cup-i products} have been used to provide new algorithms for the computation of Steenrod squares and cup-$i$ products on finite simplicial complexes.
%	See \cite{medina2018persistence} for a discussion of these algorithms and their incorporation into the field of topological data analysis.
%\end{remark}

%\subsection{conterexample}
%
%Every pair $\triangle_{i,n} = \Delta_{i,n}$ except for a fixed $i^\prime$ and $n^\prime$ with
%\[
%\triangle_{i^\prime+1, n} =
%T \Delta_{i^\prime, n}, \quad n \geq n^\prime,
%\]
%and
%\[
%\triangle_{i^\prime, n} = \begin{cases}
%T \Delta_{i^\prime, n} & n < n^\prime, \\
%\Delta_{i^\prime, n} + \Delta_{i+1, n-1} \circ \bd_n & n = n^\prime,
%\end{cases}
%\]
%
%For example, to $\Delta_0([0123])$ this adds
%\[
%\begin{split}
%[1,2,3] \ot [1,2] + [1,3] \ot [1,2,3] + [1,2,3] \ot [2,3] + \\ [0,2,3] \ot [0,2] + [0,3] \ot [0,2,3] + [0,2,3] \ot [2,3] + \\ [0,1,3] \ot [0,1] + [0,3] \ot [0,1,3] + [0,1,3] \ot [1,3] + \\ [0,1,2] \ot [0,1] + [0,2] \ot [0,1,2] + [0,1,2] \ot [1,2] \phantom{+}
%\end{split}
%\]


%\subsection{formulas}
%
%We now give a new description of Steenrod's original cup-$i$ structure.
%Our description is in a sense dual to his and the equivalence of both is stated as Proposition \ref{proposition: steenrod's equals ours}.
%For any positive integer $q$ let $\P_q$ be the collection of cardinality $q$ subsets of non-negative integers and
%\[
%\P_q(n) = \{ U \in \P_q \mid \forall u \in U, u \leq n\}.
%\]
%For $U = \{u_1 < \cdots < u_q\} \in \P_q$ let
%\[
%d_U \colon \chains(X) \to \chains(X)
%\]
%be the linear map defined on a basis element $x \in X_n$ by
%\[
%d_U (x) = d_{u_1} \cdots \, d_{u_q} (x)
%\]
%with the convention that $d_U(x) = 0$ if $n < u_q$.
%For each $u_r \in U$ define the \textbf{index of $u_r$ in $U$} as
%\[
%\ind_U(u_r) = u_r + r
%\]
%denoting $U^0$ (resp. $U^1$) the subset of $U$ containing all elements whose index in $U$ is odd (resp. even).
%Either of these sets could be empty and we declare $d_\emptyset = \id$ and $\P_0 = \{\emptyset\}$.
%
%\begin{definition} \label{definition: our cup-i products}
%	For any simplicial set $X$ and cochains $\alpha, \beta \in \cochains(X)$ define for any $c \in \chains(X)_n$
%	\[
%	(\alpha \cup_i \beta)(c) =
%	(\alpha \ot \beta) \sum_{U \in \P_{n-i}} d_{U^0}(c) \ot d_{U^1}(c)
%	\]
%	if  $i \leq n$ and to be $0$ otherwise.
%\end{definition}




%\begin{lemma} \label{l:cup-i construction coalgebra}
%	A cup-$i$ construction is canonically equivalent to a cup-$i$ coproduct structure on $\chains(X)$ for every simplicial set $X$ that is natural with respect to simplicial maps, or,
%	in more categorical terms, to a collection of natural linear transformations $\triangle_i \colon \chains \to \chains \ot \chains$ for $i \in \N$ with $\triangle_0$ a natural chain map and
%	\[
%	\bd \circ \, \triangle_i + \triangle_i \circ \bd =
%	(1+T) \triangle_{i-1}
%	\]
%	for all $i > 0$.
%\end{lemma}

%\begin{proof}
%	By naturality, a cup-$i$ structure is determined by its restriction to representable simplicial sets $\simplex^n$.
%	Since $\chains(\simplex^n)$ is finite dimensional, \cref{l:cup-i coproduct structure equivalence} applies and the claim follows.
%\end{proof}

%We will refer to a cup-$i$ construction and to its defining associated collection of natural linear maps $\triangle_i \colon \chains \to \chains \ot \chains$ interchangeably.

%\begin{remark}
%	For any two cochains $\alpha$ and $\beta$ we have after unraveling the isomorphisms in the proof above that $\alpha \cup_i \beta = (\alpha \ot \beta) \triangle_i(-)$.
%\end{remark}

%\begin{remark}
%	In more categorical words, a cup-$i$ construction is canonically equivalent to a collection of natural linear transformations $\triangle_i \colon \chains \to \chains \ot \chains$ for $i \in \N$ with $\triangle_0$ a natural chain map and
%	\[
%	\bd \circ \, \triangle_i + \triangle_i \circ \bd =
%	(1+T) \triangle_{i-1}
%	\]
%	for all $i > 0$.
%\end{remark}