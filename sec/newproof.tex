% !TEX root = ../axiomatic.tex

\section{Proof}\label{s:proof}

In this section we present the proof of our main theorem: any non-degenerate and irreducible \mbox{cup-$i$} construction is isomorphic to the canonical one.

\subsection{Irreducible chains}

A basis element $V \ot W \in \cP(\simplex^n)^{\ot 2}$ is said to be \textit{irreducible} if $V \cap W = \emptyset$ and \textit{reducible} otherwise.
Let $\pired \colon \cP(\simplex^n)^{\ot 2} \to \cP(\simplex^n)^{\ot 2}$ be the projection to the subspace generated by reducible basis elements.
Explicitly,
\[
\pired(V \ot W) =
\begin{cases}
	V \ot W, & \text{if $V \ot W$ is reducible},\\
	\hfil 0, & \text{if not}.
\end{cases}
\]
A chain $\zeta \in \cP(\simplex^n)^{\ot 2}$ is said to be \textit{irreducible} if $\pired(\zeta) = 0$.
Notice that, by \cref{l:properties}, a cup-$i$ construction $\triangle$ is irreducible if and only if $\triangle_i[n]$ is irreducible for all $i, n \in \N$.

\subsection{Partition chains}

For any $U \subseteq \set{0,\dots,n}$, its \textit{partition chain} $\zeta_U$ is defined as the sum of all irreducible basis elements $V \ot W$ with $V \union W = U$ and $V \cap W = \emptyset$.

\begin{lemma}\label{l:partition chains}
	An irreducible chain $\zeta \in \cP(\simplex^n)^{\ot 2}$ satisfies $(\pired \circ \bd)(\zeta) = 0$ iff $\zeta$ is a sum of partition chains.
\end{lemma}

\begin{proof}
	Consider an irreducible chain $\zeta = \sum_{\Lambda} V \ot W$ with $\Lambda$ a subset of the basis of $\cP(\simplex^n)^{\ot 2}$.
	Its boundary can be decomposed as follows
	\[
	\bd \zeta = \sum_{\Lambda} \Big(\sum_{w \in W} w.V \ot W \ + \ \sum_{v \in V} V \ot v.W \ +
	\sum_{\bar u \notin V \union W} \bar u.V \ot  W + V \ot \bar u.W\Big).
	\]
	Therefore,
	\[
	(\pired \circ \bd) \zeta = \sum_{\Lambda} \Big(\sum_{w \in W} w.V \ot W \ + \ \sum_{v \in V} V \ot v.W\Big).
	\]
	Let us consider two irreducible basis elements $V \ot W$ and $V' \ot W'$, and elements $v \in V$, $w \in W$, $v' \in V'$, and $w' \in W'$.
	The following simple implications hold by direct inspection
	\begin{align*}
		&(w.V \ot W = w'.V' \ot W') &\implies& &&(w = w') \wedge (V = V') \wedge (W = W'), \\
		&(V \ot v.W = V' \ot v'.W') &\implies& &&(v = v') \wedge (V = V') \wedge (W = W'), \\
		&(w.V \ot W = V' \ot v'.W') &\implies& &&(w = v') \wedge (V' = w.V) \wedge (W' = W \setminus v'), \\
		&(V \ot v.W = w'.V' \ot W') &\implies& &&(v = w') \wedge (V' = V \setminus w) \wedge (W' = v.W).
	\end{align*}
	Now, if $(\pired \circ \bd) \zeta = 0$, these imply that for $V \ot W \in \Lambda$, $v \in V$, and $w \in W$ both $w.V \ot W \setminus w \in \Lambda$ and $V \setminus v \ot v.W \in \Lambda$.
	From this it follows that $\Lambda$ contains all irreducible basis elements $V' \ot W'$ with $V' \union W' = V \ot W$, that is to say, all summands of the partition chain of $V \union W$.
	Conversely, the above also implies that $(\pired \circ \bd)\zeta_U = 0$ for any partition chain $\zeta_U$.
	\anibal{Maybe write more about this direction.}
\end{proof}

\subsection{Special cases}\label{ss:cases}

\anibal{Is the only degenerate construction the 0 construction?}

We relate cup-$i$ constructions satisfying some of our axioms to the canonical cup-$i$ construction for special values of $i,n \in \N$.
These will serve as the base case of an induction argument in \cref{ss:proof}.

\begin{lemma}\label{l:special case one}
	Let $\big\{ \triangle_i [n] \big\}_{i,n\in\N}$ be free and non-degenerate \mbox{cup-$i$} construction and, as always, let $\big\{ \Delta_i [n] \big\}_{i,n\in\N}$ be the canonical one.
	\begin{enumerate}
		\item \label{i:i>n} $\forall i,n \in \N$, $\triangle_i[n] = \Delta_i [n] = 0$ if $i > n$.
		\item \label{i:i=n} $\forall n \in \N$, $\triangle_n[n] = \Delta_n [n] = \emptyset \ot \emptyset$.
	\end{enumerate}
\end{lemma}

%\begin{proof}
%	The chain complex $\cP(\simplex^n)^{\ot 2}$ is $0$ in degrees greater than $2n$ and it is generated by $\emptyset \ot \emptyset$ in degree $2n$.
%
%	The claim in \cref{i:i>n} is immediate since $\triangle_i[n]$ is in degree $n+i > 2n$ if $i > n$.
%
%	If the conclusion of the claim in \cref{i:i=n} does not hold, there exists $n \in \N$ smallest such that $\triangle_n [n] = 0$.
%	If $n > 0$ then
%	\begin{align*}
%		(1+T) \triangle_{n-1} [n] =
%		\bd \triangle_{n} [n] + \triangle_{n} \bd \, [n] = 0,
%	\end{align*}
%	and \cref{l:consequence} implies $\triangle_{n-1} [n] = 0$.
%	From this and the assumption
%	\[
%	\triangle_{n-1}[n-1] = \emptyset \ot \emptyset
%	\]
%	we obtain
%	\begin{equation}
%		\begin{split}
%			(1+T)\triangle_{n-2} [n] =
%			\bd \triangle_{n-1} [n] + \triangle_{n-1} \bd \, [n] =
%			\sum_{u = 0}^n \{u\} \ot \{u\},
%		\end{split}
%	\end{equation}
%	which is a contradiction since $\sum_u \{u\} \ot \{u\}$ is not in the image of $(1+T)$.
%
%	The previous argument shows that $\triangle_n [n] = 0$ for every $n \in \N$.
%	This serves as the base case of an induction argument over $n-i$ that will prove $\triangle_i [n] = 0$ for every $i, n \in \N$; a contradiction to the non-degeneracy of the cup-$i$ construction.
%	For the induction step, consider
%	\begin{align*}
%		(1+T) \triangle_{i-1} [n] =
%		\bd \triangle_{i} [n] + \triangle_{i} \bd\, [n] = 0,
%	\end{align*}
%	which, by \cref{l:consequence}, implies $\triangle_{i-1} [n] = 0$.
%\end{proof}

\begin{lemma}\label{l:special case two}
	Let $\big\{ \triangle_i [n] \big\}_{i,n\in\N}$ be free and non-degenerate \mbox{cup-$i$} construction.
	For all integer $n \geq 1$ either $\triangle_{n-1} [n]$ or $T \triangle_{n-1} [n]$ is equal to
	\[
	\Delta_{n-1} [n] \defeq
	\sum_{\mathclap{\substack{u \in \{0,\dots,n\} \\ u \ \mathrm{odd}}}} \{u\} \ot \emptyset +
	\sum_{\mathclap{\substack{u \in \{0,\dots,n\} \\ u \ \mathrm{even}}}} \emptyset \ot \{u\}.
	\]
\end{lemma}

%\begin{proof}
%	By \cref{l:special case one} we have $\triangle_{n} [n] = \emptyset \ot \emptyset$ and $\triangle_{n} \bd \, [n] = 0$ for all $n \in \N$.
%	Therefore,
%	\begin{align*}
%		(1+T) \triangle_{n-1} [n] &=
%		(\bd \ot \, \id + \id \ot \bd) (\emptyset \ot \emptyset) \\ &=
%		(1+T) \sum_{u=0}^n \{u\} \ot \emptyset
%	\end{align*}
%	and we need to show that the partition of the indexing set $\{0, \dots, n\} = \Lambda_0 \sqcup \Lambda_1$ provided by \cref{l:consequence} is determined by the parity of integers.
%	Let us argue by contradiction assuming some $j$ and $j+1$ belong to the same $\Lambda_{\varepsilon}$.
%	With no loss of generality let us assume $\varepsilon = 0$ so we have
%	\[
%	\triangle_{n-1} [n] = \big( \{j\} + \{j+1\} \big) \ot \emptyset + O(j, j+1)
%	\]
%	where $O(j, j+1)$ is a sum of basis elements missing $\{j\}$ and $\{j+1\}$ from both of its tensor factors.
%	Since $\triangle_{n-1} \bd \, [n] = \sum_{u=0}^{n} \{u\} \ot \{u\}$,
%	\[
%	(1+T) \triangle_{n-2} [n] = (1+T) \big( \{j\} \ot \{j+1\} \big) + P(j, j+1)
%	\]
%	where $P(j, j+1)$ is a sum of basis elements with $j$ and $j+1$ missing from at least one of its tensor factors.
%
%	By \cref{l:kernel of sxs} every basis element in $P(j,j+1)$ is in the $\ker \cP(\sigma_j)^{\ot 2}$.
%	Using \cref{l:consequence} in the above equation implies that $\triangle_{n-2}[n]$, an element in $\ker \cP(\sigma_j)^{\ot 2}$, is equal to either $\big( \{j\} \ot \{j+1\} \big)$ or $\big( \{j+1\} \ot \{j\} \big)$ plus an element in this kernel.
%	This is a contradiction since neither of these two basis elements is in $\ker \cP(\sigma_j)^{\ot 2}$ by \cref{l:kernel of sxs}.
%\end{proof}

\subsection{Facts about our formulas}\label{ss:fact}

We reprint two statements proven in \cite{medina2023fast_sq}.

\begin{notation*}
	For $U \in \P_{n-i}^n$ we write $\bar u \notin U$ if $\bar u \in \{0, \dots, n\} \setminus U$.
	We simplify notation writing $\bar u.U$ instead of $\{\bar u\} \union U$ if $\bar u \notin U$ and $U \setminus u$ instead of $U \setminus \{u\}$ if $u \in U$.
\end{notation*}

\begin{proposition}[{\cite[Lemma~21]{medina2023fast_sq}}]\label{p:fact1}
	For $i,n \in \N$ with $i < n$ we have:
	\[
	\Delta_i \bd \, [n] =
	\sum_{\mathclap{U \in \P_{n-i}^n}} \
	\left(\
	\sum_{\mathclap{u \in U^1}} u.U^0 \ot U^1 +
	\sum_{\mathclap{u \in U^0}} U^0 \ot u.U^1
	\right).
	\]
\end{proposition}

\begin{proposition}[{\cite[Lemma~??]{medina2023fast_sq}}]\label{p:fact2}
	TBW
\end{proposition}

\subsection{Complete proof}\label{ss:proof}

We now present the proof of our main result.

\begin{proof}[Proof of \cref{t:main}]
	Let $\triangle = \set[\big]{\triangle_i [n]}_{i,n\in\N}$ be a non-degenerate and irreducible \mbox{cup-$i$} construction.
	As usual, we denote the canonical cup-$i$ construction by $\Delta$.
	Let $\set[\big]{\p(i,n)}_{i,n \in \N}$ and $\set{\q(i,n)}_{i,n \in \N}$ respectively be the following two families of propositions:
	\[
	\set[\big]{\triangle_i [n] = \Delta_ i [n]}_{i,n \in \N}
	\quad \text{and} \quad
	\set[\big]{\triangle_i [n] = T \Delta_ i [n]}_{i,n \in \N} \ .
	\]
	We need to show that:
	\begin{center}
		($\ast$) For any fixed $i$ either $\p(i,n)$ or $\q(i,n)$ hold for every $n$.
	\end{center}
	We will use an induction argument over $k = n-i$ to show this.
	For $k \leq 0$, i.e. $i \geq n$, both $\p(i,n)$ and $\q(i,n)$ hold by \cref{l:special case one}, whereas for $k = 1$ one of them does by \cref{l:special case two}.
	Let us assume that our claim holds for $k-1$.
	For any $i,n \in \N$ with $n-i = k-1$ we have
	\begin{align}
		\label{eq:z}  (1+T)\triangle_i[n] &= (1+T)\Delta_i[n],\\
		\label{eq:zz} (1+T)\triangle_{i-1}[n-1] &= (1+T)\Delta_{i-1}[n-1].
	\end{align}
	This implies that
	\begin{align}
		\triangle_i[n] = \sum_{U \in \P_{n-i}^n} U^{\xi} \ot U^{\barxi} \ + \ \Phi, \\
		\triangle_{i-1}[n-1] = \sum_{U \in \P_{n-i}^{n-1}} U^{\eta} \ot U^{\bareta} \ + \ \Psi.
	\end{align}
	where $\xi \colon \P_{n-i}^n \to \set{0,1}$ and $\eta \colon \P_{n-i}^{n-1} \to \set{0,1}$ are some functions, and both $\Phi$ and $\Psi$ are irreducible elements in the kernel of $(1+T)$.
	Since $n-i \geq 2$ and $\Phi$ is irreducible, $\Phi = (1+T) \sum_{\lambda \in \Lambda} V_\lambda \ot W_\lambda$ for some indexing set $\Lambda$ with the property that $\lambda \neq \lambda'$ implies $(1+T)(V_\lambda \ot W_\lambda) \neq (1+T)(V_{\lambda'} \ot W_{\lambda'})$.
	We can assume that:
	\begin{equation}\label{eq:form of Phi}
		\forall\, U \in \P_{n-i}^n\,,\  \forall \lambda \in \Lambda\,,\ (1+T)(U^0 \ot U^1) \neq (1+T)(V_\lambda \ot W_\lambda)
	\end{equation}
	otherwise we can modify $\xi$ in order to ensure it.
	Let us assume that $\p(i,n-1)$ holds, i.e. $\triangle_i[n-1] = \Delta_i[n-1]$.
	The case where $\q(i,n-1)$ holds instead is done similarly.
	We can rewrite
	\[
	\triangle_i[n] = \Delta_i[n] \ +\
	(1+T) \sum_{\substack{U \in \P_{n-i}^n \\ \xi(U) \neq 0}} U^0 \ot U^1 \ +\
	%	(1+T) \sum_{\lambda \in \Lambda} V_\lambda \ot W_\lambda
	\Phi.
	\]
	Our goal is to show that $\xi \equiv 0$ and $\Phi = 0$.
	Using $\p(i,n-1)$ we have
	\begin{equation}\label{eq:main proof a}
		(1+T)\triangle_{i-1}[n] = (1+T)\Delta_{i-1}[n] \ +\
		\bd\Big((1+T)\sum_{\substack{U \in \P_{n-i}^n \\ \xi(U) \neq 0}} U^0 \ot U^1 \ + \ \Phi \Big).
	\end{equation}
	Since $(1+T)\triangle_{i-1}[n] + (1+T)\Delta_{i-1}[n]$ is in the kernel of $\pired$, we have that the irreducible element
	\[
	(1+T)\sum_{\substack{U \in \P_{n-i}^n \\ \xi(U) \neq 0}} U^0 \ot U^1 \ + \ \Phi
	\]
	is in the kernel of $(\pired \circ \bd)$ so, by \cref{l:partition chains}, it is a sum of partition chains.
	More explicitly, by \eqref{eq:form of Phi}, we have
	\[
	(1+T)\sum_{\substack{U \in \P_{n-i}^n \\ \xi(U) \neq 0}} U^0 \ot U^1 \ + \ \Phi \
	= \sum_{\substack{U \in \P_{n-i}^n \\ \xi(U) \neq 0}} \zeta_U,
	\]
	where $\zeta_U$ is the partition chain of $U$.
	From this we see that $\Phi = 0$ is implied by $\xi \equiv 0$.
	By Lemma?
	\begin{align*}
		\bd\Big((1+T)\sum_{\substack{U \in \P_{n-i}^n \\ \xi(U) \neq 0}} U^0 \ot U^1 \ + \ \Phi \Big)
		&= \bd \sum_{\substack{U \in \P_{n-i}^n \\ \xi(U) \neq 0}} \zeta_U \\
		&= \sum_{\substack{U \in \P_{n-i}^n \\ \xi(U) \neq 0}} \sum_{\bar{u} \notin U} \zeta_{\bar{u}.U}.
	\end{align*}
	Combining this with \eqref{eq:main proof a} we have
	\[
	(1+T)\triangle_{i-1}[n] = (1+T)\Delta_{i-1}[n] \ +\
	\sum_{\substack{U \in \P_{n-i}^n \\ \xi(U) \neq 0}} \sum_{\bar{u} \notin U} \zeta_{\bar{u}.U}.
	\]
	Therefore,
	\[
	(1+T)\triangle_{i-1}[n] = (1+T)\Delta_{i-1}[n] \ +\
	\sum_{\substack{U \in \P_{n-i}^n \\ \xi(U) \neq 0}} \sum_{\bar{u} \notin U} \zeta_{\bar{u}.U}.
	\]
%	Additionally,
%	\[
%	\triangle_i[n] = \Delta_i[n] \ + \sum_{\substack{U \in \P_{n-i}^n \\ \xi(U) \neq 0}} \zeta_U.
%	\]
%	From this we see that $\Phi = 0$ is implied by $\xi \equiv 0$.
%	We need to show that $\xi \equiv 0$.
%	Since, by Lemma??,
%	\[
%	(1+T)\triangle_{i-1}[n] = (1+T)\Delta_{i-1}[n] \ +\
%	\sum_{\substack{U \in \P_{n-i}^n \\ \xi(U) \neq 0}} \sum_{\bar u \notin U} \zeta_{\bar u.U}.
%	\]


	%	 case of the induction and \cref{l:induction step} as the induction step (\cref{f:induction step}).
\end{proof}