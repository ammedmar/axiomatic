\subsection{}

\begin{notation}
	Given $\xi \colon \P_{n-i}(n) \to \F$ we denote by $\bar{\xi}$ the function with the same domain and range satisfying $\xi(U) \neq \bar{\xi}(U)$ for every $U$, and simplify the notation $U^{\xi(U)}$ and $U^{\bar{\xi}(U)}$ to $U^\xi$ and $U^{\bar{\xi}}$ respectively.
\end{notation}

A direct consequence of Lemma \ref{l:(1+T) triangle = 0 implies triangle = 0} is the following.

\begin{lemma}
    If
    \[
    (1+T) \triangle_i [n] =
    (1+T) \sum_{\P_{n-i}(n)} {U^0} \ot {U^1}
    \]
    for some $i < n$, then there exists $\xi \colon \P_{n-i}(n) \to \F$  such that
    \[
    \triangle_i  [n]\ =\! \sum_{\P_{n-i}(n)} {U^{\xi}} \ot {U^{\bar{\xi}}}.
    \]
\end{lemma}

\begin{notation}
    For $U \in \P_{n-i}(n)$ we write $u \notin U$ if $u \in \{0, \dots, n\} \setminus U$. If $u \notin U$ define $u.U = \{u\} \union U \in \P_{n-i+1}(n)$ and if $u \in U$ define $U \setminus u = U \setminus \{u\} \in \P_{n-i-1}(n)$.
\end{notation}

\begin{lemma} \label{l:boundary triangle}
	If for some $i < n$ there exists $\xi \colon \P_{n-i}(n) \to \F$ such that
	\[
	\triangle_i [n]\ =\! \sum_{\P_{n-i}(n)} {U^{\xi}} \ot {U^{{\bar{\xi}}}}
	\]
	then
	\begin{align*}
	\label{equation: boundary of triangle}
	\bd \triangle_i [n]\ = &
	\sum_{\P_{n-i}(n)} \left( \, \sum_{u \in U^{\bar{\xi}}} {u.U^\xi} \ot {U^{\bar{\xi}}} \ +
	\sum_{u \in U^\xi} {U^\xi} \ot {u.U^{\bar{\xi}}} \right) \\ + &
	\sum_{\P_{n-i}(n)} \sum_{x \notin U} \left( {x.U^0} \ot {U^1}\ +\ {U^0} \ot {x.U^1} \right).
	\end{align*}
\end{lemma}

\begin{lemma} \label{l:triangle boundary}
	If for some $i \leq n$
	\[
	\triangle_i [n-1]\ =\!
	\sum_{\P_{n-i-1}(n)} {U^0} \ot {U^1}
	\]
	then
	\begin{align}
	\label{equation: triangle of boundary}
	\triangle_i \bd \, [n] \ =\!
	\sum_{\P_{n-i}(n)} \left( \,
	\sum_{u \in U^1} {u.U^0} \ot {U^1} \ +
	\sum_{u \in U^0} {U^0} \ot {u.U^1} \right).
	\end{align}
\end{lemma}

\begin{proof}
	Let
	\begin{align*}
	& S_1 = \big\{ (u, V)\ |\ V \in \P_{n-i-1}(n-1) \text{ and }  u \in \{0, \dots, n \} \big\}, \\
	& S_2 = \big\{ (w, W)\ |\ W \in \P_{n-i}(n) \text{ and } w \in W \big\}
	\end{align*}
	and notice that identity \eqref{equation: triangle of boundary} is, using the isomorphism
	\begin{align*}
	\cP[n] &\cong \chains \simplex^n \\
	U &\mapsto \delta_U ,
	\end{align*}
	equivalent to
	\begin{equation} \label{e:triangle of partial recasted}
	\sum_{(u, V) \in S_1}
	\delta_{V^\xi} \circ \delta_u \ot \delta_{V^{\bar{\xi}}} \circ \delta_u \ \, =
	\sum_{\substack{(w, W) \in S_2 \\ w \in W^{\bar{\xi}}}}
	\delta_{w.W^\xi} \ot \delta_{W^{\bar{\xi}}} \ + \!
	\sum_{\substack{(w, W) \in S_2 \\ w \in W^\xi}}
	\delta_{W^\xi} \ot \delta_{w.W^{\bar{\xi}}.}
	\end{equation}
	Define $S_1 \to S_2$ by sending $\big(u, \, \{v_1 < \cdots < v_{n-i-1}\} \big)$ to $\big(u,\, \{w_1 < \cdots < w_{q}\} \big)$ with
	\[
	w_i =
	\begin{cases}
	v_i & \text{ if } v_i < u \\
	u & \text{ if } v_i < u \leq v_{i+1} \\
	v_{i-1}+1 & \text{ if } v_i < u.
	\end{cases}
	\]
	This function is a bijection since it is injective and both sets have the same cardinality.
	To establishes \eqref{e:triangle of partial recasted} we use the simplicial identities to notice that
	\[
	\delta_{V^\xi} \circ \delta_u \ot \delta_{V^{\bar{\xi}}} \circ \delta_u =
	\begin{cases}
	\delta_{w.W^\xi} \ot \delta_{W^{\bar{\xi}}} & \text{ if } w \in W^{\bar{\xi}}, \\
	\delta_{W^\xi} \ot \delta_{w.W^{\bar{\xi}}} & \text{ if } w \in W^\xi,
	\end{cases}
	\]
	if $(u, V) \mapsto (w, W)$.
\end{proof}

\begin{lemma} \label{lemma: boundary gives the lower case}
	For $i \leq n$
	\begin{equation} \label{lemma4: existence: eq1}
	\sum_{U \in \P_{n-i}(n)} \sum_{x \notin U} ({x.U^0} \ot {U^1}\ +\ {U^0} \ot {x.U^1})\ = \
	(1+T) \!\! \sum_{\P_{n-i+1}(n)} {U^0} \ot {U^1.}
	\end{equation}
\end{lemma}

\begin{proof}
	For $U = (u_1, \dots, u_{n-i}) \in \P_{n-i}(n)$ and $x \notin U$ define when possible:
	\begin{align*}
	V_{U,x} = x.U \setminus l_{U, x} & \ \text{ with } l_{U,x} = \max\{u\in U\ |\ x>u\}, \\
	W_{U,x} = x.U \setminus r_{U, x} & \ \text{ with } r_{U,x} = \max\{u \in U\ |\ x>u\}.
	\end{align*}
	Notice that $(l_{U,x}).V_{U,x} = x.U = (r_{U,x}).W_{U,x}$ and that for any $u \in U \setminus \{l_{U,x}, r_{U,x}\}$ we have $\ind_{V_{U,x}}(u) = \ind_{U}(u) = \ind_{W_{U,x}}(u)$.

	We introduce the following sets using tabbing to represent inclusion:

	\vspace*{5pt}
	$L = \{x.U^0 \ot U^1\ | \ U \in \P_{n-i}(n),\ x \notin U\}$
	\vspace*{1pt}
	\begin{tab}
		$L^{e} = \{x.U^0 \ot U^1 \in L\ | \ \ind_{x.U}(x) \text{ even}\}$
		\begin{tab}
			$L_{min}^{e} = \{x.U^0 \ot U^1 \in L^e\ | \ x < u_1 \}$ \par
			$\overline{L}_{min}^{e} = L^{e} \setminus L_{min}^{e}$
			\begin{tab}
				$\overline{L}_{min}^{e,e} = \{ x.U^0 \ot U^1 \in \overline{L}_{min}^{e}\ | \ \ind_{x.U}(l_{U,x}) \text{ even} \}$ \par
				$\overline{L}_{min}^{e,o} = \{ x.U^0 \ot U^1 \in \overline{L}_{min}^{e}\ | \ \ind_{x.U}(l_{U,x}) \text{ odd} \}$
				\vspace*{7pt}
			\end{tab}
		\end{tab}
		$L^{o} = \{x.U^0 \ot U^1\ | \ \ind_{x.U}(x) \text{ odd}\}$ \par
		\begin{tab}
			$L_{max}^{o} = \{x.U^0 \ot U^1 \in L^o\ | \ u_q < x \}$ \par
			$\overline{L}_{max}^{o} = L^{o}\setminus L_{max}^{o}$.
			\begin{tab}
				$\overline{L}_{max}^{o,e} = \{ x.U^0 \ot U^1 \in \overline{L}_{max}^{o}\ | \ \ind_{x.U}(r_{U,x}) \text{ even} \}$ \par
				$\overline{L}_{max}^{o,o} = \{ x.U^0 \ot U^1 \in \overline{L}_{max}^{o}\ | \ \ind_{x.U}(r_{U,x}) \text{ odd} \}$.
			\end{tab}
		\end{tab}
	\end{tab}

	\vspace*{5pt}
	Similar subsets of $R = \{U^0\ot x.U^1\ | \ U \in \P_{n-i}(n),\ x \in \overline{U}\}$ are defined analogously, and we claim the following four identities:
	\begin{equation}\label{lemma4: existence: eq2}
	\overline{R}_{min}^{o,o} = \overline{L}_{max}^{o,o}\ , \qquad \overline{R}_{min}^{o,e} = \overline{R}_{max}^{e,o}\ , \qquad
	\overline{L}_{min}^{e,o} = \overline{L}_{max}^{o,e}\ , \qquad \overline{L}_{min}^{e,e} = \overline{R}_{max}^{e,e}.
	\end{equation}
	We show only the proof of the first one. The other three are proven analogously.
	\begin{alignat*}{2}
	&\boxed{x.U^0\ot U^1 \in \overline{L}_{max}^{o,o}}\ &\Longrightarrow\ &\boxed{x.U^0\ot U^1 =\, W_U(x)^0\ot r_{U,x}.W_{U,x}^1 \in \overline{R}_{min}^{o,o} \! } \\
	&\boxed{U^0\ot x.U^1 \in \overline{R}_{min}^{o,o}}\ &\Longrightarrow\ &\boxed{U^0\ot x.U^1 =\, l_{U,x}.V_U(x)^0\ot V_{U,x}^1 \in \overline{L}_{max}^{o,o} \ \ }
	\end{alignat*}
	The identities in (\ref{lemma4: existence: eq2}) imply
	\begin{equation} \label{lemma4: existence: eq3}
	\sum_{\overline{L}_{max}^{o},\, \overline{L}_{min}^{e}} d_{x.U^0} \ot d_{U^1}\ \ +
	\sum_{\overline{R}_{max}^{e},\, \overline{R}_{min}^{o}} d_{U^0} \ot d_{x.U^1}\ =\ 0.
	\end{equation}
	Let us now consider the right hand side of (\ref{lemma4: existence: eq1})
	\[
	(1+T) \sum_{U \in \P_{n-i+1}(n)} {U^0} \ot {U^1} \, = \!\!
	\sum_{L^0_{max},\, L^r_{min}} {x.U^0} \ot {U^1}\ + \!
	\sum_{R^e_{max}, \, R^o_{min}} {U^0} \ot {x.U^1,}
	\]
	Thanks to (\ref{lemma4: existence: eq3}), the above expression is equivalent to
	\[
	(1+T) \sum_{U \in \P_{n-i+1}(n)} {U^0} \ot {U^1} \, = \
	\sum_{L} {x.U^0} \ot {U^1}\ + \
	\sum_{R} {U^0} \ot {x.U^1.}
	\]
	as claimed.
\end{proof}

%\begin{corollary}
%	The map $\Delta \colon W \to \mathcal Z(2)$ is a free, non-degenerate Steenrod construction.
%\end{corollary}
%
%\begin{proof}
%	...
%\end{proof}