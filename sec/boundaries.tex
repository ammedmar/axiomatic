\subsection{}

\begin{notation}
	Given $\xi \colon \P_{n-i}(n) \to \F$ we denote by $\bar{\xi}$ the function with the same domain and range satisfying $\xi(U) \neq \bar{\xi}(U)$ for every $U$, and simplify the notation $U^{\xi(U)}$ and $U^{\bar{\xi}(U)}$ to $U^\xi$ and $U^{\bar{\xi}}$ respectively.
\end{notation}

A direct consequence of Lemma \ref{l:(1+T) triangle = 0 implies triangle = 0} is the following.

\begin{lemma}
    If
    \[
    (1+T) \triangle_i [n] =
    (1+T) \sum_{\P_{n-i}(n)} {U^0} \ot {U^1}
    \]
    for some $i < n$, then there exists $\xi \colon \P_{n-i}(n) \to \F$  such that
    \[
    \triangle_i  [n]\ =\! \sum_{\P_{n-i}(n)} {U^{\xi}} \ot {U^{\bar{\xi}}}.
    \]
\end{lemma}

\begin{notation}
    For $U \in \P_{n-i}(n)$ we write $u \notin U$ if $u \in \{0, \dots, n\} \setminus U$. If $u \notin U$ define $u.U = \{u\} \union U \in \P_{n-i+1}(n)$ and if $u \in U$ define $U \setminus u = U \setminus \{u\} \in \P_{n-i-1}(n)$.
\end{notation}

\begin{lemma} \label{l:boundary triangle}
	If for some $i < n$ there exists $\xi \colon \P_{n-i}(n) \to \F$ such that
	\[
	\triangle_i [n]\ =\! \sum_{\P_{n-i}(n)} {U^{\xi}} \ot {U^{{\bar{\xi}}}}
	\]
	then
	\begin{align*}
	\label{equation: boundary of triangle}
	\bd \triangle_i [n]\ = &
	\sum_{\P_{n-i}(n)} \left( \, \sum_{u \in U^{\bar{\xi}}} {u.U^\xi} \ot {U^{\bar{\xi}}} \ +
	\sum_{u \in U^\xi} {U^\xi} \ot {u.U^{\bar{\xi}}} \right) \\ + &
	\sum_{\P_{n-i}(n)} \sum_{x \notin U} \left( {x.U^0} \ot {U^1}\ +\ {U^0} \ot {x.U^1} \right).
	\end{align*}
\end{lemma}

\begin{lemma} \label{l:triangle boundary}
	If for some $i \leq n$
	\[
	\triangle_i [n-1]\ =\!
	\sum_{\P_{n-i-1}(n)} {U^0} \ot {U^1}
	\]
	then
	\begin{align}
	\label{equation: triangle of boundary}
	\triangle_i \bd \, [n] \ =\!
	\sum_{\P_{n-i}(n)} \left( \,
	\sum_{u \in U^1} {u.U^0} \ot {U^1} \ +
	\sum_{u \in U^0} {U^0} \ot {u.U^1} \right).
	\end{align}
\end{lemma}

\anibal{proof in the other paper}

%\begin{proof}
%	Let
%	\begin{align*}
%	& S_1 = \big\{ (u, V) \mid V \in \P_{n-i-1}(n-1) \text{ and }  u \in \{0, \dots, n \} \big\}, \\
%	& S_2 = \big\{ (w, W) \mid W \in \P_{n-i}(n) \text{ and } w \in W \big\}
%	\end{align*}
%	and notice that identity \eqref{equation: triangle of boundary} is, using the isomorphism
%	\begin{align*}
%	\cP[n] &\cong \chains \simplex^n \\
%	U &\mapsto \delta_U ,
%	\end{align*}
%	equivalent to
%	\begin{equation} \label{e:triangle of partial recasted}
%	\sum_{(u, V) \in S_1}
%	\delta_{V^\xi} \circ \delta_u \ot \delta_{V^{\bar{\xi}}} \circ \delta_u \ \, =
%	\sum_{\substack{(w, W) \in S_2 \\ w \in W^{\bar{\xi}}}}
%	\delta_{w.W^\xi} \ot \delta_{W^{\bar{\xi}}} \ + \!
%	\sum_{\substack{(w, W) \in S_2 \\ w \in W^\xi}}
%	\delta_{W^\xi} \ot \delta_{w.W^{\bar{\xi}}.}
%	\end{equation}
%	Define $S_1 \to S_2$ by sending $\big(u, \, \{v_1 < \cdots < v_{n-i-1}\} \big)$ to $\big(u,\, \{w_1 < \cdots < w_{q}\} \big)$ with
%	\[
%	w_i =
%	\begin{cases}
%	v_i & \text{ if } v_i < u \\
%	u & \text{ if } v_i < u \leq v_{i+1} \\
%	v_{i-1}+1 & \text{ if } v_i < u.
%	\end{cases}
%	\]
%	This function is a bijection since it is injective and both sets have the same cardinality.
%	To establishes \eqref{e:triangle of partial recasted} we use the simplicial identities to notice that
%	\[
%	\delta_{V^\xi} \circ \delta_u \ot \delta_{V^{\bar{\xi}}} \circ \delta_u =
%	\begin{cases}
%	\delta_{w.W^\xi} \ot \delta_{W^{\bar{\xi}}} & \text{ if } w \in W^{\bar{\xi}}, \\
%	\delta_{W^\xi} \ot \delta_{w.W^{\bar{\xi}}} & \text{ if } w \in W^\xi,
%	\end{cases}
%	\]
%	if $(u, V) \mapsto (w, W)$.
%\end{proof}

\begin{lemma} \label{lemma: boundary gives the lower case}
	For $i \leq n$
	\begin{equation} \label{lemma4: existence: eq1}
	\sum_{U \in \P_{n-i}(n)} \sum_{x \notin U} ({x.U^0} \ot {U^1}\ +\ {U^0} \ot {x.U^1})\ = \
	(1+T) \!\! \sum_{\P_{n-i+1}(n)} {U^0} \ot {U^1.}
	\end{equation}
\end{lemma}

\anibal{also proven in the other paper}

%\begin{proof}
%	For $U = (u_1, \dots, u_{n-i}) \in \P_{n-i}(n)$ and $x \notin U$ define when possible:
%	\begin{align*}
%	V_{U,x} = x.U \setminus l_{U, x} & \ \text{ with } l_{U,x} = \max \{u \in U \mid u<x \}, \\
%	W_{U,x} = x.U \setminus r_{U, x} & \ \text{ with } r_{U,x} = \min \{u \in U \mid x<u \}.
%	\end{align*}
%	Notice that
%	\[
%	(l_{U,x}).V_{U,x} = x.U = (r_{U,x}).W_{U,x}
%	\]
%	and that
%	\[
%	\ind_{V_{U,x}}(u) = \ind_{U}(u) = \ind_{W_{U,x}}(u)
%	\]
%	for any $u \in U \setminus \{l_{U,x}, r_{U,x}\}$.
%
%	We introduce the following sets using tabbing to represent inclusion:
%
%	\vspace*{5pt}
%	$L = \{x.U^0 \ot U^1 \mid U \in \P_{n-i}(n),\ x \notin U\}$
%	\vspace*{1pt}
%	\begin{tab}
%		$L^{(0)} = \{x.U^0 \ot U^1 \in L \mid \ind_{x.U}(x) = 0\}$
%		\begin{tab}
%			$L_{\max}^{(0)} = \{x.U^0 \ot U^1 \in L^{(0)} \mid u_{n-i} < x \}$ \par
%			$\oL_{\max}^{(0)} = L^{(0)} \setminus L_{\max}^{(0)}$
%			\begin{tab}
%				$\oL_{\max}^{(0,0)} = \{ x.U^0 \ot U^1 \in \oL_{\max}^{(0)} \mid \ind_{x.U}(l_{U,x}) = 0 \}$ \par
%				$\oL_{\max}^{(0,1)} = \{ x.U^0 \ot U^1 \in \oL_{\max}^{(0)} \mid \ind_{x.U}(l_{U,x}) = 1 \}$
%				\vspace*{7pt}
%			\end{tab}
%		\end{tab}
%		$L^{(1)} = \{x.U^0 \ot U^1 \mid \ind_{x.U}(x) \text{ odd}\}$ \par
%		\begin{tab}
%			$L_{\min}^{(1)} = \{x.U^0 \ot U^1 \in L^{(1)} \mid u_{n-i} < x \}$ \par
%			$\oL_{\min}^{(1)} = L^{(1)} \setminus L_{\min}^{(1)}$
%			\begin{tab}
%				$\oL_{\min}^{(1,0)} = \{ x.U^0 \ot U^1 \in \oL_{\min}^{(1)} \mid \ind_{x.U}(r_{U,x}) = 0 \}$ \par
%				$\oL_{\min}^{(1,1)} = \{ x.U^0 \ot U^1 \in \oL_{\min}^{(1)} \mid \ind_{x.U}(r_{U,x}) = 1 \}$.
%			\end{tab}
%		\end{tab}
%	\end{tab}
%
%	\vspace*{5pt}
%	Similar subsets of $R = \{U^0\ot x.U^1 \mid U \in \P_{n-i}(n),\ x \in \overline{U}\}$ are defined analogously, and we claim the following four identities:
%	\begin{equation}\label{lemma4: existence: eq2}
%	\oL_{\max}^{(0,0)} = \oR_{\min}^{(0,0)}\ , \qquad
%	\oL_{\max}^{(0,1)} = \oL_{\min}^{(1,0)}\ , \qquad
%	\oR_{\min}^{(0,1)} = \oR_{\max}^{(1,0)}\ , \qquad
%	\oL_{\min}^{(1,1)} = \oR_{\max}^{(1,1)}.
%	\end{equation}
%	We show only the proof of the first one. The other three are proven analogously.
%	\begin{alignat*}{2}
%	&\boxed{x.U^0 \ot U^1 \in \oL_{\max}^{(0,0)} \hspace*{1pt} }\ &\Longrightarrow\
%	&\boxed{x.U^0 \ot U^1 =\, W_{U,x}^0 \ot r_{U,x}.W_{U,x}^1 \in \oR_{\min}^{(0,0)} \! } \\
%	&\boxed{U^0 \ot x.U^1 \in \oR_{\min}^{(0,0)}}\ &\Longrightarrow\
%	&\boxed{U^0 \ot x.U^1 =\, l_{U,x}.V_{U,x}^0 \ot V_{U,x}^1 \in \oL_{\max}^{(0,0)} \hspace*{8pt}}
%	\end{alignat*}
%	The identities in (\ref{lemma4: existence: eq2}) imply
%	\begin{equation} \label{lemma4: existence: eq3}
%	\sum_{\oL_{\max}^{(0)} \union \, \oL_{\min}^{(1)}} x.U^0 \ot U^1\ \ +
%	\sum_{\oR_{\min}^{(0)} \union \, \oR_{\max}^{(1)}} U^0 \ot x.U^1\ =\ 0.
%	\end{equation}
%
%	\anibal{Continue here}
%
%
%	Let us now consider the right hand side of (\ref{lemma4: existence: eq1})
%	\[
%	(1+T) \sum_{U \in \P_{n-i+1}(n)} {U^0} \ot {U^1} \, = \!\!
%	\sum_{L^o_{max},\, L^r_{min}} {x.U^0} \ot {U^1}\ + \!
%	\sum_{R^e_{max}, \, R^o_{min}} {U^0} \ot {x.U^1,}
%	\]
%	Thanks to (\ref{lemma4: existence: eq3}), the above expression is equivalent to
%	\[
%	(1+T) \sum_{U \in \P_{n-i+1}(n)} {U^0} \ot {U^1} \, = \
%	\sum_{L} {x.U^0} \ot {U^1}\ + \
%	\sum_{R} {U^0} \ot {x.U^1.}
%	\]
%	as claimed.
%\end{proof}

%\begin{corollary}
%	The map $\Delta \colon W \to \mathcal Z(2)$ is a free, non-degenerate Steenrod construction.
%\end{corollary}
%
%\begin{proof}
%	...
%\end{proof}