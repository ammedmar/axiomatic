\subsection{}

\begin{notation}
	Given $\xi \colon \P_{n-i}(n) \to \F$ we denote by $\bar{\xi}$ the function with the same domain and range satisfying $\xi(U) \neq \bar{\xi}(U)$ for every $U$, and simplify the notation $U^{\xi(U)}$ and $U^{\bar{\xi}(U)}$ to $U^\xi$ and $U^{\bar{\xi}}$ respectively.
\end{notation}

A direct consequence of Lemma \ref{l:(1+T) triangle = 0 implies triangle = 0} is the following.

\begin{lemma}
    If
    \[
    (1+T) \triangle_i [n] =
    (1+T) \sum_{\P_{n-i}(n)} {U^0} \ot {U^1}
    \]
    for some $i < n$, then there exists $\xi \colon \P_{n-i}(n) \to \F$  such that
    \[
    \triangle_i  [n]\ =\! \sum_{\P_{n-i}(n)} {U^{\xi}} \ot {U^{\bar{\xi}}}.
    \]
\end{lemma}

\anibal{Make a more general statement that includes other type of summands. It must also apply to the second lemma in the induction step}

\begin{notation}
    For $U \in \P_{n-i}(n)$ we write $u \notin U$ if $u \in \{0, \dots, n\} \setminus U$. If $u \notin U$ define $u.U = \{u\} \union U \in \P_{n-i+1}(n)$ and if $u \in U$ define $U \setminus u = U \setminus \{u\} \in \P_{n-i-1}(n)$.
\end{notation}

\begin{lemma} \label{l:triangle boundary}
	If for some $i \leq n$
	\[
	\triangle_i [n-1]\ =\!
	\sum_{\P_{n-i-1}(n)} {U^0} \ot {U^1}
	\]
	then
	\begin{align}
	\label{e:triangle of boundary}
	\triangle_i \bd \, [n] \ =\!
	\sum_{\P_{n-i}(n)} \left( \,
	\sum_{u \in U^1} {u.U^0} \ot {U^1} \ +
	\sum_{u \in U^0} {U^0} \ot {u.U^1} \right).
	\end{align}
\end{lemma}

\anibal{proof in the other paper}

\begin{lemma} \label{l:boundary gives the lower case}
	For $i \leq n$
	\begin{equation*}
	(1+T) \!\! \sum_{\P_{n-i+1}(n)} {U^0} \ot {U^1} =
	\sum_{U \in \P_{n-i}(n)} \sum_{x \notin U} ({x.U^0} \ot {U^1}\ +\ {U^0} \ot {x.U^1})
	\end{equation*}
\end{lemma}

\anibal{also proven in the other paper}