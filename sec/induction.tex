\subsection{Induction step}

\begin{lemma}
     If for some $i \leq n - 2$ there exists $\xi : \P_{n-i}(n) \to \{-, +\}$ such that
     \[
     \triangle_i(\id_{[n]})\ =\! \sum_{\P_{n-i}(n)} \delta_{U^{\xi}} \otimes \delta_{U^{{\bar{\xi}}}}
     \]
     then,
     \begin{align*}
     \triangle_i(\id_{[n-1]}) = \Delta_i(\id_{[n-1]})
     \ & \Rightarrow \
     \triangle_i(\id_{[n]}) = \Delta_i(\id_{[n]}) \\
     \triangle_i(\id_{[n-1]}) = T \Delta_i(\id_{[n-1]})
     \ & \Rightarrow \
     \triangle_i(\id_{[n]}) = T \Delta_i(\id_{[n]})
     \end{align*}
\end{lemma}

%\begin{proof}
%	Let us assume $\triangle_i(\id_{[n-1]}) = \sum_{\P_{n-1-i}(n-1)} \delta_{U^-} \otimes \delta_{U^+}$. The other case is done analogously. Rewriting \eqref{equation: boundary of triangle} we have
%    \begin{align*}
%    \partial \triangle_i(\id_{[n]})\ & =\! \sum_{\P_{n-i}(n)} \left(\sum_{u \notin U} \delta_{u.U^\xi} \otimes \delta_{U^{\bar{\xi}}} +
%    \sum_{u \notin U} \delta_{U^\xi} \otimes \delta_{u.U^{\bar{\xi}}} \right) \\ & +\!
%    \sum_{\P_{n-i}(n)} \left(\sum_{u \in U^{\bar{\xi}}} \delta_{u.U^\xi} \otimes \delta_{U^{\bar{\xi}}} +
%    \sum_{u \in U^{\xi}} \delta_{U^\xi} \otimes \delta_{u.U^{\bar{\xi}}} \right)
%    \end{align*}
%    and by adding $\triangle(\partial \, \id_{[n]})$ to it and using identity \eqref{equation: triangle of boundary} we get
%    \begin{align}
%    (1 + T) \triangle_{i-1}(\id_{[n]})\ & =\! \sum_{\P_{n-i}(n)} \left(\sum_{u \notin U} \delta_{u.U^\xi} \otimes \delta_{U^{\bar{\xi}}} +
%    \sum_{u \notin U} \delta_{U^\xi} \otimes \delta_{u.U^{\bar{\xi}}} \right) \label{equation: top}\\ +\
%    (1 + T) & \sum_{\substack{\P_{n-i}(n) \\ \xi(U) \neq -}} \left(\sum_{u \in U^+} \delta_{u.U^-} \otimes \delta_{U^{+}} +
%    \sum_{u \in U^-} \delta_{U^-} \otimes \delta_{u.U^+} \right).
%    \end{align}
%    For $j \in \{0, \dots, n\}$, Lemma \ref{lemma: condition to be in the kernel of s} tells us that the basis elements appearing in the right hand of \eqref{equation: top} are in the kernel of $(s_j)_\ast^{\otimes 2}$ for any $j$. Let $\Lambda_{j, j+1}^\pm$ be the subset of pairs $(u, U)$ with $U \in \P_{n-i}(n)$ and $u \in U^\pm$ such that $u \in \{j, j+1\}$ and $\xi(U) \neq -$. Since the maps $(1+T)$ and $(s_j)_\ast^{\otimes 2}$ commute, applying $(s_j)_\ast^{\otimes 2}$ to the identity above implies that
%    \begin{equation} \label{equation: ptt}
%    \sum_{\Lambda_{j, j+1}^-} \delta_{U^- \setminus u} \otimes \delta_{U^+} +
%    \sum_{\Lambda_{j, j+1}^+} \delta_{U^-} \otimes \delta_{U^+ \setminus u}
%    \end{equation}
%    is in the kernel of $(1+T)$. We can reduce the sum above by noticing that if both $j$ and $j+1$ are in $U$, they are both in one of $U^-$ or $U^+$; so the basis elements associated to $(j, U)$ and $(j+1, U)$ in the sum above cancel each other. Denoting by $\Lambda^\pm_j$ the subset of $\Lambda^\pm_{j, j+1}$ containing $j$ but not $j+1$ and similarly $\Lambda^\pm_{j+1}$, the element \eqref{equation: ptt} can be written as
%    \begin{equation} \label{equation: linda}
%    \sum_{\Lambda_j^-,\, \Lambda_{j+1}^-} \delta_{U^- \setminus u} \otimes \delta_{U^+} \ +
%    \sum_{\Lambda_j^+,\, \Lambda_{j+1}^+} \delta_{U^-} \otimes \delta_{U^+ \setminus u}.
%    \end{equation}
%    The fact that \eqref{equation: linda} is in the kernel of $(1+T)$ implies the existence of an involution $\phi$ of $\Lambda = \Lambda^-_{j} \sqcup \Lambda^-_{j+1} \sqcup \Lambda^+_{j} \sqcup \Lambda^+_{j+1}$ defined by a choice of canceling pairs. Since $i \leq n-2$, this involution has no fixed points. So, for any $(u, U) \in \Lambda$, if $\phi(u, U) = (u^\prime, U^\prime)$ then
%    \[
%    U \setminus u = U^\prime \setminus u^\prime, \qquad u \neq u^\prime, \qquad \xi(U) = \xi(U^\prime)
%    \]
%    where the last equality holds since $(u^\prime, U^\prime) \in \Lambda$. We remark that the first two conditions completely determine $U^\prime$.
%
%    We introduce a symmetric relation in $\P_{n-i}(n)$ writing $V \sim W$ if there exists $k$ such that $V = k.W \setminus(k+1)$ or $W = k.V \setminus(k+1)$. The previous analysis implies that if $V \sim W$ then $\xi(V) = \xi(W)$. The map $\xi$ is constant since there exists a sequence
%    \[
%    V \sim \cdots \sim W
%    \]
%    for any for any pair $V, W \in \P_{n-i}(n)$.
%\end{proof}

%\begin{figure}
%    \centering
%    \begin{tikzpicture}[scale = .65]
%    \draw (0,0)--(0,6);
%    \draw (0,0)--(11,0);
%    \draw (0,0)--(6,6);
%    \draw (1,0)--(7,6);
%    \draw (3,0)--(9,6);
%    \draw (4,0)--(10,6);
%    \draw[dashed] (5,0)--(11,6);
%
%    \draw[dotted, shorten >=10pt,shorten <=10pt] (3,2)--(4,1);
%    \draw[dotted, shorten >=10pt,shorten <=10pt] (5,4)--(6,3);
%    \draw[dotted, shorten >=10pt,shorten <=10pt] (7,6)--(8,5);
%
%    \fill (7,4) circle (0.1);
%    \fill (8,4) circle (0.1);
%    \fill (7,3) circle (0.1);
%    \fill[white] (8,3) circle (0.1);
%    \draw (8,3) circle (0.1);
%
%    \node[above] at (7,4) {p};
%    \node[above] at (8,4) {p};
%    \node[above] at (7,3) {q};
%    \node[above] at (8,3) {q};
%
%    \draw[dotted, shorten >=10pt,shorten <=10pt] (7,2)--(8,1);
%    \draw[dotted, shorten >=10pt,shorten <=10pt] (9,4)--(10,3);
%    \draw[dotted, shorten >=10pt,shorten <=10pt] (11,6)--(12,5);
%
%    \node[below] at (11,0) {$n$};
%    \node[left] at (0,6) {$i$};
%    \end{tikzpicture}
%    \caption{Induction step \label{figure}}
%\end{figure}

\begin{lemma}
    Let $\{\p(n, i)\}_{n, i}$ and $\{\q(n, i)\}_{n, i}$ each be one of the following two families of propositions:
    \[ \label{proposition: uniqueness: eq1}
    \big\{ \triangle_i (\id_{[n]}) = \Delta_ i(\id_{[n]}) \big\}_{n, i} \qquad
    \big\{ \triangle_i (\id_{[n]}) = T \Delta_ i(\id_{[n]}) \big\}_{n, i.}
    \]
    Then, for $i < n-1$ the following implication holds:
    \[ \label{proposition: uniqueness: eq2}
    \boxed{\p(n,i+1) \wedge \p(n-1,i+1) \wedge \q(n-1,i)}\ \Longrightarrow\ \boxed{\q(n,i)}
    \]
\end{lemma}

%\begin{proof}
%        Let $\big\{ \p(n,i) \big\}_{n,i \in \Z}$ and $\big\{ \q(n,i) \big\}_{n,i \in \Z}$ both be the family $\big\{ (\Delta_i)_n = \nabla^{n \choose n-i} \big\}_{n,i \in \Z.}$ The other options are treated analogously.
%
%		From $\p(n, i+1)$ and $\p(n-1, i+1)$ we have
%		\[
%		\partial \triangle_{i+1}(\id_{[n]}) + \triangle_{i+1} (\partial \id_{[n]}) = \partial \Delta_{i+1}(\id_{[n]}) + \Delta_{i+1} (\partial \id_{[n]})
%		\]
%		or, equivalently,
%		\[
%		(1+T) \triangle_{i} (\id_{[n]}) = (1+T) \Delta_{i} (\id_{[n]})
%		\]
%		Lemma \ref{lemma: extra lemma} implies
%		\begin{equation} \label{proposition: uniqueness: induction step: eq1}
%		\Delta_{i}(\id_{[n]}) \ = \!\!\! \sum_{U \in \P_{n-i}(n)} d_{U^{\xi}}(\id_{[n]}) \tensor d_{U^{{\bar{\xi}}}}(\id_{[n]})
%		\end{equation}
%		for some ${\bar{\xi}},\, \xi: \P_{n-i}(n)\to\{+,-\}$. To finish the proof we need to show $\xi(U) = -$ and ${\bar{\xi}}(U) = +$
%		for each $U \in \P_{n-i}(n)$.  Therefore,
%		\[
%		(1+T)\,(\Delta_{i-1})_n =
%		\partial_{n+i}\, (\Delta_{i})_n\ +\ (\Delta_{i})_{n-1}\, \partial_{n}
%		\]
%\end{proof}

%\begin{proposition} \label{proposition: uniqueness}
%    Steenrod's cup-$i$ construction is isomorphic to $\triangle$.
%\end{proposition}
%
%\begin{proof}
%    content...
%\end{proof}