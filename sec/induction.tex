\subsection{Induction step}

\begin{lemma} \label{l:boundary triangle}
	If for some $i < n$ there exists $\xi \colon \P_{n-i}(n) \to \F$ such that
	\[
	\triangle_i [n]\ =\! \sum_{\P_{n-i}(n)} {U^{\xi}} \ot {U^{\barxi}}
	\]
	then
	\begin{align*}
	\label{equation: boundary of triangle}
	\bd \triangle_i [n]\ = &
	\sum_{\P_{n-i}(n)} \left( \, \sum_{u \in U^\barxi} u.U^\xi \ot U^\barxi \ +
	\sum_{u \in U^\xi} U^\xi \ot u.U^\barxi \right) \\ + &
	\sum_{\P_{n-i}(n)} \sum_{x \notin U} \left( x.U^\xi \ot U^\barxi \ +\ U^\xi \ot {x.U^\barxi} \right).
	\end{align*}
\end{lemma}

\begin{lemma} \label{l:condition to be in the kernel of sxs}
	Let $j \in \{0, \dots, n-1\}$.
	An element $V \otimes W \in \cP[n] \otimes \cP[n]$ is in the kernel of $(\cP \sigma_j \ot \cP \sigma_j)$ if an only if both $j$ and $j+1$ are missing from either $V$ or $W$.
\end{lemma}

\begin{proof}
	TBW
\end{proof}

\begin{lemma}
	If for some $i \leq n-2$ there exists $\xi \colon \P_{n-i}(n) \to \F$ such that
	\[
	\triangle_i [n]\ =\! \sum_{\P_{n-i}(n)} {U^{\xi}} \ot {U^{\barxi}}
	\]
	and either $\triangle_i [n-1]$ is equal to $\Delta_i [n-1]$ or $\Delta_i [n-1]$, then $\xi$ is constant, i.e., $\triangle_i [n] = \Delta_i [n]$ or $\triangle_i [n] = T \Delta_i [n]$.
\end{lemma}

\begin{proof}
	Let us assume $\triangle_i [n-1] = \Delta_i [n-1] = \sum_{\P_{n-1-i}(n-1)} U^0 \ot U^1$.
	The other case is done analogously.

	By \cref{l:boundary triangle} we have
    \begin{align*}
    \partial \triangle_i [n]\ & =\! \sum_{\P_{n-i}(n)} \left(\sum_{u \notin U} {u.U^\xi} \ot {U^\barxi} +
    \sum_{u \notin U} {U^\xi} \ot {u.U^\barxi} \right) \\ & +\!
    \sum_{\P_{n-i}(n)} \left(\sum_{u \in U^\barxi} {u.U^\xi} \ot {U^\barxi} +
    \sum_{u \in U^{\xi}} {U^\xi} \ot {u.U^\barxi} \right).
    \end{align*}
    Adding $\triangle_i \bd \, [n]$ to it and using identity \eqref{e:triangle of boundary} we deduce that $(1+T) \triangle_{i-1} [n] = \bd \triangle_i [n] + \triangle_i \bd \, [n]$ is equal to
    \begin{align}
    \label{e:top} & \sum_{\P_{n-i}(n)} \left(\sum_{u \notin U} u.U^\xi \ot U^\barxi +
    \sum_{u \notin U} U^\xi \ot u.U^\barxi \right) \\ +\
    \label{e:bottom} (1+T) & \sum_{\substack{\P_{n-i}(n) \\ \xi(U) \neq 0}} \left(\sum_{u \in U^1} {u.U^0} \ot {U^1} +
    \sum_{u \in U^0} {U^0} \ot {u.U^1} \right).
    \end{align}
    We will use Lemma \ref{l:condition to be in the kernel of sxs} to show that \eqref{e:top}, the first sum above, is in the kernel $K^{(j)}$ of $\cP \sigma_j \ot \cP \sigma_j$ for every $j \in \{0, \dots, n-1\}$.
    Consider $U \in \P_{n-i}(n)$.
    If $\{j, j+1\} \cap U = \emptyset$ then $u.U^\xi \ot U^\barxi \in K^{(j)}$ and $U^\xi \ot u.U^\barxi \in K^{(j)}$ for every $u \notin U$.
    If $\{j, j+1\} \cap U = \{j, j+1\}$ then the same conclusion follows from the fact that $\ind_U(j) = \ind_U(j+1)$.
    If $\{j, j+1\} \cap U = \{j\}$ then $u.U^\xi \ot U^\barxi \in K^{(j)}$ and $U^\xi \ot u.U^\barxi \in K^{(j)}$ for every $u \notin U$ with $u \neq j+1$.
    Furthermore, if $j \in U^\xi$ then $(j+1).U^\xi \ot U^\barxi \in K^{(j)}$ and $U^\xi \ot (j+1).U^\barxi \notin K^{(j)}$.
    An analogous statement holds if $j \in U^\barxi$ and a similar analysis applies to the case $\{j, j+1\} \cap U = \{j+1\}$.
    We will show that the basis elements in \eqref{e:top} that are not in $K^{(j)}$ cancel in pairs after the application of $\cP\sigma_j \ot \cP\sigma_j$.
    Let $\Gamma_j^\xi \subseteq \P_{n-i}(n)$ consists of elements $U$ with $\{j, j+1\} \cap U = \{j\}$ and $j \in U^\xi$.
    Let $\Gamma_{j+1}^\xi$, $\Gamma_j^\barxi$, and $\Gamma_{j+1}^\barxi$ be defined analogously.
    The claim follows from the existence of the bijections $\Gamma_{j}^\xi \cong \Gamma_{j+1}^\xi$ and $\Gamma_{j}^\barxi \cong \Gamma_{j+1}^\barxi$ defined by the assignments $U \mapsto (j+1).\big( U \setminus \{j\} \big)$ and $U \mapsto j.\big( U \setminus \{j+1\} \big)$, since two summands related by one of these bijections are sent to the same element by $\cP\sigma_j \ot \cP\sigma_j$.

	We will impose conditions on \eqref{e:bottom} following an analysis similar to the one just given.
	Notice that $(1+T)$ commutes with $\cP\sigma_j \otimes \cP\sigma_j$ and that the only basis elements in \eqref{e:bottom} not in $K^{(j)}$ are associated to pairs $(U, u)$ with $\xi(U) \neq 0$ and $u = j$ or $ u = j+1$.
	Let $\Lambda_{j}^0$ be the subset of $\{U \in \P_{n-i}(n) \mid \xi(U) \neq 0\}$ consisting of sets $U$ with $j \in U^0$ and $j+1 \notin U$.
	We define $\Lambda_{j}^1$, $\Lambda_{j+1}^0$, and $\Lambda_{j+1}^1$ analogously.
	The set $\Lambda_{j \wedge j+1}$ is defined by the conditions $\xi(U) \neq 0$ and $j,j+1 \in U$.

	Observe that the sum
	\[
	\sum_{\Lambda_{j \wedge j+1}} \left( \sum_{u \in U^1} {u.U^0} \ot {U^1} +
	\sum_{u \in U^0} {U^0} \ot {u.U^1} \right)
	\]
	is in $K^{(j)}$ since, given that $\ind_U(j) = \ind_U(j+1)$, the only non-zero summands associated to $(U,j)$ and $(U,j+1)$ cancel each other.
	Therefore, applying $\cP\sigma_j \otimes \cP\sigma_j$ to
	\[
	\sum_{\substack{\P_{n-i}(n) \\ \xi(U) \neq 0}} \left(\sum_{u \in U^1} {u.U^0} \ot {U^1} +
	\sum_{u \in U^0} {U^0} \ot {u.U^1} \right)
	\]
	yields
	\begin{align*}
	&\sum_{\Lambda_{j}^0} U^0 \setminus \{j\} \ot {U^1} \ + \
	\sum_{\Lambda_{j}^1} U^0  \ot {U^1} \setminus \{j\} \\ + \
	&\sum_{\Lambda_{j+1}^0} U^0 \setminus \{j+1\} \ot {U^1} \ + \
	\sum_{\Lambda_{j+1}^1} U^0 \ot {U^1} \setminus \{j+1\}.
	\end{align*}
	which must be in the kernel of $(1+T)$.
	This implies the existence of an involution $\phi^{(j)}$ of $\Lambda = \Lambda^0_{j} \sqcup \Lambda^1_{j} \sqcup \Lambda^1_{j+1} \sqcup \Lambda^1_{j+1}$ defined by a choice of canceling pairs.
	By the freeness of $\Delta$ and since $i \leq n-2$, this involution has no fixed points.
	It follows that two elements $U$ and $U^\prime$ with $U \cap \{j, j+1\} = U^\prime \cap \{j, j+1\}$ cannot be related by $\phi^{(j)}$ since then $U = U^\prime$.
	Therefore, $\phi^{(j)}(U) = U^\prime$ if and only if $U^\prime = j.(U \setminus \{j+1\})$ or $U^\prime = (j+1).\big( U \setminus \{j+1\} \big)$.
	Recall that by definition $\xi(U) = \xi(U^\prime) \neq 0$.
	This analysis applies to any $j \in \{0, \dots, n\}$ and we introduce a relation in $\P_{n-i}(n)$ writing $U \sim U^\prime$ if $U^\prime = j.(U \setminus \{j+1\})$ or $U^\prime = (j+1).\big( U \setminus \{j+1\} \big)$ for some $j$.
	By the previous analysis, if $U \sim U^\prime$ then $\xi(U) = \xi(U^\prime)$.
	The lemma now follows from observing that any two elements $V$ and $W$ in $\P_{n-i}(n)$ are related by a sequence
	\[
	V \sim \dots \sim W,
	\]
	so $\xi \colon \P_{n-i}(n) \to \F$ must be constant.
\end{proof}


\anibal{Lemma}

Furthermore, if $\triangle$ is irreducible then
\begin{alignat*}{2}
&\boxed{\triangle_i [n-1] = \Delta_i [n-1] \hspace*{7pt}}\ &\Longrightarrow\
&\boxed{\triangle_i [n] = \Delta_i [n] \hspace*{7pt}} \\
&\boxed{\triangle_i [n-1] = T \Delta_i [n-1]}\ &\Longrightarrow\
&\boxed{\triangle_i [n] = T \Delta_i [n]}
\end{alignat*}

------------

\anibal{proof}

The last step, proving that $\xi \equiv 0$, will be proven by contradiction.
If not then $\xi \equiv 1$, which is equivalent to $\triangle_i [n] = T \Delta_i$.
By assumption $\triangle_i [n-1] = \Delta_i [n-1]$ so
\begin{align*}
(1+T) \triangle_{i-1}[n] &=
\bd T \Delta_i [n] + \Delta_i \bd \, [n] \\ &=
T \bd \Delta_i [n] + \bd \Delta_i [n] + \bd \Delta_i [n] + \Delta_i \bd \, [n] \\ &=
(1+T) \bd \Delta_i [n] + (1+T) \Delta_{i-1} [n] \\ &=
(1+T) \Delta_i \bd \, [n] + (1+T) \Delta_{i-1} [n]
\end{align*}
Using \eqref{e:triangle of boundary} and the definition of $\Delta_{i-1} [n]$ we have
\begin{align*}
(1+T) \triangle_{i-1}[n] &=
(1+T) \sum_{\P_{n-i}(n)} \left( \,
\sum_{u \in U^1} {u.U^0} \ot {U^1} \ +
\sum_{u \in U^0} {U^0} \ot {u.U^1} \right) \\ &+
(1+T) \sum_{\P_{n-i+1}(n)} U^0 \ot U^1.
\end{align*}
By \cref{l:freeness recasted} there exists functions $\eta \colon \P_{n-i}(n) \to \F$ and $\zeta \colon \P_{n-i+1}(n) \to \F$ such that
\begin{align*}
\triangle_{i-1}[n] \ &=
\sum_{\P_{n-i}(n)} \left( \,
\sum_{u \in U^\bareta} u.U^\eta \ot U^\bareta \ +
\sum_{u \in U^\eta} U^\eta \ot u.U^\bareta \right) \\ &+
\sum_{\P_{n-i+1}(n)} U^\zeta \ot U^\barzeta.
\end{align*}
\begin{align*}
\label{equation: boundary of triangle}
\bd \triangle_i [n]\ = &
\sum_{\P_{n-i}(n)} \left( \, \sum_{u \in U^\barxi} u.U^\xi \ot U^\barxi \ +
\sum_{u \in U^\xi} U^\xi \ot u.U^\barxi \right) \\ + &
\sum_{\P_{n-i}(n)} \sum_{x \notin U} \left( x.U^\xi \ot U^\barxi \ +\ U^\xi \ot {x.U^\barxi} \right).
\end{align*}



\newpage


%\begin{figure}
%    \centering
%    \begin{tikzpicture}[scale = .65]
%    \draw (0,0)--(0,6);
%    \draw (0,0)--(11,0);
%    \draw (0,0)--(6,6);
%    \draw (1,0)--(7,6);
%    \draw (3,0)--(9,6);
%    \draw (4,0)--(10,6);
%    \draw[dashed] (5,0)--(11,6);
%
%    \draw[dotted, shorten >=10pt,shorten <=10pt] (3,2)--(4,1);
%    \draw[dotted, shorten >=10pt,shorten <=10pt] (5,4)--(6,3);
%    \draw[dotted, shorten >=10pt,shorten <=10pt] (7,6)--(8,5);
%
%    \fill (7,4) circle (0.1);
%    \fill (8,4) circle (0.1);
%    \fill (7,3) circle (0.1);
%    \fill[white] (8,3) circle (0.1);
%    \draw (8,3) circle (0.1);
%
%    \node[above] at (7,4) {p};
%    \node[above] at (8,4) {p};
%    \node[above] at (7,3) {q};
%    \node[above] at (8,3) {q};
%
%    \draw[dotted, shorten >=10pt,shorten <=10pt] (7,2)--(8,1);
%    \draw[dotted, shorten >=10pt,shorten <=10pt] (9,4)--(10,3);
%    \draw[dotted, shorten >=10pt,shorten <=10pt] (11,6)--(12,5);
%
%    \node[below] at (11,0) {$n$};
%    \node[left] at (0,6) {$i$};
%    \end{tikzpicture}
%    \caption{Induction step \label{figure}}
%\end{figure}

\begin{lemma}
    Let $\{\p(n, i)\}_{n, i}$ and $\{\q(n, i)\}_{n, i}$ each be one of the following two families of propositions:
    \[ \label{proposition: uniqueness: eq1}
    \big\{ \triangle_i  [n] = \Delta_ i [n] \big\}_{n, i} \qquad
    \big\{ \triangle_i  [n] = T \Delta_ i [n] \big\}_{n, i.}
    \]
    Then, for $i < n-1$ the following implication holds:
    \[ \label{proposition: uniqueness: eq2}
    \boxed{\p(n,i+1) \wedge \p(n-1,i+1) \wedge \q(n-1,i)}\ \Longrightarrow\ \boxed{\q(n,i)}
    \]
\end{lemma}

\begin{proof}
        Let $\big\{ \p(n,i) \big\}_{n,i \in \Z}$ and $\big\{ \q(n,i) \big\}_{n,i \in \Z}$ both be the family $\big\{ (\Delta_i)_n = \nabla^{n \choose n-i} \big\}_{n,i \in \Z.}$ The other options are treated analogously.

		From $\p(n, i+1)$ and $\p(n-1, i+1)$ we have
		\[
		\partial \triangle_{i+1} [n] + \triangle_{i+1} (\partial \id_{[n]}) = \partial \Delta_{i+1} [n] + \Delta_{i+1} (\partial \id_{[n]})
		\]
		or, equivalently,
		\[
		(1+T) \triangle_{i}  [n] = (1+T) \Delta_{i}  [n]
		\]
		Lemma \ref{lemma: extra lemma} implies
		\begin{equation} \label{proposition: uniqueness: induction step: eq1}
		\Delta_{i} [n] \ = \!\!\! \sum_{U \in \P_{n-i}(n)} d_{U^{\xi}} [n] \tensor d_{U^{\barxi}} [n]
		\end{equation}
		for some $\barxi,\, \xi: \P_{n-i}(n)\to\{+,-\}$.
To finish the proof we need to show $\xi(U) = -$ and $\barxi(U) = +$
		for each $U \in \P_{n-i}(n)$.
 Therefore,
		\[
		(1+T)\,(\Delta_{i-1})_n =
		\partial_{n+i}\, (\Delta_{i})_n\ +\ (\Delta_{i})_{n-1}\, \partial_{n}
		\]
\end{proof}

\begin{proposition} \label{proposition: uniqueness}
    Steenrod's cup-$i$ construction is isomorphic to $\triangle$.
\end{proposition}

\begin{proof}
    content...
\end{proof}