% !TEX root = ../axiomatic.tex

\section{The \pdfEinfty\ perspective}\label{s:operads}

In this section, independent of the main results of this work, we assume familiarity with the theory of operads over the category of chain complexes, in particular, of $E_\infty$-operads and their algebras and coalgebras.
The cup-$i$ products axiomatized here have been extended to a full $E_\infty$-algebra on simplicial cochains in three explicit ways \cite{mcclure2003multivariable,berger2004combinatorial,medina2020prop1}.
The importance of $E_\infty$-algebras is highlighted by a result of Mandell \cite{mandell2006homotopy_type} stating that under mild finiteness hypothesis two spaces are weakly homotopy equivalent if and only if their cochains are quasi-isomorphic as $E_\infty$-algebras over the integers.

Let us denote by $\chains$ the functor of \textit{integral} simplicial chains.
Recall that the \textbf{Eilenberg--Zilber operad} is the operad of natural linear transformations $\cZ = \set{\Hom(\chains, \chains^{\ot r})}_{r\geq1}$, whose homology is freely generated by the appropriate composition of the Alexander--Whitney diagonal with itself.
Recall that an $E_\infty$-operad $\cO$ is an operad with the same homology as $\cZ$ whose symmetric action on each arity is free.
A natural $E_\infty$-coalgebra on simplicial chains defined using $\cO$ is equivalent to an operad morphism from $\cO$ to $\cZ$ inducing an isomorphism in homology.

%We discuss Steenrod's simplicial \mbox{cup-$i$} construction as generalized by the $E_\infty$-structures of McClure--Smith \cite{mcclure2003multivariable}, Berger--Fresse \cite{berger2004combinatorial}, and the author \cite{medina2020prop1}.

\subsection{Diagonal and join}

As can be seen clearly from \cref{e:prop cup-i}, the Alexander--Whitney diagonal and the join of simplices generate the original formulas of Steenrod.
In fact, they generate a full $E_\infty$-coalgebra structure.
More precisely, let the \textbf{algebraic join product} $\ast \colon \chains^{\ot 2}(\simplex^n) \to \chains(\simplex^n)$ be defined (over the integers) by
\begin{multline}
	\ast \big(\left[v_0, \dots, v_p \right] \ot \left[v_{p+1}, \dots, v_q\right]\big) =\\
	\begin{cases}
		(-1)^p \sign \pi \left[v_{\pi(0)}, \dots, v_{\pi(q)}\right] &
		\text{ if } v_i \neq v_j \text{ for } i \neq j, \\
		\hfil 0 &
		\text{ if not},
	\end{cases}
\end{multline}
where $\pi$ is the permutation that orders the vertices.
As proven in \cite{medina2020prop1}, the suboperad of $\cZ$ resulting from arbitrary compositions of the Alexander--Whitney diagonal, the join product and the permutation of factors is the image of an $E_\infty$-operad $\forget(\Med)$ defining an $E_\infty$-coalgebra on simplicial chains.
We remark that this structure is induced from an analogous one in the cellular context \cite{medina2021prop2}.
%, more specifically, a natural coalgebra structure over certain operad $\forget(\Med)$.

%This can be extended to a full $E_\infty$-coalgebra on chains as follows.
%An \textbf{$\Med$-bialgebra} is a chain complex $B$ with three operations
%\begin{equation}\label{e:generators}
%	\Delta \colon B \to B \ot B, \qquad
%	\varepsilon \colon B \to \Ftwo, \qquad
%	\ast \colon B \ot B \to B,
%\end{equation}
%such that the first two define the structure of a counital coalgebra on $B$, and the third one satisfies
%\begin{gather*}
%\varepsilon \circ \ast = 0, \\
%\bd \circ \ast + \ast \circ (\bd \ot \, \id) + \ast \circ (\id \ot \bd) =
%\varepsilon \ot \id + \id \ot \varepsilon.
%\end{gather*}
%As proven in \cite{medina2020prop1}, any $\Med$-bialgebra is an $E_\infty$-coalgebra.
%More explicitly, the associated $E_\infty$-structure consists of all maps of the form $B \to B^{\ot r}$ for $r \geq 0$ resulting from the composition of generators in \cref{e:generators} and permutations of tensor factors.
%The $E_\infty$-operad controlling this structure is denoted $\forget(\Med)$.

%The complex of normalized chains of a standard simplicial set is naturally an $\Med$-bialgebra with the Alexander--Whitney coproduct $\Delta$, the augmentation map $\varepsilon$, and the \textbf{join product} $\ast \colon \chains^{\ot 2}(\simplex^n) \to \chains(\simplex^n)$ defined by
%\[
%\ast \big(\left[v_0, \dots, v_p \right] \ot \left[v_{p+1}, \dots, v_q\right]\big) =
%\begin{cases} \left[v_{\pi(0)}, \dots, v_{\pi(q)}\right] & \text{ if } v_i \neq v_j \text{ for } i \neq j, \\
%\hfil 0 & \text{ if not}, \end{cases}
%\]
%where $\pi$ is the permutation that orders the vertices.
%We remark that this structure is induced from an analogue one in the cellular context \cite{medina2021prop2}.
%The natural $\forget(\Med)$-coalgebra structure on $\chains(\simplex^n)$ extends to $\chains(X)$ for every simplicial set $X$.

\subsection{Surjections}

We introduce the following notation
\begin{align*}
	\Delta^1 &= \Delta, &
	\ast^1 &= \ast, \\
	\Delta^{m+1} &= (\Delta^m \ot \, \id) \circ \Delta, &
	\ast^{m+1} &= \ast \circ (\ast^m \ot \, \id).
\end{align*}
Since the Alexander--Whitney coproduct and the join product are coassociative and associative respectively, the above choices are not essential.

Consider an arbitrary surjection $s \colon \set{1,\dots,\ell} \to \set{1,\dots,r}$ which we represent by its order image $\big(s(1),\dots,s(\ell)\big)$.
We define a natural linear transformation associated to $s$ by
\begin{equation}\label{e:surjection action}
	\Big( \ast^{\bars{s^{-1}(1)}} \ot \dotsb \ot \ast^{\bars{s^{-1}(r)}} \Big) \circ \pi_s \circ \Delta^{\ell-1}
\end{equation}
where $\ast^0 = \id$ and $\pi_s$ is the shuffle permutation defined by
\[
\big( \pi s(1), \dots, \pi s(\ell) \big) =
\big( 1, \dots, 1, \dots, r, \dots, r \big).
\]
This assignment defines the \textbf{surjection operad} $\cX$ of McClure--Smith \cite{mcclure2003multivariable} as the image of $\UM$ in $\cZ$ \cite[Appendix 1]{medina2020prop1}.
Passing to $\Ftwo$-coefficients, we can inspect \cref{e:prop cup-i} to conclude that the surjections
\[
\big\{ (1,2), (1,2,1), (1,2,1,2), \dots \big\}
\]
define Steenrod's original simplicial \mbox{cup-$i$} construction.
In fact, $\cX(2)$ is isomorphic to $W$ so its set of bases is in bijection with the set of a simplicial \mbox{cup-$i$} constructions satisfying our axioms.

\subsection{Barratt--Eccles}

Let $G$ be a finite group, for us it will be a symmetric group, and consider the simplicial set
\[
\begin{split}
	EG_m &= \big\{ (g_0, \dots, g_m) \mid g_i \in G \big\}, \\
	\face_j(g_0, \dots, g_m) &= (g_0, \dots, \widehat g_j, \dots, g_m), \\
	\dege_j(g_0, \dots, g_m) &= (g_0, \dots, g_j, g_j, \dots, g_m).
\end{split}
\]
The partial composition of permutations
\[
\circ_j \colon \Sym_p \times \Sym_q \to \Sym_{p+q-1}
\]
induces an operad structure on the collection of chain complexes $\cE = \set{\chains(E \Sym_r)}_{r\geq1}$.
This $E_\infty$-operad, studied by Berger--Fresse \cite{berger2004combinatorial}, is referred to as the \textbf{Barratt--Eccles operad}.
%These authors also define a quasi-isomorphism of operads $\mathrm{TR} \colon \cE \to \cX$, where $\cX$ is given a different sign convention, and use it to define a natural .
These authors also defined a natural $\cE$-coalgebra on simplicial chains $\cE \to \cZ$.
%contained in the image of $\UM$.
%\anibal{Is this true integrally?}
Furthermore, passing to $\Ftwo$-coefficients, their structure factors through the McClure--Smith structure $\cE \xra{\TR} \cX \to \cZ$, with $\TR$ an isomorphism in arity~2.
%\[
%\mathrm{TR}(\id, T, \id, \dots, T^i) =
%\begin{cases}
%(1,2,\dots,1,2) & \text{ if } i \text{ is even}, \\
%(1,2,\dots,2,1) & \text{ if } i \text{ is odd}.
%\end{cases}
%\]
This can be used to see that in Berger--Fresse's formalism the set of elements $\big\{ (\id, T, \id, \dots, T^i) \big\}_{i \in \N}$ defines a simplicial \mbox{cup-$i$} construction that agrees with Steenrod's, and that the set of bases of $\cE(2)$ is in bijection with the set of a simplicial \mbox{cup-$i$} constructions satisfying our axioms.