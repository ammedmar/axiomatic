%\subsection{}
%
%For the rest of the section we assume $\triangle$ is a free non-degenerate and irreducible cup-$i$ construction.
%Most of the lemmas below hold with fewer assumptions but we have no use for these.

\begin{lemma} \label{l:special cases i=n and i>n}
	Let $\triangle$ be free and non-degenerate cup-$i$ construction.
	\begin{enumerate}
		\item $\forall i,n \in \N$, $\triangle_i[n] = 0$ if $i > n$.
		\item $\forall n \in \N$, $\triangle_n[n] = \emptyset \ot \emptyset$.
	\end{enumerate}
\end{lemma}

\begin{proof}
	The chain complex $\cP(\simplex^n) \ot \cP(\simplex^n)$ is $0$ in degrees greater than $2n$ and it is generated by $\emptyset \ot \emptyset$ in degree $2n$.

	The first claim is immediate since $\triangle_i[n]$ is in degree $n+i > 2n$ if $i > n$.

	Let us consider the second claim.
	If the desired conclusion does not hold, there exists $n \in \N$ smallest such that $\triangle_n [n] = 0$.
	If $n > 0$ then
	\begin{align*}
	(1+T) \triangle_{n-1} [n] =
	\bd \triangle_{n} [n] + \triangle_{n} \bd \, [n] = 0,
	\end{align*}
	and Lemma \ref{l:(1+T) triangle = 0 implies triangle = 0} implies $\triangle_{n-1} [n] = 0$.
	From this and the assumption
	\[
	\triangle_{n-1}[n-1] = \emptyset \ot \emptyset
	\]
	we obtain
	\begin{equation}
	\begin{split}
	(1+T)\triangle_{n-2} [n] =
	\bd \triangle_{n-1} [n] + \triangle_{n-1} \bd \, [n] =
	\sum_{u = 0}^n \{u\} \ot \{u\},
	\end{split}
	\end{equation}
		which is a contradiction since $\sum_u \{u\} \ot \{u\}$ is not in the image of $(1+T)$.

	The previous argument shows that $\triangle_n [n] = 0$ for every $n \in \N$.
	This serves as the base case of an induction argument in $n-i$ proving that $\triangle_i [n] = 0$ for every $i, n \in \N$.
	For the induction step, consider
	\begin{align*}
	(1+T) \triangle_{i-1} [n] =
	\bd \triangle_{i} [n] + \triangle_{i} \bd\, [n] = 0,
	\end{align*}
	which, by Lemma \ref{l:(1+T) triangle = 0 implies triangle = 0}, implies $\triangle_{i-1} [n] = 0$.
\end{proof}

\begin{lemma} \label{l:condition to be in the kernel of sxs}
	Consider a codegeneracy $\sigma_j \colon [n] \to [n-1]$.
	A basis element $V \ot W \in \cP(\simplex^n) \ot \cP(\simplex^n)$ is in the kernel of $\cP(\sigma_j) \ot \cP(\sigma_j)$ if an only if both $j$ and $j+1$ are missing from either $V$ or $W$.
\end{lemma}

%\begin{proof}
%	This is a direct consequence of \cref{e:action of P on a codegeneracy}.
%\end{proof}

\begin{lemma}
	Let $\triangle$ be free and non-degenerate cup-$i$ construction.
	For all integer $n \geq 1$ either $\triangle_{n-1} [n]$ or $T \triangle_{n-1} [n]$ is equal to
	\[
	\sum_{\substack{u \in \{0,\dots,n\} \\ u \ \mathrm{odd}}} \{u\} \ot \emptyset
	\quad +
	\sum_{\substack{u \in \{0,\dots,n\} \\ u \ \mathrm{even}}} \emptyset \ot \{u\}.
	\]
\end{lemma}

%\begin{proof}
%	By \cref{l:special cases i=n and i>n} we have $\triangle_{n} [n] = \emptyset \ot \emptyset$ and $\triangle_{n} \bd \, [n] = 0$ for all $n \in \N$.
%	Therefore,
%	\begin{align*}
%	(1+T) \triangle_{n-1} [n] &=
%	(\bd \ot \, \id + \id \ot \bd) (\emptyset \ot \emptyset) \\ &=
%	(1+T) \sum_{u=0}^n \{u\} \ot \emptyset
%	\end{align*}
%	and we need to show that the partition of summands provided by \cref{l:(1+T) triangle = 0 implies triangle = 0} is determined by the parity of the associated integer.
%	Let us argue by contradiction assuming some $j$ and $j+1$ belong to the same subset in the partition.
%	With no loss of generality we have
%	\[
%	\triangle_{n-1} [n] = \big( \{j\} + \{j+1\} \big) \ot \emptyset + O(j, j+1)
%	\]
%	where $O(j, j+1)$ is a sum of basis elements missing $\{j\}$ and $\{j+1\}$ from both of its tensor factors.
%	Since $\triangle_{n-1} \bd \, [n] = \sum_{u=0}^{n} \{u\} \ot \{u\}$,
%	\[
%	(1+T) \triangle_{n-2} [n] = (1+T) \big( \{j\} \ot \{j+1\} \big) + P(j, j+1)
%	\]
%	where $P(j, j+1)$ is a sum of basis elements with $j$ and $j+1$ missing from at least one of its tensor factors.
%
%	By \cref{l:condition to be in the kernel of sxs} every basis element in $P(j,j+1)$ is in the kernel of $\cP(\sigma_j) \ot \cP(\sigma_j)$.
%	Using \cref{l:(1+T) triangle = 0 implies triangle = 0} in the above equation implies that $\triangle_{n-2} [n]$, an element in $\ker \big( \cP(\sigma_j) \ot \cP(\sigma_j) \big)$, is equal to either $\big( \{j\} \ot \{j+1\} \big)$ or $\big( \{j+1\} \ot \{j\} \big)$ plus an element in this kernel.
%	This is a contradiction since neither of these two basis elements is in $\ker \big( \cP(\sigma_j) \ot \cP(\sigma_j) \big)$ by \cref{l:condition to be in the kernel of sxs}.
%\end{proof}