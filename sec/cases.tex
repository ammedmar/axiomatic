\subsection{}

For the rest of the section we assume $\triangle$ is a free non-degenerate and irreducible cup-$i$ construction.
Most of the lemmas below hold with fewer assumptions but we have no use for these.

\begin{lemma}
	In $\cP[n] \ot \cP[n]$,	$\triangle_n[n] = \emptyset \ot \emptyset$ and $\triangle_i [n] = 0$ if $i > n$.
\end{lemma}

\begin{proof}
	The chain complex $\cP[n] \ot \cP[n]$ is $0$ in degrees greater than $2n$ and it is generated by $\emptyset \ot \emptyset$ in degree $2n$.
	If $i > 0$ the claim is immediate.
	If the desired conclusion does not hold, there exists $n \in \N$ smallest such that $\triangle_n  [n] = 0$.
	If $n > 0$ then
    \begin{align*}
    (1+T) \triangle_{n-1} [n] =
    \bd \triangle_{n} [n] + \triangle_{n} \bd \, [n] = 0,
    \end{align*}
    and Lemma \ref{l:(1+T) triangle = 0 implies triangle = 0} implies $\triangle_{n-1} [n] = 0$.
	From this and the assumption
    \[
    \triangle_{n-1} ({[n-1]}) = \emptyset \ot \emptyset
    \]
    we obtain
    \begin{equation}
    \begin{split}
    (1+T)\triangle_{n-2} [n] =
    \bd \triangle_{n-1} [n] + \triangle_{n-1} \bd \, [n] =
    \sum_{u = 0}^n \{u\} \ot \{u\},
    \end{split}
    \end{equation}
 	which is a contradiction since $\sum_u \{u\} \ot \{u\}$ is not in the image of $(1+T)$.

    The previous argument shows that $\triangle_n [n] = 0$ for every $n \in \N$.
	This serves as the base case of an induction argument in $n-i$ proving that $\triangle_i [n] = 0$ for every $i, n \in \N$.
	For the induction step, consider
    \begin{align*}
    (1+T) \triangle_{i-1} [n] =
    \bd \triangle_{i} [n] + \triangle_{i} \bd\, [n] = 0,
    \end{align*}
    which, by Lemma \ref{l:(1+T) triangle = 0 implies triangle = 0}, implies $\triangle_{i-1}  [n] = 0$.
\end{proof}

\anibal{Fix this lemma to be more useful on the places invoked}

\begin{lemma} \label{l:condition to be in the kernel of sxs}
	Let $j \in \{0, \dots, n-1\}$.
	An element $V \otimes W \in \cP[n] \otimes \cP[n]$ is in the kernel of $(\cP \sigma_j \ot \cP \sigma_j)$ if an only if both $j$ and $j+1$ are missing from either $V$ or $W$.
\end{lemma}

\begin{proof}
	TBW
\end{proof}

\begin{lemma}
	Let $n \geq 1$.
	In $\cP[n] \ot \cP[n]$, either $\triangle_{n-1} [n]$ or $T \triangle_{n-1} [n]$ is equal to
	\[
	\sum_{\substack{u \leq n \\ u \text{ odd }}} \{u\} \ot \emptyset \ +
	\sum_{\substack{u \leq n \\ u \text{ even }}} \emptyset \ot \{u\}.
	\]
\end{lemma}

\begin{proof}
	Since $\triangle_{n} [n] = \emptyset \ot \emptyset$ and $\triangle_{n} \bd \, [n] = 0$ we have
	\[
	(1+T) \triangle_{n-1} [n] = (1+T) \sum_{u=0}^n \{u\} \ot \emptyset
	\]
	and we need to show that the partition provided by Lemma \ref{l:(1+T) triangle = 0 implies triangle = 0} is that of even and odd non-negative integers less than $n$.
	Let us argue by contradiction and assume some $j$ and $j+1$ belong to the same subset in the partition.
	With no loss of generality we have
	\[
	\triangle_{n-1} [n] = \big( \{j\} + \{j+1\} \big) \ot \emptyset + O(j, j+1)
	\]
	where $O(j, j+1)$ is a sum of basis elements missing $\{j\}$ and $\{j+1\}$ from both of its tensor factors.
	Since $\triangle_{n-1} \bd \, [n] = \sum_{u=0}^{n} \{u\} \ot \{u\}$,
	\[
	(1+T) \triangle_{n-2} [n] = (1+T) \big( \{j\} \ot \{j+1\} \big) + P(j, j+1)
	\]
	where $P(j, j+1)$ is a sum of basis elements with $j$ and $j+1$ missing from at least one of its tensor factors.

	\anibal{Fix this last part}
	Using Lemma \ref{l:(1+T) triangle = 0 implies triangle = 0} and Lemma \ref{l:condition to be in the kernel of sxs} we get a contradiction since $\delta_j \ot \delta_{j+1}$ is not in the kernel of $\cP \sigma_j \ot \cP \sigma_j$.
\end{proof}