\subsection{Special cases}

%In this subsection we prove Theorem \ref{theorem: main reformulated} for all pairs of integers satisfying $i > n-2$.

\begin{lemma}
	If $i > n$ then $\triangle_i [n] = 0$.
\end{lemma}

%\begin{proof}
%	This is immediate since $\chains(\simplex^n)^{\otimes 2}_i = 0$ in this range.
%\end{proof}

\begin{lemma} \label{lemma: triangle_n of n = n x n}
    For any $n \in \N$
    \[
    \triangle_n [n] = \id_{[n]} \otimes \id_{[n]}.
    \]
\end{lemma}

%\begin{proof}
%	For any $n \in \N$, the vector space $\chains(\simplex^n)^{\otimes 2}_n$ is one dimensional and either $\triangle_n [n] = 0$ or $\triangle_n [n] = \id_{[n]} \otimes \id_{[n]}$. If the desired conclusion does not hold, there exists $n \in \N$ smallest such that $\triangle_n  [n] = 0$. If $n > 0$ then
%    \begin{align*}
%    (1 + T) \triangle_{n-1} [n] =
%    \partial \triangle_{n} [n] + \triangle_{n}(\partial\, \id_{[n]}) =
%    0,
%    \end{align*}
%    and Lemma \ref{lemma: (1+T) triangle = 0 implies triangle = 0} implies $\triangle_{n-1} [n] = 0$. From this and the assumption
%    \[
%    \triangle_{n-1} (\id_{[n-1]}) = \id_{[n-1]} \otimes \id_{[n-1]}
%    \]
%    we obtain
%    \begin{equation}
%    \begin{split}
%    (1+T)\triangle_{n-2} [n] =
%    \partial \triangle_{n-1} [n] + \triangle_{n-1}(\partial \id_{[n]}) =
%    \sum_{u = 0}^n \delta_u \otimes \delta_u,
%    \end{split}
%    \end{equation}
% 	which is a contradiction since $\sum \delta_u \otimes \delta_u$ is not in the image of $(1+T)$.
%
%    The previous argument shows that $\triangle_n [n] = 0$ for every $n \in \N$. This serves as the base case of an induction argument in $n-i$ proving that $\triangle_i [n] = 0$ for every $i, n \in \N$. For the induction step, consider
%    \begin{align*}
%    (1+T) \triangle_{i-1} [n] =
%    \partial \triangle_{i} [n] + \triangle_{i}(\partial\, \id_{[n]}) =
%    0,
%    \end{align*}
%    which, by Lemma \ref{lemma: (1+T) triangle = 0 implies triangle = 0}, implies $\triangle_{i-1}  [n] = 0$.
%\end{proof}

\begin{lemma}
	For $n \geq 1$, either $\triangle_{n-1} [n]$ or $T \triangle_{n-1} [n]$ is equal to
	\[
	\sum_{\substack{u \leq n \\ u \text{ odd }}} \delta_u \otimes \id_{[n]} \ + \sum_{\substack{u \leq n \\ u \text{ even }}} \id_{[n]} \otimes \delta_u.
	\]
\end{lemma}

%\begin{proof}
%	Since $\triangle_{n} [n] = \id_{[n]} \otimes \id_{[n]}$ and $\triangle_{n}(\partial\, \id_{[n]}) = 0$ we have
%	\[
%	(1+T) \triangle_{n-1} [n] = (1 + T) \sum_{k = 0}^n \delta_u \otimes \id_{[n]}
%	\]
%	and we need to show that the partition provided by Lemma \ref{lemma: (1+T) triangle = 0 implies triangle = 0} is that of even and odd non-negative integers less than $n$. Let us argue by contradiction and assume some $j$ and $j+1$ belong to the same subset in the partition. With out loss of generality we have
%	\[
%	\triangle_{n-1} [n] = (\delta_j + \delta_{j+1}) \otimes \id_{[n]} + O(j, j+1)
%	\]
%	where $O(j, j+1)$ is an element in the span of basis elements missing both $\delta_j$ and $\delta_{j+1}$ from both of its tensor factors. Since $\triangle_{n-1}(\partial\, \id_{[n-1]}) = \sum_{k=0}^n \delta_k \otimes \delta_k$,
%	\[
%	(1+T) \triangle_{n-2} [n] = (1+T)(\delta_j \otimes \delta_{j+1}) + P(j, j+1)
%	\]
%	where $P(j, j=1)$ is an element in the span of basis elements with $j$ and $j+1$ missing from at least one of its tensor factors. Using Lemma \ref{lemma: (1+T) triangle = 0 implies triangle = 0} and Lemma \ref{lemma: condition to be in the kernel of s} we get a contradiction since $\delta_j \otimes \delta_{j+1}$ is not in the kernel of $(s_j)_\ast^{\otimes 2}$.
%\end{proof}