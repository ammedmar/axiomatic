% !TEX root = ../axiomatic.tex

\section{Preliminaries}\label{s:preliminaries}

In this section we review the basic theory of simplicial sets including the construction of their normalized chains.

\subsection{Simplicial sets}

We denote the set of non-negative integers by $\N$.
The \textbf{simplex category} is defined to have an object $[n] = \{0, \dots, n\}$ for every $n \in \N$ and a morphism in $\simplex \big([m], [n] \big)$ for each order-preserving function from $[m]$ to $[n]$.
As is commonly done we denote the identity $[n] \to [n]$ simply as $[n]$.
The morphisms $\delta_i \colon [n-1] \to [n]$ and $\sigma_i \colon [n+1] \to [n]$ defined for $0 \leq i \leq n$ by
\[
\delta_i(k) =
\begin{cases} k & k < i, \\ k+1 & i \leq k, \end{cases}
\quad \text{ and } \quad
\sigma_i(k) =
\begin{cases} k & k \leq i, \\ k-1 & i < k, \end{cases}
\]
generate all morphisms in the simplex category.
They satisfy the \textbf{cosimplicial identities}:
\begin{equation}\label{e:cosimplicial identities}
\begin{split}
\delta_j \, \delta_i &=
\delta_i \, \delta_{j-1}, \hspace*{18.3pt} i < j, \\
\sigma_j \, \sigma_i &=
\sigma_i \, \sigma_{j+1}, \hspace*{15.9pt} i \leq j, \\
\sigma_j \, \delta_i &=
\begin{cases}
\delta_i \, \sigma_{j-1}, & i < j, \\
\hfil \id, & i = j,\, j+1, \\
\delta_{i-1} \, \sigma_{j}, & i > j + 1,
\end{cases}
\end{split}
\end{equation}
which can be used to uniquely express any morphism in the form
\begin{equation}\label{e:canonical factorization}
\delta_{u_{p}} \cdots\, \delta_{u_1}\, \sigma_{v_1} \cdots\, \sigma_{v_q}
\end{equation}
for some sets of non-negative integers $U = \{u_1 < \cdots < u_{p}\}$ and $V = \{v_1 < \cdots < v_{q}\}$.
We simplify notation and write the \textbf{canonical factorization} \eqref{e:canonical factorization} simply as $\delta_U \sigma_V$.

A \textbf{simplicial set} $X$ is a contravariant functor from the simplex category to the category of sets and a \textbf{simplicial map} is a natural transformation between two simplicial sets.
We denote this category by $\sSet$.
As is customary we use the notation:
\[
X \big( [n] \big) = X_n, \qquad
X(\delta_i) = d_i, \qquad
X(\sigma_i) = s_i.
\]
Elements in $X_n$ are referred to as \mbox{\textbf{$n$-simplices}} with the integer $n$ called its \textbf{dimension} and omitted when unspecified.
Simplices in the image of any $s_i$ are termed \textbf{degenerate}.
We extend the notation of canonical factorizations writing $X(\delta_U \sigma_V) = s_V d_U$ with $d_\emptyset$ and $s_\emptyset$ representing the identity.
We say that a simplex $y$ is a (proper) \textbf{face} of a simplex $x$ if $y = d_U(x)$ for some (non-empty) $U$.

For each $n \in \N$, the simplicial set $\simplex^n$, referred to as the \textbf{$n^\th$ standard simplicial set}, is defined by
\[
\simplex^n_m = \simplex \big( [m], [n] \big), \qquad
d_i(\delta_U\, \sigma_V) = \delta_U \sigma_V \delta_i, \qquad
s_i(\delta_U\, \sigma_V) = \delta_U \sigma_V \sigma_i.
\]
Natural constructions on simplicial sets are controlled by their action on standard simplicial sets since for any simplicial set $X$ we have
\[
X_n \cong \colim_{\simplex^n \downarrow X} \simplex^n.
\]

\subsection{Normalized chains}

We will work with homologically graded chain complexes of $\F$-modules, regarding, in the usual way, the set of linear maps between them and their tensor product as chain complexes:
\begin{align*}
\Hom(C, C^\prime)_m &= \big\{f \mid \forall k \in \Z, f(C_k) \subseteq C^\prime_{k+m} \big\},
&\bd f &= \bd \circ f + f \circ \bd, \\
(C \ot C^\prime)_m &= \bigoplus_{p + q = m} C_p \ot C^\prime_q \,,
&\bd (c \ot c^\prime) &= \bd c \ot c^\prime + c \ot \bd c^\prime.
\end{align*}

The functor of \textbf{normalized chains} (with $\F$-coefficients) $\chains \colon \sSet \to \Ch$ is defined on objects as follows:
\[
\chains(X)_n = \frac{\F \{ X_n \}}{\F \{ s(X_{n-1}) \}}
\]
where $s(X_{n-1}) = \bigcup_{i=0}^{n-1} s_i(X_{n-1})$, and $\bd_n \colon \chains(X)_n \to \chains(X)_{n-1}$ is given by
\[
\bd_n = \sum_{i = 0}^{n} d_{i}.
\]
The tensor product functor $\chains \ot \chains$ will play an important role as well.
The functor of \textbf{normalized cochains} $\cochains$ is defined by composing $\chains$ with the linear duality functor $\Hom(-, \F)$.
Notice that in this definition $\cochains(X)$ is concentrated in non-positive degrees.
If $x$ is a $n$-simplex we abuse notation and use the same symbol $x$ to denote the associated basis element in $\chains(X)_n$ and its dual basis element in $\cochains(X)_{-n}$.