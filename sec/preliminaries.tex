% !TEX root = ../axiomatic.tex

\section{Simplicial sets and their normalized chains} \label{s:preliminaries}

We are interested in the cochains with $\F$-coefficients of spaces.
As is usually done, we represent spaces as simplicial sets and take advantage of the fact that all natural construction on simplicial sets need only be defined on the standard simplicial sets $\simplex^n$.

The \textbf{simplex category} is defined to have an object $[n] = \{0, \dots, n\}$ for every $n \in \N$ and a morphism in $\simplex \big([m], [n] \big)$ for each order-preserving function from $[m]$ to $[n]$.
As is commonly done we denote the identity $[n] \to [n]$ simply as $[n]$.
The morphisms $\delta_i \colon [n-1] \to [n]$ and $\sigma_i \colon [n+1] \to [n]$ defined for $0 \leq i \leq n$ by
\[
\delta_i(k) =
\begin{cases} k & k < i, \\ k+1 & i \leq k, \end{cases}
\quad \text{ and } \quad
\sigma_i(k) =
\begin{cases} k & k \leq i, \\ k-1 & i < k, \end{cases}
\]
generate all morphisms in the simplex category.
They satisfy the \textbf{cosimplicial identities}:
\[
\begin{split}
\delta_j \, \delta_i &= \delta_i \, \delta_{j-1} \hspace*{18.5pt} i < j \\
\sigma_j \, \sigma_i &= \sigma_i \, \sigma_{j+1} \hspace*{15.5pt} i \leq j \\
\sigma_j \, \delta_i &=
\begin{cases}
\id & i = j,\, j+1 \\
\delta_i \, \sigma_{j-1} & i < j \\
\delta_{i-1} \, \sigma_{j} & i > j + 1,
\end{cases}
\end{split}
\]
which can be used to see that any morphism in the simplex category can be uniquely written as
\begin{equation} \label{e:canonical presentation of a morphism}
\delta_{u_{p}} \cdots\, \delta_{u_1}\, \sigma_{v_1} \cdots\, \sigma_{v_q}
\end{equation}
for some sets of non-negative integers $U = \{u_1 < \cdots < u_{p}\}$ and $V = \{v_1 < \cdots < v_{q}\}$.
We simplify notation and write the \textbf{canonical presentation} \eqref{e:canonical presentation of a morphism} simply as $\delta_U\, \sigma_V$.

A \textbf{simplicial set} $X$ is a contravariant functor from the simplex category to the category of sets and a \textbf{simplicial map} is a natural transformation between two simplicial sets.
As is customary we use the notation:
\[
X \big( [n] \big) = X_n, \qquad
X(\delta_i) = d_i, \qquad
X(\sigma_i) = s_i.
\]
Elements in $X_n$ are referred to as \mbox{\textbf{$n$-simplices}} with the integer $n$ called its \textbf{dimension} and omitted when unspecified.
Simplices in the image of any $s_i$ are termed \textbf{degenerate}.
We extend the notation of canonical presentations writing $X(\delta_U \sigma_V) = s_V d_U$ with $d_\emptyset = s_\emptyset = \id$.
We say that a simplex $y$ is a (proper) \textbf{face} of a simplex $x$ if $y = d_U(x)$ for some (non-empty) $U$.


For each $n \in \N$, the simplicial set $\simplex^n$ referred to as the \textbf{$n^\th$ standard simplicial set} is defined by
\[
\simplex^n_m = \simplex \big( [m], [n] \big) \qquad
d_i(\delta_U\, \sigma_V) = \delta_U\, \sigma_V \, \delta_i \qquad
s_i(\delta_U\, \sigma_V) = \delta_U\, \sigma_V \, \sigma_i.
\]

We will work with homologically graded chain complexes of $\F$-modules, regarding, in the usual way, the set of linear maps between them and their tensor product as chain complexes:
\begin{gather*}
\Hom(C, C^\prime)_m = \big\{f \mid \forall k \in \Z, c \in C_k \Rightarrow f(c) \in C^\prime_{k+m} \big\},
\qquad
\bd f = \bd \circ f + f \circ \bd, \\
(C \ot C^\prime)_m = \bigoplus_{p + q = m} C_p \ot C^\prime_q \,,
\qquad
\bd (c \ot c^\prime) = \bd c \ot c^\prime + c \ot \bd c^\prime.
\end{gather*}

The functor of \textbf{normalized chains} (with $\F$-coefficients) $\chains \colon \sSet \to \Ch$ is defined on objects as follows:
\[
\chains(X)_n = \frac{\F \{ X_n \}}{\F \{ s(X_{n-1}) \}}
\]
where $s(X_{n-1}) = \bigcup_{i=0}^{n-1} s_i(X_{n-1})$, and $\bd_n \colon \chains(X)_n \to \chains(X)_{n-1}$ is given by
\[
\bd_n = \sum_{i = 0}^{n} d_{i}.
\]

The functor of \textbf{normalized cochains} $\cochains$ is defined by composing $\chains$ with the linear duality functor $\Hom(-, \F)$.
Notice that in this definition $\cochains(X)$ is concentrated in non-positive degrees.
If $x$ is a $n$-simplex we abuse notation and use the same symbol $x$ to denote the associated basis element in $\chains(X)_n$ and its dual basis element in $\cochains(X)_{-n}$.