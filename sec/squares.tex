% !TEX root = ../axiomatic.tex

\section{Steenrod squares} \label{s:squares}

This section is included to provide context to our results and is logically independent of the rest of this paper.

By an acyclic carrier argument \cite{eilenberg1953acyclic} all non-degenerate \mbox{cup-$i$} constructions are homotopic to each other.
Any of them gives rise to the following celebrated cohomology operations introduced by Steenrod in \cite{steenrod1947products} through an explicit \mbox{cup-$i$} construction reviewed in \cref{ss:original}.

\begin{definition*}
	For any non-degenerate \mbox{cup-$i$} construction the \textbf{$k^\th$ Steenrod square} is defined on cocycle representatives by
	\[
	\begin{tikzcd} [column sep=tiny, row sep=0]
	\Sq^k \colon &[-10pt] \rH^{-n} \arrow[r] & \rH^{-n-k} \\
	& {[\alpha]} \arrow[r, mapsto] & \big[ \alpha \cup_{n-k} \alpha \big].
	\end{tikzcd}
	\]
\end{definition*}

These operations have been axiomatically characterized \cite{steenrod1962cohomology}, and one can interpret \cref{t:main} as a continuation of this result capturing the isomorphism type, and not just the homotopy type, of Steenrod's original construction.

\begin{remark*}[Discrimination tools]
	We can illustrate the additional discriminatory power offered by these operations considering the following isomorphisms:
	\begin{enumerate}
		\item As graded vector spaces, $\rH^\vee(\R P^2; \Ftwo) \cong \rH^\vee(S^1 \wedge S^2; \Ftwo)$
		but they are distinguished by $\Sq^1$.
		\item As graded modules, $\rH^\vee(\mathbb C P^2; \Z) \cong \rH^\vee(S^2 \wedge S^4; \Z)$
		but they are distinguished by $\Sq^2$.
		\item As graded rings, $\rH^\vee(\Sigma \mathbb C P^2; \Z) \cong \rH^\vee(\Sigma (S^2 \wedge S^4); \Z)$
		but they are distinguished by $\Sq^2$.
	\end{enumerate}
\end{remark*}

\begin{remark*}[Persistent homology]
	Persistent homology is a method used on highly intensive data analysis tasks \cite{carlsson2008images, chan2013viral, lee2017quantifying} and for which various software projects exist \cite{bauer2021ripser, gudhi, medina2021giotto}.
	Based on \cite{medina2021per_st}, the project \href{https://github.com/Steenroder/steenroder}{\texttt{steenroder}} incorporates into the persistent homology pipeline the additional information Steenrod squares provide using the formulas presented in \cref{d:my cup-i}.
\end{remark*}

\begin{remark*}[Relations]
	Cup-$i$ products provide an effective construction of coboundaries coherently witnessing the commutativity relation of the cup product in cohomology at the cochain level.
	There are two notable relations satisfied by Steenrod squares;
	the first one, known as the \textit{Cartan relation}, expresses the interaction between these operations and the cup product:
	\begin{equation*}
	\Sq^k \big( [\alpha] [\beta] \big) =
	\sum_{\mathclap{i+j=k}} \, \Sq^i\big([\alpha]\big) Sq^j\big([\beta]\big),
	\end{equation*}
	whereas the second, the \textit{Adem relation} \cite{adem1952iteration}, expresses dependencies appearing through iteration:
	\begin{equation*}
	\Sq^i \Sq^j =
	\sum_{k=0}^{\lfloor i/2 \rfloor} \binom{j-k-1}{i-2k} \Sq^{i+j-k} \Sq^k.
	\end{equation*}
	Explicit cochains witnessing them were recently defined in \cite{medina2020cartan} and \cite{medina2021adem} respectively, and used in the classification of invertible fermionic topological phases \cite{kapustin2017fermionic, barkeshli2021classification}.
\end{remark*}

\begin{remark*}[Transverse intersections]
	Thom showed that every mod~$2$ homology class of a space is represented by the push-forward of the fundamental class of a closed manifold $W$ along some map to it.
	Furthermore, if the target $M$ is a closed manifold, and therefore satisfies Poincar\'{e} duality
	\[
	PD \colon H^k(M ;\Ftwo) \to H_{\bars{M}-k}(M; \Ftwo),
	\]
	the cohomology class dual to the homology class represented by the intersection of two transverse maps $V \to M$ and $W \to M$ is the cohomology class $[\alpha] [\beta]$ where $[\alpha]$ and $[\beta]$ are respectively dual to the homology classes represented by $V \to M$ and $W \to M$.
	By taking $[\alpha] = [\beta]$ of degree $-k$ we have that $\Sq^k \big( [\alpha] \big)$ is represented by the transverse self-intersection of $W \to M$.
	A comparison lifting this one between intersection and the cup-$0$ product of cochains was given in \cite{medina2021flowing}.
	A generalization of this result to \mbox{cup-$i$} products is the focus of current research \cite{medina2022foundations}
\end{remark*}