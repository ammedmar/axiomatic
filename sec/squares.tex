% !TEX root = ../axiomatic.tex

\section{Steenrod squares} \label{s:squares}

By an acyclic carrier argument \cite{eilenberg1953acyclic} all non-degenerate cup-$i$ constructions are homotopic to each other.
Any of them gives rise to the following celebrated cohomology operations introduced by Steenrod in \cite{steenrod1947products} through an explicit cup-$i$ construction, reviewed and shown to agree with the canonical one on \cref{s:others}.

\begin{definition}
	For any non-degenerate cup-$i$ construction the \textbf{$k^\th$ Steenrod square} is defined on cocycle representatives by
	\[
	\begin{tikzcd} [column sep=tiny, row sep=0]
	\Sq^k \colon &[-10pt] \rH^{-n} \arrow[r] & \rH^{-n-k} \\
	& {[\alpha]} \arrow[r, mapsto] & \big[ \alpha \cup_{n-k} \alpha \big].
	\end{tikzcd}
	\]
\end{definition}

Steenrod squares are axiomatically characterized \cite{steenrod1962cohomology}, and one can interpret \cref{t:main} as a continuation of these efforts capturing the isomorphism type, and not just the homotopy type, of Steenrod's original construction (\cref{s:others}).

\begin{remark}[Relations]
	Cup-$i$ products provide an effective construction of coboundaries coherently enforcing the commutativity relation of the cup product in cohomology.
	There are two notable relations satisfied by Steenrod squares.
	The first one, known as the \textit{Cartan relation}, expresses the interaction between these operations and the cup product:
	\begin{equation*}
	\Sq^k \big( [\alpha] [\beta] \big) =
	\sum_{i+j=k} \Sq^i \big( [\alpha]\big)\, Sq^j \big([\beta] \big),
	\end{equation*}
	whereas the second, the \textit{Adem relation} \cite{adem1952iteration}, expresses dependencies appearing through iteration:
	\begin{equation*}
	\Sq^i \Sq^j =
	\sum_{k=0}^{\lfloor i/2 \rfloor} \binom{j-k-1}{i-2k} \Sq^{i+j-k} \Sq^k.
	\end{equation*}
	Explicit cochains enforcing them were recently given in \cite{medina2020cartan} and \cite{medina2021adem} respectively using the canonical cup-$i$ construction.
\end{remark}

\begin{remark}[Odd primes]
	Steenrod squares are parameterized by classes on the mod $2$ homology of $\Sym_2$.
	Steenrod used this group homology viewpoint to non-constructively define operations on the mod $p$ cohomology of spaces \cite{steenrod1952reduced, steenrod1953cyclic, steenrod1962cohomology} for any prime $p$.
	To define these constructively for odd primes, explicit cup-$(p,i)$ constructions were introduced in \cite{medina2021may_st} (for simplicial and cubical sets) using May's operadic viewpoint \cite{may1970general}.
	These were implemented in the computer algebra system \texttt{ComCH} \cite{medina2021comch}.
\end{remark}

\begin{remark}[Transverse intersections]
	From a geometric viewpoint, the cup product can be interpreted in terms of intersections of cycles in certain cases.
	For any space, Thom showed that every mod $2$ homology class is represented by the push-forward of the fundamental class of a closed manifold $W$ along some map to the space.
	Furthermore, if the target $M$ is a closed manifold, and therefore satisfies Poincar\'{e} duality
	\[
	PD \colon H^k(M ;\Ftwo) \to H_{\bars{M}-k}(M; \Ftwo),
	\]
	the cohomology class dual to the homology class represented by the intersection of two transverse maps $V \to X$ and $W \to M$, or more precisely their pull-back $W \times_M V \to M$, is the cohomology class $[\alpha] [\beta]$ where $[\alpha]$ and $[\beta]$ are respectively dual to the homology classes represented by $V \to M$ and $W \to M$.
	By taking $[\alpha] = [\beta]$ we have that $\Sq^k \big( [\alpha] \big)$ with $\alpha$ of degree $-k$ is represented by the transverse self-intersection of $W \to M$, that is, the intersection of this map and a generic perturbation of itself.
	In manifold topology, the relationship at the (co)homology level between cup product and intersection is classical.
	For a comparison between these at the level of (co)chain see \cite{medina2021flowing}.
	A generalization of this result to cup-$i$ products is the focus of current research \cite{medina2022foundations}
\end{remark}