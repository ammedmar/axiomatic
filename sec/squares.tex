% !TEX root = ../axiomatic.tex

\section{Steenrod squares} \label{s:squares}

%Let us consider a cup-$i$ construction $W \ot C_\bullet \to C_\bullet^{\ot 2}$.
%Using the linear duality functor and passing to fix points it gives a chain map
%\begin{equation*}
%\begin{tikzcd}
%\Hom\left(C_\bullet \ot C_\bullet, \Ftwo \right)^{\Sym_2} \arrow[r] &
%\Hom\left(W \ot C_\bullet, \Ftwo \right)^{\Sym_2},
%\end{tikzcd}
%\end{equation*}
%which we can complete, using isomorphisms \eqref{e:adjunction isomorphism} and \eqref{e:invariant hom iso hom coinvariants} of \cref{s:preliminaries}, to a commutative diagram
%\begin{equation*}
%\begin{tikzcd}
%\Hom\left(C_\bullet \ot C_\bullet, \Ftwo \right)^{\Sym_2} \arrow[r] &
%\Hom\left(W \ot C_\bullet, \Ftwo \right)^{\Sym_2} \arrow[d] \\
%\left(C^\bullet \ot C^\bullet\right)^{\Sym_2} \arrow[u]&
%\Hom\left(W_{\Sym_2} \ot C_\bullet, \Ftwo \right) \arrow[d] \\
%C^\bullet \arrow[u, "doubleing"] \arrow[r, dashed]&
%\Hom\left(W_{\Sym_2}, C^\bullet\right),
%\end{tikzcd}
%\end{equation*}
%where the choice of coefficients ensures that the \textit{doubleing map} $\alpha \mapsto \alpha \ot \alpha$ is linear.
%Using the adjunction isomorphism, the dashed arrow defines a linear map
%\begin{equation} \label{e:Steenrod squares parameterized}
%\begin{tikzcd}[row sep=0pt, column sep = small]
%C^\bullet \ot W_{\Sym_2} \arrow[r] &[-10pt] C^\bullet \\
%\alpha \ot e_i \arrow[r, |->] & (\alpha \ot \alpha)\triangle_i(-)
%\end{tikzcd}
%\end{equation}
%descending to mod $2$ cohomology.
%The Steenrod squares are defined by reindexing this map.
%Explicitly,
%\begin{definition} \label{d:steenrod squares}
%	The \textit{$k^\th$ Steenrod square} is defined by
%	\begin{equation} \label{e:steenrod squares}
%	\begin{tikzcd}[row sep=0pt, column sep=tiny]
%	Sq^k \colon H^{-n} \arrow[r] & H^{-n-k} \\
%	\phantom{Sq^k \colon}{[\alpha]} \arrow[r, |->] & \big[ (\alpha \ot \alpha)\triangle_{n-k}(-) \big].
%	\end{tikzcd}
%	\end{equation}
%	for any cup-$i$ construction $\triangle$.
%\end{definition}
%
%\subsection{Additional comments}

%\begin{remark}[Transverse intersections]
%	From a geometric viewpoint, the cup product can be interpreted in terms of intersections of cycles in certain cases.
%	For any space, Thom showed that every mod $2$ homology class is represented by the push-forward of the fundamental class of a closed manifold $W$ along some map to the space.
%	Furthermore, if the target $M$ is a closed manifold, and therefore satisfies Poincar\'{e} duality
%	\[
%	PD \colon H^k(M ;\Ftwo) \to H_{\bars{M}-k}(M; \Ftwo),
%	\]
%	the cohomology class dual to the homology class represented by the intersection of two transverse maps $V \to X$ and $W \to M$, or more precisely their pull-back $W \times_M V \to M$, is the cohomology class $[\alpha] [\beta]$ where $[\alpha]$ and $[\beta]$ are respectively dual to the homology classes represented by $V \to M$ and $W \to M$.
%	By taking $[\alpha] = [\beta]$ we have that $Sq^k \big( [\alpha] \big)$ with $\alpha$ of degree $-k$ is represented by the transverse self-intersection of $W \to M$, that is, the intersection of this map and a generic perturbation of itself.
%	In manifold topology, the relationship at the (co)homology level between cup product and intersection is classical.
%	For a comparison between these at the level of (co)chain see \cite{medina2021flowing}.
%	A generalization of this result to cup-$i$ products is the focus of current research \cite{medina2022foundations}
%\end{remark}
%
%\begin{remark}[Odd primes]
%	For the reader familiar with group homology, we remark that Steenrod squares are parameterized by classes on the mod $2$ homology of $\Sym_2$.
%	Steenrod used this group homology viewpoint to non-constructively define operations on the mod $p$ cohomology of spaces \cite{steenrod1952reduced, steenrod1953cyclic, steenrod1962cohomology} for any prime $p$.
%	To define these constructively, analogues of explicit cup-$i$ coproducts for odd primes were introduced in \cite{medina2021maysteenrod} using May's operadic viewpoint \cite{may1970general} and implemented in the computer algebra system \texttt{ComCH} \cite{medina2021computer}.
%\end{remark}